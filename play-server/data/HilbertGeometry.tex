%%%%%%%%%%%%%%%%%%%%%%%%%%%%%%%%%%%%%%%%%%%%%%%%%%%%%%%%%%%%%%%%%%%%%%%%%%%
%% Project Gutenberg's The Foundations of Geometry                       %%
%% by David Hilbert                                                      %%
%%                                                                       %%
%% This eBook is for the use of anyone anywhere at no cost and with      %%
%% almost no restrictions whatsoever.  You may copy it, give it away or  %%
%% re-use it under the terms of the Project Gutenberg License included   %%
%% with this eBook or online at www.gutenberg.net                        %%
%%                                                                       %%
%% Packages and substitutions:                                           %%
%%                                                                       %%
%% book:     Basic book class. Required.                                 %%
%% amsmath:  Basic AMS math. Required.                                   %%
%% amssymb:  Basic AMS symbols. Required.                                %%
%% tabularx: For making tables. Required.                                %%
%% wrapfig:  For placing images within text. Required.                   %%
%% graphicx: Basic graphics for images. Required.                        %%
%% geometry: Set margins.                                                %%
%%                                                                       %%
%% Things to Check:                                                      %%
%%                                                                       %%
%% Spellcheck: OK                                                        %%
%% LaCheck: OK                                                           %%
%% Lprep/gutcheck: OK                                                    %%
%% PDF pages, excl. Gutenberg boilerplate: 92                            %%
%% PDF pages, incl. Gutenberg boilerplate: 101                           %%
%% ToC page numbers: OK                                                  %%
%% Images: 52 png                                                        %%
%% Fonts: OK                                                             %%
%%                                                                       %%
%% Compile History:                                                      %%
%%                                                                       %%
%% Dec 20 05: rfrank47                                                   %%
%%            This LaTeX document was processed with pdflatex, invoking  %%
%%            pdfeTeX, Version 3.141592-1.20a-rc4-2.1 (MiKTeX 2.4)       %%
%%            The Miktex compiler was run three times, until the Table   %%
%%            of Contents was complete.                                  %%
%%            Fifty-two illustrations (fig001.png through fig052.png)    %%
%%            are included as .eps files in the images directory.        %%
%%                                                                       %%
%% Dec 23 05: jt                                                         %%
%%            pdflatex hilbert                                           %%
%%            pdflatex hilbert                                           %%
%%            pdflatex hilbert                                           %%
%%                                                                       %%
%%                                                                       %%
%%%%%%%%%%%%%%%%%%%%%%%%%%%%%%%%%%%%%%%%%%%%%%%%%%%%%%%%%%%%%%%%%%%%%%%%%%%

\documentclass[11pt,oneside]{book}
\listfiles
\usepackage{amsmath}
\usepackage{amssymb}
\usepackage{tabularx}
\usepackage{wrapfig}
\usepackage{graphicx}

\usepackage{geometry}
\geometry{margin=1.3in}

\begin{document}
\newcommand{\rfa}[1]{\subsection*{\begin{center}#1\end{center}}}
\newcommand{\rfb}[0]{\hspace{-3.2mm}}

\thispagestyle{empty}
\small
\begin{verbatim}
Project Gutenberg's The Foundations of Geometry,
by David Hilbert

This eBook is for the use of anyone anywhere at no cost and with
almost no restrictions whatsoever.  You may copy it, give it away or
re-use it under the terms of the Project Gutenberg License included
with this eBook or online at www.gutenberg.net


Title: The Foundations of Geometry

Author: David Hilbert

Release Date: December 23, 2005 [EBook #17384]

Language: English

Character set encoding: TeX

*** START OF THIS PROJECT GUTENBERG EBOOK FOUNDATIONS OF GEOMETRY ***




Produced by Joshua Hutchinson, Roger Frank, David Starner and
the Online Distributed Proofreading Team at http://www.pgdp.net



\end{verbatim}
\normalsize
\newpage

\frontmatter
\pagestyle{empty}
%%-----File: 002.png-----%%

\begin{titlepage}
\begin{center}
\vspace*{1cm}
{\Large The}\\[4mm]
{\Huge Foundations of Geometry}\\[12mm]
{\small BY}\\[3mm]
{\large DAVID HILBERT, PH\@. D.}\\[3mm]
{\tiny PROFESSOR OF MATHEMATICS, UNIVERSITY OF G\"OTTINGEN}\\[3cm]
{\small AUTHORIZED TRANSLATION}\\[3mm]
BY\\[3mm]
{\large E.~J. TOWNSEND, PH.\ D.}\\[3mm]
{\tiny UNIVERSITY OF ILLINOIS}\\[4cm]
REPRINT EDITION\\[1cm]
THE OPEN COURT PUBLISHING COMPANY\\[4mm]
LA SALLE \hspace*{4cm} ILLINOIS\\[4mm]
1950
\end{center}
\end{titlepage}

%%-----File: 003.png-----%%
\mainmatter
\vspace*{7cm}
\begin{center}
{\tiny TRANSLATION COPYRIGHTED}\\[3mm]
{\tiny BY}\\[3mm]
{\small \scshape{The Open Court Publishing Co.}}\\[3mm]
{\tiny 1902.}
%%-----File: 004.png-----%%
\end{center}
\clearpage
\begin{center}{\Large PREFACE.}\end{center}
\medskip
\begin{small}
The material contained in the following translation was given
in substance by Professor Hilbert as a course of lectures on
euclidean geometry at the University of G\"ottingen during the
winter semester of 1898--1899. The results of his investigation
were re-arranged and put into the form in which they appear here
as a memorial address published in connection with the celebration
at the unveiling of the Gauss-Weber monument at G\"ottingen,
in June, 1899. In the French edition, which appeared soon after,
Professor Hilbert made some additions, particularly in the concluding
remarks, where he gave an account of the results of a recent
investigation made by Dr. Dehn. These additions have been
incorporated in the following translation.

As a basis for the analysis of our intuition of space, Professor
Hilbert commences his discussion by considering three systems of
things which he calls points, straight lines, and planes, and sets
up a system of axioms connecting these elements in their mutual
relations. The purpose of his investigations is to discuss systematically
the relations of these axioms to one another and also the
bearing of each upon the logical development of euclidean geometry.
Among the important results obtained, the following are
worthy of special mention:

1. The mutual independence and also the compatibility of the
given system of axioms is fully discussed by the aid of various new
systems of geometry which are introduced.

2. The most important propositions of euclidean geometry are
demonstrated in such a manner as to show precisely what axioms
underlie and make possible the demonstration.

3. The axioms of congruence are introduced and made the
basis of the definition of geometric displacement.

4. The significance of several of the most important axioms
and theorems in the development of the euclidean geometry is
clearly shown; for example, it is shown that the whole of the
%%-----File: 005.png-----%%
euclidean geometry may be developed without the use of the axiom
of continuity; the significance of Desargues's theorem, as a condition
that a given plane geometry may be regarded as a part of a
geometry of space, is made apparent, etc.

5. A variety of algebras of segments are introduced in accordance
with the laws of arithmetic.

This development and discussion of the foundation principles
of geometry is not only of mathematical but of pedagogical importance.
Hoping that through an English edition these important
results of Professor Hilbert's investigation may be made more
accessible to English speaking students and teachers of geometry,
I have undertaken, with his permission, this translation. In its
preparation, I have had the assistance of many valuable suggestions
from Professor Osgood of Harvard, Professor Moore of Chicago,
and Professor Halsted of Texas. I am also under obligations
to Mr. Henry Coar and Mr. Arthur Bell for reading the
proof.

\begin{scshape}
\begin{flushright}
E. J. Townsend
\end{flushright}
University of Illinois.
\end{scshape}
\end{small}
\clearpage

%%-----File: 006.png-----%%
\thispagestyle{empty}
\setcounter{page}{0}
\newpage
\rfa{CONTENTS}
\bigskip
\begin{tabular}[\linewidth]{l@{}rl@{\ }r}
&&& \parbox{.05\textwidth}{\footnotesize PAGE} \\
\multicolumn{3}{l}{Introduction~\dotfill}&\pageref{p0}\\
\multicolumn{4}{c}{}\\
\multicolumn{4}{c}{CHAPTER I.}\\
\multicolumn{4}{c}{\small THE FIVE GROUPS OF AXIOMS.}\\
\multicolumn{4}{c}{}\\
\S &1. &The elements of geometry and the five groups of axioms~\dotfill &\pageref{p1} \\
\S &2. &Group I: Axioms of connection~\dotfill &\pageref{p2}\\
\S &3. &Group II: Axioms of Order~\dotfill &\pageref{p3}\\
\S &4. &Consequences of the axioms of connection and order~\dotfill &\pageref{p4}\\
\S &5. &Group III: Axiom of Parallels (Euclid's axiom)~\dotfill &\pageref{p5}\\
\S &6. &Group IV: Axioms of congruence~\dotfill &\pageref{p6}\\
\S &7. &Consequences of the axioms of congruence~\dotfill &\pageref{p7}\\
\S &8. &Group V: Axiom of Continuity (Archimedes's axiom)~\dotfill &\pageref{p8}\\
\multicolumn{4}{c}{}\\
\multicolumn{4}{c}{CHAPTER II.}\\
\multicolumn{4}{c}{\small THE COMPATIBILITY AND MUTUAL INDEPENDENCE OF THE AXIOMS.}\\
\multicolumn{4}{c}{}\\
\S &9. &Compatibility of the axioms~\dotfill &\pageref{p9}\\
\S &10. &Independence of the axioms of parallels. Non-euclidean geometry~\dotfill &\pageref{p10}\\
\S &11. &Independence of the axioms of congruence~\dotfill &\pageref{p11}\\
\S &12. &Independence of the axiom of continuity. Non-archimedean geometry~\dotfill &\pageref{p12}\\
\multicolumn{4}{c}{}\\
\multicolumn{4}{c}{CHAPTER III.}\\
\multicolumn{4}{c}{\small THE THEORY OF PROPORTION.}\\
\multicolumn{4}{c}{}\\
\S &13. &Complex number-systems~\dotfill &\pageref{p13}\\
\S &14. &Demonstration of Pascal's theorem~\dotfill &\pageref{p14}\\
\S &15. &An algebra of segments, based upon Pascal's theorem~\dotfill &\pageref{p15}\\
\S &16. &Proportion and the theorems of similitude~\dotfill &\pageref{p16}\\
\S &17. &Equations of straight lines and of planes~\dotfill &\pageref{p17}\\
\multicolumn{4}{c}{}\\
\multicolumn{4}{c}{CHAPTER IV.}\\
\multicolumn{4}{c}{\small THE THEORY OF PLANE AREAS.}\\
\multicolumn{4}{c}{}\\
\S &18. &Equal area and equal content of polygons~\dotfill &\pageref{p18}\\
\S &19. &Parallelograms and triangles having equal bases and equal altitudes~\dotfill &\pageref{p19}\\
\S &20. &The measure of area of triangles and polygons~\dotfill &\pageref{p20}\\
\S &21. &Equality of content and the measure of area~\dotfill &\pageref{p21}\\
\end{tabular}
\newpage
\begin{tabular}{l@{}rl@{\ }r}
\multicolumn{4}{c}{}\\
\multicolumn{4}{c}{CHAPTER V.}\\
\multicolumn{4}{c}{\small DESARGUES'S THEOREM.}\\
\multicolumn{4}{c}{}\\
\S &22. &Desargues's theorem and its demonstration for plane geometry&\\
&& by aid of the axioms of congruence~\dotfill &\pageref{p22}\\
\S &23. &The impossibility of demonstrating Desargues's theorem for the&\\
&& plane without the help of the axioms of congruence~\dotfill &\pageref{p23}\\
\S &24. &Introduction of an algebra of segments based upon Desargues's theorem&\\
&& and independent of the axioms of congruence~\dotfill &\pageref{p24}\\
\S &25. &The commutative and the associative law of addition for our new&\\
&& algebra of segments~\dotfill &\pageref{p25}\\
\S &26. &The associative law of multiplication and the two distributive laws&\\
&& for the new algebra of segments~\dotfill &\pageref{p26}\\
\S &27. &Equation of the straight line, based upon the new algebra of segments~\dotfill &\pageref{p27}\\
\S &28. &The totality of segments, regarded as a complex number system~\dotfill &\pageref{p28}\\
\S &29. &Construction of a geometry of space by aid of a&\\
&& desarguesian number system~\dotfill &\pageref{p29}\\
\S &30. &Significance of Desargues's theorem~\dotfill &\pageref{p30}\\
\multicolumn{4}{c}{}\\
\multicolumn{4}{c}{CHAPTER VI.}\\
\multicolumn{4}{c}{\small PASCAL'S THEOREM.}\\
\multicolumn{4}{c}{}\\
\S &31. &Two theorems concerning the possibility of proving Pascal's theorem~\dotfill &\pageref{p31}\\
\S &32. &The commutative law of multiplication for an&\\
&& archimedean number system~\dotfill &\pageref{p32}\\
\S &33. &The commutative law of multiplication for a&\\
&& non-archimedean number system~\dotfill &\pageref{p33}\\
\S &34. &Proof of the two propositions concerning Pascal's theorem.&\\
&& Non-pascalian geometry.~\dotfill &\pageref{p34}\\
\S &35. &The demonstration, by means of the theorems of Pascal and Desargues,&\\
&& of any theorem relating to points of intersection~\dotfill &\pageref{p35}\\
\multicolumn{4}{c}{}\\
\multicolumn{4}{c}{CHAPTER VII.}\\
\multicolumn{4}{c}{\small GEOMETRICAL CONSTRUCTIONS BASED UPON THE AXIOMS I--V.}\\
\multicolumn{4}{c}{}\\
\S &36. &Geometrical constructions by means of a straight-edge and a&\\
&& transferer of segments~\dotfill &\pageref{p36}\\
\S &37. &Analytical representation of the co-ordinates of points&\\
&& which can be so constructed~\dotfill &\pageref{p37}\\
\S &38. &The representation of algebraic numbers and of integral rational functions&\\
&& as sums of squares~\dotfill &\pageref{p38}\\
\S &39. &Criterion for the possibility of a geometrical construction by means of&\\
&& a straight-edge and a transferer of segments~\dotfill &\pageref{p39}\\
\multicolumn{3}{l}{Conclusion~\dotfill}&\pageref{pc}\\
\end{tabular}

\clearpage
\pagenumbering{arabic}
\thispagestyle{empty}
%%-----File: 010.png-----%%
\label{folio1}
\begin{flushright}
  \parbox{3in}{\setlength{\parindent}{1em}
  ``All human knowledge begins with intuitions,
  thence passes to concepts and
  ends with ideas.''}
\end{flushright}
\begin{flushright}
  \parbox{2.5in}{\setlength{\parindent}{1em}
  Kant, \emph{Kritik der reinen Vernunft,}
  \emph{Elementariehre}, Part 2, Sec. 2.}
\end{flushright}

\pagestyle{headings}
\rfa{INTRODUCTION.}\label{p0}

Geometry, like arithmetic, requires for its logical
development only a small number of simple,
fundamental principles. These fundamental principles
are called the axioms of geometry. The choice
of the axioms and the investigation of their relations
to one another is a problem which, since the time of
Euclid, has been discussed in numerous excellent
memoirs to be found in the mathematical literature.\footnote{Compare
the comprehensive and explanatory report of G. Veronese,
\textit{Grundz\"uge der Geometrie}, German translation by A. Schepp, Leipzig, 1894
(Appendix). See also F. Klein, ``Zur ersten Verteilung des Lobatschefskiy-Preises,''
\emph{Math. Ann.}, Vol. 50.}
This problem is tantamount to the logical analysis of
our intuition of space.

The following investigation is a new attempt to
choose for geometry a \emph{simple} and \emph{complete} set of \emph{independent}
axioms and to deduce from these the most important
geometrical theorems in such a manner as to
bring out as clearly as possible the significance of the
different groups of axioms and the scope of the conclusions
to be derived from the individual axioms.
\newpage
%%-----File: 012.png-----%%
\begin{center}{\Large THE FIVE GROUPS OF AXIOMS.}\end{center}
\rfa{\S\!~1.\ {\small THE ELEMENTS OF GEOMETRY AND THE FIVE GROUPS OF AXIOMS.}}\label{p1}

Let us consider three distinct systems of things.
The things composing the first system, we will
call \emph{points} and designate them by the letters $A$, $B$,
$C$,\ldots; those of the second, we will call \emph{straight
lines} and designate them by the letters $a$, $b$, $c$,\ldots;
and those of the third system, we will call \emph{planes} and
designate them by the Greek letters $\alpha$, $\beta$, $\gamma$,\ldots
The points are called the \emph{elements of linear geometry};
the points and straight lines, the \emph{elements of plane geometry};
and the points, lines, and planes, the \emph{elements
of the geometry of space} or the \emph{elements of space}.

We think of these points, straight lines, and planes
as having certain mutual relations, which we indicate
by means of such words as ``are situated,'' ``between,''
``parallel,'' ``congruent,'' ``continuous,'' etc.
The complete and exact description of these relations
follows as a consequence of the \emph{axioms of geometry}.
These axioms may be arranged in five groups. Each
of these groups expresses, by itself, certain related
fundamental facts of our intuition. We will name
these groups as follows:

\begin{tabular}{rcl}
I,&1--7.&Axioms of \emph{connection}.\\
II,&1--5.&Axioms of \emph{order}.\\
III.&&Axiom of \emph{parallels} (Euclid's axiom).\\
%%-----File: 013.png-----%%
IV,&1--6.&Axioms of \emph{congruence}.\\
V.&&Axiom of \emph{continuity} (Archimedes's axiom).\\
\end{tabular}

\rfa{\S\!~2.\ {\small GROUP I: AXIOMS OF CONNECTION.}}\label{p2}

The axioms of this group establish a connection
between the concepts indicated above; namely, points,
straight lines, and planes. These axioms are as follows:

\begin{description}
\item[I, 1.] \emph{Two distinct points $A$ and $B$ always completely
determine a straight line $a$. We write $AB = a$
or $BA = a$}.
\end{description}

Instead of ``determine,'' we may also employ other
forms of expression; for example, we may say $A$
``lies upon'' $a$, $A$ ``is a point of'' $a$, $a$ ``goes through''
$A$ ``and through'' $B$, $a$ ``joins'' $A$ ``and'' or ``with''
$B$, etc. If $A$ lies upon $a$ and at the same time upon
another straight line $b$, we make use also of the expression:
``The straight lines'' $a$ ``and'' $b$ ``have the
point $A$ in common,'' etc.

\begin{description}
\item[I, 2.] \emph{Any two distinct points of a straight line completely
determine that line; that is, if $AB = a$ and
$AC=a$, where $B \neq C$, then is also $BC=a$}.

\item[I, 3.] \emph{Three points $A$, $B$, $C$ not situated in the same
straight line always completely determine a plane
$\alpha$. We write $ABC=a$}.
\end{description}

We employ also the expressions: $A$, $B$, $C$, ``lie
in'' $a$; $A$, $B$, $C$ ``are points of'' $a$, etc.

\begin{description}
\item[I, 4.] \emph{Any three points $A$, $B$, $C$ of a plane $\alpha$, which
do not lie in the same straight line, completely determine
that plane}.

\item[I, 5.] \emph{If two points $A$, $B$ of a straight line $a$ lie in
a plane $\alpha$, then every point of $a$ lies in $a$}.
\end{description}
%%-----File: 014.png-----%%
In this case we say: ``The straight line a lies in
the plane $\alpha$,'' etc.

\begin{description}
\item[I, 6.] \emph{If two planes $\alpha$, $\beta$ have a point $A$ in common,
then they have at least a second point $B$ in common.}

\item[I, 7.] \emph{Upon every straight line there exist at least two
points, in every plane at least three points not
lying in the same straight line, and in space there
exist at least four points not lying in a plane.}
\end{description}

Axioms I, 1--2 contain statements concerning points
and straight lines only; that is, concerning the elements
of plane geometry. We will call them, therefore,
the \emph{plane axioms of group I}, in order to distinguish
them from the axioms I, 3--7, which we will
designate briefly as the \emph{space axioms} of this group.

Of the theorems which follow from the axioms
I, 3--7, we shall mention only the following:

\begin{itemize}
\item[]{\scshape Theorem 1.} Two straight lines of a plane have
either one point or no point in common; two
planes have no point in common or a straight
line in common; a plane and a straight line
not lying in it have no point or one point in
common.

\item[]{\scshape Theorem 2.} Through a straight line and a point
not lying in it, or through two distinct straight
lines having a common point, one and only one
plane may be made to pass.
\end{itemize}

\rfa{\S\!~3.\ {\small GROUP II: AXIOMS OF ORDER.}\footnote{These
axioms were first studied in detail by M.~Pasch in his \emph{Vorlesungen
\"uber neuere Geometrie}, Leipsic, 1882. Axiom II, 5 is in particular due to him.}}\label{p3}

The axioms of this group define the idea expressed
by the word ``between,'' and make possible, upon the
%%-----File: 015.png-----%%
basis of this idea, an \emph{order of sequence} of the points
upon a straight line, in a plane, and in space. The
points of a straight line have a certain relation to one
another which the word ``between'' serves to describe.
The axioms of this group are as follows:

\begin{description}
\item[II, 1.] \emph{If $A$, $B$, $C$ are points of a straight line and
$B$ lies between $A$ and $C$, then $B$ lies also between
$C$ and $A$.}
\medskip
\begin{figure}[!htb]
\begin{center}
\includegraphics*[width=4in]{images/f001.png}\\
{\small Fig. 1.}
\end{center}
\end{figure}
\medskip
\item[II, 2.] \emph{If $A$ and $C$ are two points of a straight line,
then there exists at least one point $B$ lying between
$A$ and $C$ and at least one point $D$ so situated that
$C$ lies between $A$ and $D$.}
\medskip
\begin{figure}[!htb]
\begin{center}
\includegraphics*[width=4in]{images/f002.png}\\
{\small Fig. 2.}
\end{center}
\end{figure}
\medskip
\item[II, 3.] \emph{Of any three points situated on a straight line,
there is always one and only one which lies between
the other two.}

\item[II, 4.] \emph{Any four points $A$, $B$, $C$, $D$ of a straight line
can always be so arranged that $B$ shall lie between
$A$ and $C$ and also between $A$ and $D$, and, furthermore,
that $C$ shall lie between $A$ and $D$ and also
between $B$ and $D$}.
\end{description}

{\scshape Definition}. We will call the system of two points
$A$ and $B$, lying upon a straight line, a \emph{segment} and
denote it by $AB$ or $BA$. The points lying between $A$
and $B$ are called the \emph{points of the segment $AB$} or the
\emph{points lying within the segment $AB$}. All other points
of the straight line are referred to as the \emph{points lying}
%%-----File: 016.png-----%%
\emph{outside the segment $AB$}. The points $A$ and $B$ are called
the \emph{extremities} of the segment $AB$.

\begin{figure}[htb]
\begin{center}
\includegraphics*[width=3in]{images/f003.png}\\
{\small Fig. 3.}
\end{center}
\end{figure}

\begin{description}
\item[II, 5.] \emph{Let $A$, $B$, $C$ be three points not lying in the
same straight line and
let $a$ be a straight line lying in the plane $ABC$ and not passing through any of the
points $A$, $B$, $C$. Then, if the straight line $a$ passes through a point
of the segment $AB$, it will also pass through either a point of the segment
$BC$ or a point of the segment $AC$.}
Axioms II, 1--4 contain statements concerning the
points of a straight line only, and, hence, we will call
them the \emph{linear axioms of group II}. Axiom II, 5 relates
to the elements of plane geometry and, consequently,
shall be called the \emph{plane axiom of group II}.
\end{description}

\rfa{\S\!~4.\ {\small CONSEQUENCES OF THE AXIOMS OF CONNECTION AND ORDER.}}\label{p4}

By the aid of the four linear axioms II, 1--4, we
can easily deduce the following theorems:

\begin{itemize}
\item[]{\scshape Theorem 3.} Between any two points of a straight
line, there always exists an unlimited number of
points.

\item[]{\scshape Theorem 4.} If we have given any finite number
of points situated upon a straight line, we can
always arrange them in a sequence $A$, $B$, $C$,
$D$, $E$,$\ldots$, $K$ so that $B$ shall lie between $A$
and $C$, $D$, $E$,$\ldots$, $K$; $C$ between $A$, $B$ and $D$,
%%-----File: 017.png-----%%
$E$,\ldots, $K$; $D$ between $A$, $B$, $C$ and $E$,\ldots$K$,
etc. Aside from this order of sequence, there
exists but one other possessing this property
namely, the reverse order $K$,\ldots, $E$, $D$, $C$,
$B$, $A$.

\begin{figure}[htb]
\begin{center}
\includegraphics*[width=4in]{images/f004.png}\\
{\small Fig. 4.}
\end{center}
\end{figure}

\item[]{\scshape Theorem 5.} Every straight line $a$, which lies in
a plane $\alpha$, divides the remaining points of this
plane into two regions having the following
properties: Every point $A$ of the one region determines
with each point $B$ of the other region
a segment $AB$ containing a point of the straight
line $a$. On the other hand, any two points $A$,
$A'$ of the same region determine a segment
$AA'$ containing no point of $a$.
\end{itemize}

\begin{figure}[htb]
\begin{center}
\includegraphics*[width=3in]{images/f005.png}\\
{\small Fig. 5.}
\end{center}
\end{figure}

If $A$, $A'$, $O$, $B$ are four points of a straight line $a$,
where $O$ lies between $A$ and $B$ but not between $A$ and
$A'$, then we may say: The points $A$, $A'$ are situated
\emph{on the line a upon one and the same side of the point $O$,}
%%-----File: 018.png-----%%
and the points $A$, $B$ are situated \emph{on the straight line $a$
upon different sides of the point $O$}.

\begin{figure}[htb]
\begin{center}
\includegraphics*[width=3in]{images/f006.png}\\
{\small Fig. 6.}
\end{center}
\end{figure}

All of the points of
$a$ which lie upon the same side of $O$, when taken
together, are called the \emph{half-ray} emanating from $O$.
Hence, each point of a straight line divides it into
two half-rays.

Making use of the notation of theorem $5$, we say:
The points $A$, $A'$ lie \emph{in the plane $\alpha$ upon one and the
same side of the straight line $a$}, and the points $A$, $B$ lie
\emph{in the plane $\alpha$ upon different sides of the straight line $a$}.

{\scshape Definitions}. A system of segments $AB$, $BC$,
$CD$, \ldots, $KL$ is called a \emph{broken line} joining $A$ with $L$
and is designated, briefly, as the broken line $ABCDE$ \ldots
$KL$. The points lying within the segments $AB$,
$BC$, $CD$, \ldots, $KL$, as also the points $A$, $B$, $C$, $D$, \ldots,
$K$, $L$, are called \emph{the points of the broken line}. In
particular, if the point $A$ coincides with $L$, the broken
line is called a \emph{polygon} and is designated as the polygon
$ABCD \ldots K$. The segments $AB$, $BC$, $CD$, \ldots, $KA$
are called the \emph{sides of the polygon} and the points $A$, $B$,
$C$, $D$, \ldots, $K$ the \emph{vertices}. Polygons having $3$, $4$,
$5$, \ldots, $n$ vertices are called, respectively, {\textit triangles},
{\textit quadrangles}, {\textit pentagons}, \ldots, {\textit n-gons}. If the vertices of
a polygon are all distinct and none of them lie within
the segments composing the sides of the polygon,
and, furthermore, if no two sides have a point in common,
then the polygon is called a \emph{simple polygon}.

With the aid of theorem $5$, we may now obtain,
without serious difficulty, the following theorems:

\begin{itemize}
\item[]{\scshape Theorem} 6. Every simple polygon, whose vertices
all lie in a plane $\alpha$, divides the points of
this plane, not belonging to the broken line
constituting the sides of the polygon, into two
%%-----File: 019.png-----%%
regions, an interior and an exterior, having the
following properties: If $A$ is a point of the interior
region (interior point) and $B$ a point of
the exterior region (exterior point), then any
broken line joining $A$ and $B$ must have at least
one point in common with the polygon. If, on
the other hand, $A$, $A'$ are two points of the interior
and $B$, $B'$ two points of the exterior region,
then there are always broken lines to be
found joining $A$ with $A'$ and $B$ with $B'$ without
having a point in common with the polygon.
There exist straight lines in the plane $\alpha$ which
lie entirely outside of the given polygon, but
there are none which lie entirely within it.

\begin{figure}[htb]
\begin{center}
\includegraphics*[width=3in]{images/f007.png}\\
{\small Fig. 7.}
\end{center}
\end{figure}

\item[]{\scshape Theorem} 7. Every plane $\alpha$ divides the remaining
points of space into two regions having the
following properties: Every point $A$ of the one
region determines with each point $B$ of the
other region a segment $AH$, within which lies
a point of $\alpha$. On the other hand, any two points
$A$, $A'$ lying within the same region determine a
segment $AA'$ containing no point of $\alpha$.
\end{itemize}
%%-----File: 020.png-----%%
Making use of the notation of theorem 7, we may
now say: The points $A$, $A'$ are situated in space \emph{upon
one and the same side of the plane $\alpha$}, and the points $A$, $B$
are situated in space \emph{upon different sides of the plane $\alpha$}.

Theorem 7 gives us the most important facts relating
to the order of sequence of the elements of
space. These facts are the results, exclusively, of the
axioms already considered, and, hence, no new space
axioms are required in group II\@.

\rfa{\S\!~5.\ {\small GROUP III: AXIOM OF PARALLELS. (EUCLID'S AXIOM.)}}\label{p5}

The introduction of this axiom simplifies greatly
the fundamental principles of geometry and facilitates
in no small degree its development. This axiom may
be expressed as follows:

\begin{itemize}
\item[ ]III\@. \emph{In a plane $\alpha$ there can be drawn through any
point $A$, lying outside of a straight line $a$, one and
only one straight line which does not intersect the
line $a$. This straight line is called the parallel to
$a$ through the given point $A$.}
\end{itemize}

This statement of the axiom of parallels contains
two assertions. The first of these is that, in the plane
$\alpha$, there is always a straight line passing through $A$
which does not intersect the given line $a$. The second
states that only one such line is possible. The latter
of these statements is the essential one, and it may
also be expressed as follows:

\begin{itemize}
\item[]{\scshape Theorem 8.} If two straight lines $a$, $b$ of a plane
do not meet a third straight line $c$ of the same
plane, then they do not meet each other.
\end{itemize}

For, if $a$, $b$ had a point $A$ in common, there would
then exist in the same plane with $c$ two straight lines
%%-----File: 021.png-----%%
$a$ and $b$ each passing through the point $A$ and not
meeting the straight line $c$. This condition of affairs
is, however, contradictory to the second assertion contained
in the axiom of parallels as originally stated.
Conversely, the second part of the axiom of parallels,
in its original form, follows as a consequence of theorem
8.

The axiom of parallels is a \emph{plane axiom}.

\rfa{\S\!~6.\ {\small GROUP IV\@. AXIOMS OF CONGRUENCE.}}\label{p6}

The axioms of this group define the idea of congruence
or displacement.

Segments stand in a certain relation to one another
which is described by the word ``\emph{congruent}.''

\begin{itemize}
\item[ ]IV, I\@. \emph{If $A$, $B$ are two points on a straight line $a$,
and if $A'$ is a point upon the same or another
straight line $a'$, then, upon a given side of $A'$ on
the straight line $a'$, we can always find one and
only one point $B'$ so that the segment $AB$ (or $BA$)
is congruent to the segment $A'B'$. We indicate
this relation by writing }
\[
AB\equiv A'B'.
\]
\emph{Every segment is congruent to itself; that is, we
always have}
\[
AB\equiv AB.
\]
\end{itemize}

We can state the above axiom briefly by saying
that every segment can be \emph{laid off} upon a given side
of a given point of a given straight line in one and
and only one way.

\begin{itemize}
\item[ ]IV, 2. \emph{If a segment $AB$ is congruent to the segment
$A'B'$ and also to the segment $A''B''$, then the segment
$A'B'$ is congruent to the segment $A''B''$; that
%%-----File: 022.png-----%%
is, if $AB \equiv A'B$ and $AB \equiv A''B''$, then
$A'B' \equiv A''B''$.}

\item[ ]IV, 3. \emph{Let $AB$ and $BC$ be two segments of a straight
line $a$ which have no points in common aside from
the point $B$, and, furthermore, let $A'B'$ and $B'C'$
be two segments of the same or of another straight
line $a'$ having, likewise, no point other than $B'$ in
common.  Then, if $AB \equiv A'B'$ and $BC \equiv B'C'$,
we have $AC \equiv A'C'$.}
\end{itemize}

\begin{figure}[htb]
\begin{center}
\includegraphics*[width=3in]{images/f008.png}\\
{\small Fig. 8.}
\end{center}
\end{figure}

{\scshape Definitions}. Let $\alpha$ be any arbitrary plane and $h$,
$k$ any two distinct half-rays lying in $\alpha$ and emanating
from the point $O$ so as to form a part of two different
straight lines. We call the system formed by these
two half-rays $h$, $k$ an \emph{angle} and represent it by the
symbol $\angle(h, k)$ or $\angle(k, h)$. From axioms II, 1--5, it
follows readily that the half-rays $h$ and $k$, taken together
with the point $O$, divide the remaining points
of the plane a into two regions having the following
property: If $A$ is a point of one region and $B$ a point
of the other, then every broken line joining $A$ and $B$
either passes through $O$ or has a point in common
with one of the half-rays $h$, $k$. If, however, $A$, $A'$
both lie within the same region, then it is always possible
to join these two points by a broken line which
neither passes through $O$ nor has a point in common
with either of the half-rays $h$, $k$. One of these two
regions is distinguished from the other in that the segment
%%-----File: 023.png-----%%
joining any two points of this region lies entirely
within the region. The region so characterised is
called the \emph{interior of the angle $(h,k)$}. To distinguish
the other region from this, we call it the \emph{exterior of
the angle $(h,k)$}. The half rays $h$ and $k$ are called the
\emph{sides of the angle}, and the point $O$ is called the \emph{vertex
of the angle}.

\begin{description}
\item[IV, 4.] \emph{Let an angle $(h,k)$ be given in the plane
$\alpha$ and let a straight line $a'$ be given in a plane $\alpha'$.
Suppose also that, in the plane $\alpha$, a definite side
of the straight line $a'$ be assigned. Denote by $h'$ a
half-ray of the straight line $a'$ emanating from a
point $O'$ of this line. Then in the plane $\alpha'$ there
is one and only one half-ray $k'$ such that the angle
$(h,k)$, or $(k,h)$, is congruent to the angle $(h',k')$
and at the same time all interior points of the angle
$(h',k')$ lie upon the given side of $a'$. We express
this relation by means of the notation
\[
   \angle (h,k) \equiv \angle (h',k')
\]
Every angle is congruent to itself; that is,
\[
\angle (h,k) \equiv \angle (h,k)
\]
or
\[
\angle (h,k) \equiv \angle (k,h)
\]
}
\end{description}

We say, briefly, that every angle in a given plane
can be \emph{laid off} upon a given side of a given half-ray in
one and only one way.

\begin{description}
\item[IV, 5.] \emph{If the angle $(h,k)$ is congruent to the angle
$(h',k')$ and to the angle $(h'',k'')$, then the angle
$(h',k')$ is congruent to the angle $(h'',k'')$; that is
to say, if $\angle (h, k) \equiv \angle (h', k')$ and
$\angle (h, k) \equiv \angle (h'',k'')$, then
$\angle (h',k') \equiv \angle (h'',k'')$.}
\end{description}

Suppose we have given a triangle $ABC$. Denote
%%-----File: 024.png-----%%
by $h$, $k$ the two half-rays emanating from $A$ and passing
respectively through $B$ and $C$. The angle $(h,k)$
is then said to be the angle included by the sides $AB$
and $AC$, or the one opposite to the side $BC$ in the
triangle $ABC$. It contains all of the interior points
of the triangle $ABC$ and is represented by the symbol
$\angle BAC$, or by $\angle A$.

\begin{description}
\item[IV, 6.] \emph{If, in the two triangles $ABC$ and $A'B'C'$
the congruences
\[
AB \equiv A'B', \: AC \equiv A'C', \: \angle BAC \equiv \angle B'A'C'
\]
hold, then the congruences
\[
\angle ABC \equiv \angle A'B'C' \:\mbox{and}\; \angle ACB \equiv \angle A'C'B'
\]
also hold.}
\end{description}

Axioms IV, 1--3 contain statements concerning the
congruence of segments of a straight line only. They
may, therefore, be called the \emph{linear} axioms of group
IV\@. Axioms IV, 4, 5 contain statements relating to
the congruence of angles. Axiom IV, 6 gives the connection
between the congruence of segments and the
congruence of angles. Axioms IV, 4--6 contain statements
regarding the elements of plane geometry and
may be called the \emph{plane} axioms of group IV.

\rfa{\S\!~7.\ {\small CONSEQUENCES OF THE AXIOMS OF CONGRUENCE.}}\label{p7}

Suppose the segment $AB$ is congruent to the segment
$A'B'$. Since, according to axiom IV, 1, the segment
$AB$ is congruent to itself, it follows from axiom
IV, 2 that $A'B'$ is congruent to $AB$; that is to say, if
$AB \equiv A'B'$, then $A'B' \equiv AB$. We say, then, that the
two segments are congruent to one another.
%%-----File: 025.png-----%%

Let $A$, $B$, $C$, $D,\ldots,$ $K$, $L$ and $A'$, $B'$, $C'$, $D',\ldots,$
$K'$, $L'$ be two series of points on the straight lines $a$
and $a'$, respectively, so that all the corresponding segments
$AB$ and $A'B'$, $AC$ and $A'C'$, $BC$ and $B'C',\ldots,$
$KL$ and $K'L'$ are respectively congruent, then \emph{the two
series of points are said to be congruent to one another}.
$A$ and $A'$, $B$ and $B',\ldots,$ $L$ and $L'$ are called \emph{corresponding
points} of the two congruent series of points.

From the linear axioms IV, 1--3, we can easily deduce
the following theorems:

\begin{itemize}
\item[]{\scshape Theorem 9.} If the first of two congruent series
of points $A$, $B$, $C$, $D,\ldots,$ $K$, $L$ and $A'$, $B'$,
$C'$, $D',\ldots,$ $K'$, $L'$ is so arranged that $B$ lies
between $A$ and $C$, $D,\ldots,$ $K$, $L$, and $C$ between
$A$, $B$ and $D,\ldots,$ $K$, $L$, etc., then the points $A'$,
$B'$, $C'$, $D',\ldots,$ $K'$, $L'$ of the second series are
arranged in a similar way; that is to say, $B'$
lies between $A'$ and $C'$, $D',\ldots,$ $K'$, $L'$, and $C'$
lies between $A'$, $B'$ and $D',\ldots,$ $K'$, $L'$, etc.
\end{itemize}

Let the angle $(h,k)$ be congruent to the angle
$(h',k')$. Since, according to axiom IV, 4, the angle
$(h,k)$ is congruent to itself, it follows from axiom IV,
5 that the angle $(h',k')$ is congruent to the angle
$(h,k)$. We say, then, that the angles $(h,k)$ and $(h',k')$
are \emph{congruent to one another}.

{\scshape Definitions}. Two angles having the same vertex
and one side in common, while the sides not common
form a straight line, are called \emph{supplementary angles}.
Two angles having a common vertex and whose sides
form straight lines are called \emph{vertical angles}. An angle
which is congruent to its supplementary angle is called
a \emph{right angle}.

Two triangles $ABC$ and $A'B'C'$ are said to be \emph{congruent}
%%-----File: 026.png-----%%
to one another when all of the following congruences
are fulfilled:
 \[
\begin{array}{ccc}
AB \equiv A'B',            &  AC \equiv A'C',           & BC \equiv B'C',  \\
\angle A \equiv \angle A', & \angle B \equiv \angle B', &  \angle C \equiv \angle C'.
\end{array}
\]

\begin{itemize}
\item []{\scshape Theorem 10.} (First theorem of congruence for
triangles). If, for the two triangles $ABC$ and
$A'B'C'$, the congruences
\[
   AB \equiv A'B', \: AC \equiv A'C', \: \angle A \equiv \angle A'
\]
hold, then the two triangles are congruent to
each other.
\end{itemize}

{\scshape Proof. } From axiom IV, 6, it follows that the
two congruences
\[
\angle B \equiv \angle B' \: \mbox{and}\: \angle C \equiv \angle C'
\]
are fulfilled, and it is, therefore, sufficient to show that
the two sides $BC$ and $B'C'$ are congruent. We will
assume the contrary to be true, namely, that $BC$ and
$B'C'$ are not congruent, and show that this leads to a
contradiction. We take upon $B'C'$ a point $D'$ such that
$BC \equiv B'D'$. The two triangles $ABC$ and $A'B'D'$ have,
then, two sides and the included angle of the one
agreeing, respectively, to two sides and the included
angle of the other. It follows from axiom IV, $6$ that
the two angles $BAC$ and $B'A'D'$ are also congruent to
each other. Consequently, by aid of axiom IV, $5$,
the two angles $B'A'C'$ and $B'A'D'$ must be congruent.

\begin{figure}[htb]
\begin{center}
\includegraphics*[width=3in]{images/f009.png}\\
{\small Fig. 9.}
\end{center}
\end{figure}

%%-----File: 027.png-----%%
This, however, is impossible, since, by axiom IV, 4,
an angle can be laid off in one and only one way on a
given side of a given half-ray of a plane. From this
contradiction the theorem follows.

We can also easily demonstrate the following theorem:

\begin{itemize}
\item[]{\scshape Theorem 11.} (Second theorem of congruence
for triangles). If in any two triangles one side
and the two adjacent angles are respectively
congruent, the triangles are congruent.
\end{itemize}

We are now in a position to demonstrate the following
important proposition.

\begin{itemize}
\item[]{\scshape Theorem 12.} If two angles $ABC$ and $A'B'C'$ are
congruent to each other, their supplementary
angles $CBD$ and $C'B'D'$ are also congruent.
\end{itemize}

\begin{figure}[htb]
\begin{center}
\includegraphics*[width=3in]{images/f010.png}\\
{\small Fig. 10.}
\end{center}
\end{figure}

{\scshape Proof.} Take the points $A'$, $C'$, $D'$ upon the sides
passing through $B'$ in such a way that
\[
A'B' \equiv AB,\ C'B' \equiv CB,\ D'B' \equiv DB.
\]
Then, in the two triangles $ABC$ and $A'B'C'$, the sides
$AB$ and $BC$ are respectively congruent to $A'B'$ and
$C'B'$. Moreover, since the angles included by these
sides are congruent to each other by hypothesis, it
follows from theorem 10 that these triangles are congruent;
that is to say, we have the congruences
\[
AC \equiv A'C,\ \angle BAC \equiv \angle B'A'C'.
\]
On the other hand, since by axiom IV, 3 the segments
%%-----File: 028.png-----%%
$AD$ and $A'D'$ are congruent to each other, it follows
again from theorem~10 that the triangles $CAD$ and
$C'A'D'$ are congruent, and, consequently, we have the
congruences:
\[
  CD \equiv C'D',\ \angle ADC \equiv \angle A'D'C'.
\]
From these congruences and the consideration of the
triangles $BCD$ and $B'C'D'$, it follows by virtue of
axiom~IV, 6 that the angles $CBD$ and $C'B'D'$ are congruent.

As an immediate consequence of theorem~12, we
have a similar theorem concerning the congruence of
vertical angles.

\begin{itemize}
\item []{\scshape Theorem 13.} Let the angle ($h$, $k$) of the plane $\alpha$
be congruent to the angle ($h'$, $k'$) of the plane
$\alpha'$, and, furthermore, let $l$ be a half-ray in the
plane $\alpha$ emanating from the vertex of the angle
($h$, $k$) and lying within this angle. Then, there
always exists in the plane $\alpha'$ a half-ray $l'$ emanating
from the vertex of the angle ($h'$, $k'$) and
lying within this angle so that we have
\[
\angle (h, l) \equiv \angle (h', l'),\ \angle (k, l) \equiv \angle (k', l').
\]
\end{itemize}

\begin{figure}[htb]
\begin{center}
\includegraphics*[width=3in]{images/f011.png}\\
{\small Fig. 11.}
\end{center}
\end{figure}

{\scshape Proof}. We will represent the vertices of the angles
($h$, $k$) and ($h'$, $k'$) by $O$ and $O'$, respectively, and
so select upon the sides $h$, $k$, $h'$, $k'$ the points $A$, $B$,
$A'$, $B'$ that the congruences
\[
  OA \equiv O'A',\ OB \equiv O'B'
\]
%%-----File: 029.png-----%%
are fulfilled. Because of the congruence of the triangles
$OAB$ and $O'A'B'$, we have at once
\[
AB \equiv A'B',\ \angle OAB \equiv O'A'B',\ \angle OBA \equiv \angle O'B'A'.
\]
Let the straight line $AB$ intersect $l$ in $C$. Take the
point $C'$ upon the segment $A'B'$ so that $A'C' \equiv AC$.
Then, $O'C'$ is the required half-ray. In fact, it follows
directly from these congruences, by aid of axiom
IV, 3, that $BC \equiv B'C'$. Furthermore, the triangles
$OAC$ and $O'A'C'$ are congruent to each other, and the
same is true also of the triangles $OCB$ and $O'B'C'$.
With this our proposition is demonstrated.

In a similar manner, we obtain the following proposition.

\begin{itemize}
\item[]{\scshape Theorem 14.} Let $h$, $k$, $l$ and $h'$, $k'$, $l'$ be two sets
of three half-rays, where those of each set emanate
from the same point and lie in the same
plane. Then, if the congruences
\[
\angle (h, l) \equiv \angle (h', l'),\ \angle (k, l) \equiv \angle (k', l')
\]
are fulfilled, the following congruence is also
valid; viz.:
\[
\angle (h,k) \equiv \angle (h', k').
\]
\end{itemize}

By aid of theorems 12 and 13, it is possible to deduce
the following simple theorem, which Euclid held--although
it seems to me wrongly--to be an axiom.

\begin{itemize}
\item[]{\scshape Theorem 15.} All right angles are congruent to
one another.
\end{itemize}

{\scshape Proof.} Let the angle $BAD$ be congruent to its
supplementary angle $CAD$, and, likewise, let the angle
$B'A'D'$ be congruent to its supplementary angle
$C'A'D'$. Hence the angles $BAD$, $CAD$, $B'A'D'$, and
$C'A'D'$ are all right angles. We will assume that the
%%-----File: 030.png-----%%
contrary of our proposition is true, namely, that the
right angle $B'A'D'$ is not congruent to the right angle
$BAD$, and will show that this assumption leads to a
contradiction. We lay off the angle $B'A'D'$ upon the
half-ray $AB$ in such a manner that the side $AD''$ arising
from this operation falls either within the angle
$BAD$ or within the angle $CAD$. Suppose, for example,
the first of these possibilities to be true. Because
of the congruence of the angles $B'A'D'$ and
$BAD''$, it follows from theorem 12 that angle $C'A'D'$
is congruent to angle $CAD''$, and, as the angles $B'A'D'$
and $C'A'D'$ are congruent to each other, then, by
IV, 5, the angle $BAD''$ must be congruent to $CAD''$.

\begin{figure}[htb]
\begin{center}
\includegraphics*[width=4in]{images/f012.png}\\
{\small Fig. 12.}
\end{center}
\end{figure}

Furthermore, since the angle $BAD$ is congruent to the
angle $CAD$, it is possible, by theorem 13, to find within
the angle $CAD$ a half-ray $AD'''$ emanating from $A$, so
that the angle $BAD''$ will be congruent to the angle
$CAD'''$, and also the angle $DAD''$ will be congruent
to the angle $DAD'''$. The angle $BAD''$ was shown
to be congruent to the angle $CAD''$ and, hence, by
axiom IV, 5, the angle $CAD''$, is congruent to the
angle $CAD'''$. This, however, is not possible; for,
according to axiom IV, 4, an angle can be laid off in
a plane upon a given side of a given half-ray in only
one way. With this our proposition is demonstrated.
%%-----File: 031.png-----%%
We can now introduce, in accordance with common
usage, the terms ``\emph{acute angle}'' and ``\emph{obtuse angle}.''

The theorem relating to the congruence of the
base angles $A$ and $B$ of an equilateral triangle $ABC$
follows immediately by the application of axiom IV,
6 to the triangles $ABC$ and $BAC$. By aid of this theorem,
in addition to theorem 14, we can easily demonstrate
the following proposition.

\begin{itemize}
\item[]{\scshape Theorem 16.} (Third theorem of congruence for
triangles.) If two triangles have the three sides
of one congruent respectively to the corresponding
three sides of the other, the triangles are
congruent.
\end{itemize}

Any finite number of points is called a \emph{figure}. If
all of the points lie in a plane, the figure is called a
\emph{plane figure}.

Two figures are said to be \emph{congruent} if their points
can be arranged in a one-to-one correspondence so
that the corresponding segments and the corresponding
angles of the two figures are in every case congruent
to each other.

Congruent figures have, as may be seen from theorems
9 and 12, the following properties: Three points
of a figure lying in a straight line are likewise in a
straight line in every figure congruent to it. In congruent
figures, the arrangement of the points in corresponding
planes with respect to corresponding lines
is always the same. The same is true of the sequence
of corresponding points situated on corresponding
lines.

The most general theorems relating to congruences
in a plane and in space may be expressed as follows:
%%-----File: 032.png-----%%
\begin{itemize}
\item[]{\scshape Theorem 17.} If $(A, B, C,\ldots)$ and $(A', B', C',\ldots)$
are congruent plane figures and P is a
point in the plane of the first, then it is always
possible to find a point $P$ in the plane of the
second figure so that $(A, B, C,\ldots , P)$ and $(A', B', C', \ldots , P')$
shall likewise be congruent figures.
If the two figures have at least three
points not lying in a straight line, then the selection
of $P'$ can be made in only one way.

\item[]{\scshape Theorem 18.} If $(A, B, C, \ldots)$ and $(A', B', C',\ldots=$
are congruent figures and $P$ represents
any arbitrary point, then there can always be
found a point $P'$ so that the two figures $(A, B, C, \ldots, P)$
and $(A', B', C', \ldots, P')$ shall
likewise be congruent. If the figure $(A, B, C, \ldots , P)$
contains at least four points not lying
in the same plane, then the determination of
$P'$ can be made in but one way.
\end{itemize}

This theorem contains an important result; namely,
that all the facts concerning space which have reference
to congruence, that is to say, to displacements
in space, are (by the addition of the axioms of groups
I and II) exclusively the consequences of the six
linear and plane axioms mentioned above. Hence, it
is not necessary to assume the axiom of parallels in
order to establish these facts.

If we take, in, addition to the axioms of congruence,
the axiom of parallels, we can then easily establish
the following propositions:

\begin{itemize}
\item[]{\scshape Theorem 19.} If two parallel lines are cut by a
third straight line, the alternate-interior angles
and also the exterior-interior angles are congruent
%%-----File: 033.png-----%%
\label{folio26}
Conversely, if the alternate-interior or
the exterior-interior angles are congruent, the
given lines are parallel.

\item[]{\scshape Theorem 20.} The sum of the angles of a triangle
is two right angles.
\end{itemize}

{\scshape Definitions.} If $M$ is an arbitrary point in the
plane $a$, the totality of all points $A$, for which the segments
$MA$ are congruent to one another, is called \emph{a
circle}. $M$ is called the \emph{centre of the circle}.

From this definition can be easily deduced, with
the help of the axioms of groups III and IV, the
known properties of the circle; in particular, the possibility
of constructing a circle through any three
points not lying in a straight line, as also the congruence
of all angles inscribed in the same segment of
a circle, and the theorem relating to the angles of an
inscribed quadrilateral.

\rfa{\S\!~8.\ {\small GROUP V. AXIOM OF CONTINUITY. (ARCHIMEDEAN AXIOM.)}}\label{p8}

This axiom makes possible the introduction into
geometry of the idea of continuity. In order to state
this axiom, we must first establish a convention concerning
the equality of two segments. For this purpose,
we can either base our idea of equality upon the
axioms relating to the congruence of segments and
define as ``\emph{equal}'' the correspondingly congruent segments,
or upon the basis of groups I and II, we may
determine how, by suitable constructions (see Chap.
V, \S~24), a segment is to be laid off from a point of a
given straight line so that a new, definite segment is
obtained ``\emph{equal}'' to it. In conformity with such a
%%-----File: 034.png-----%%
\label{folio27}
convention, the axiom of Archimedes may be stated
as follows:

\begin{quote}\emph{V. Let $A_1$ be any point upon a straight line between
the arbitrarily chosen points $A$ and $B$. Take the
points $A_2$, $A_3$, $A_4,\ldots$ so that $A_1$ lies between $A$
and $A_2$, $A_2$ between $A_1$ and $A_3$, $A_3$ between $A_2$ and
$A_4$ etc. Moreover, let the segments
\[
  A A_1, \; A_1 A_2, \; A_2 A_3, \; A_3 A_4, \;\ldots
\]
be equal to one another. Then, among this series
of points, there always exists a certain point $A_n$
such that $B$ lies between $A$ and $A_n$.}
\end{quote}

The axiom of Archimedes is a \emph{linear} axiom.

{\scshape Remark}.\footnote{Added
by Professor Hilbert in the French translation.---\textit{Tr.}}
To the preceeding five groups of axioms,
we may add the following one, which, although
not of a purely geometrical nature, merits particular
attention from a theoretical point of view. It may be
expressed in the following form:

\begin{quote}
{\scshape Axiom of Completeness}.\footnote{See Hilbert,
``Ueber den Zahlenbegriff,'' \textit{Berichte der deutschen Mathematiker-Vereinigung},
1900.}
\emph{(Vollst\"andigkeit): To a
system of points, straight lines, and planes, it is
impossible to add other elements in such a manner
that the system thus generalized shall form a new
geometry obeying all of the five groups of axioms.
In other words, the elements of geometry form a
system which is not susceptible of extension, if we
regard the five groups of axioms as valid.}
\end{quote}

This axiom gives us nothing directly concerning
the existence of limiting points, or of the idea of convergence.
Nevertheless, it enables us to demonstrate
Bolzano's theorem by virtue of which, for all sets of
%%-----File: 035.png-----%%
points situated upon a straight line between two definite
points of the same line, there exists necessarily
a point of condensation, that is to say, a limiting point.
From a theoretical point of view, the value of this
axiom is that it leads indirectly to the introduction
of limiting points, and, hence, renders it possible to
establish a one-to-one correspondence between the
points of a segment and the system of real numbers.
However, in what is to follow, no use will be made of
the ``axiom of completeness.''
\newpage
%%-----File: 036.png-----%%
\begin{center}{\Large COMPATIBILITY AND MUTUAL INDEPENDENCE OF THE AXIOMS.}\end{center}
\rfa{\S\!~9.\ {\small COMPATIBILITY OF THE AXIOMS.}}\label{p9}

The axioms, which we have discussed in the previous
chapter and have divided into five groups,
are not contradictory to one another; that is to say,
it is not possible to deduce from these axioms, by any
logical process of reasoning, a proposition which is
contradictory to any of them. To demonstrate this,
it is sufficient to construct a geometry where all of the
five groups are fulfilled.

To this end, let us consider a domain $\Omega$ consisting
of all of those algebraic numbers which may be obtained
by beginning with the number one and applying
to it a finite number of times the four arithmetical
operations (addition, subtraction, multiplication,
and division) and the operation $\sqrt{1+\omega^2}$, where $\omega$ represents
a number arising from the five operations already
given.

Let us regard a pair of numbers $(x, y)$ of the domain
$\Omega$ as defining a point and the ratio of three such
numbers $(u:v:w)$ of $\Omega$, where $u$, $v$ are not both equal
to zero, as defining a straight line. Furthermore, let
the existence of the equation
\[
ux+vy+w = 0
\]
express the condition that the point $(x, y)$ lies on the
%%-----File: 037.png-----%%
\label{folio30}
straight line $(u:v:w)$. Then, as one readily sees,
axioms I, 1--2 and III are fulfilled. The numbers of
the domain $\Omega$ are all real numbers. If now we take
into consideration the fact that these numbers may be
arranged according to magnitude, we can easily make
such necessary conventions concerning our points and
straight lines as will also make the axioms of order
(group II) hold. In fact, if $(x_1,y_1)$, $(x_2, y_2)$, $(x_3, y_3)$,
$\ldots$ are any points whatever of a straight line, then
this may be taken as their sequence on this straight
line, providing the numbers $x_1$, $x_2$, $x_3,\ldots,$ or the
numbers $y_1$, $y_2$, $y_3, \ldots,$ either all increase or decrease
in the order of sequence given here. In order that
axiom II, 5 shall be fulfilled, we have merely to assume
that all points corresponding to values of $x$ and
$y$ which make $ux+vy+w$ less than zero or greater
than zero shall fall respectively upon the one side or
upon the other side of the straight line $(u:v:w)$.
We can easily convince ourselves that this convention
is in accordance with those which precede, and
by which the sequence of the points on a straight line
has already been determined.

The laying off of segments and of angles follows
by the known methods of analytical geometry. A
transformation of the form
\begin{eqnarray*}
x' & = & x + a \\
y' & = & y + b
\end{eqnarray*}
produces a translation of segments and of angles.

\begin{figure}[htb]
\begin{center}
\includegraphics*[width=4in]{images/f013.png}\\
{\small Fig. 13.}
\end{center}
\end{figure}

Furthermore, if, in the accompanying figure, we represent
the point $(0, 0)$ by $O$ and the point $(1, 0)$ by $E$,
then, corresponding to a rotation of the angle $COE$
%%-----File: 038.png-----%%
about $O$ as a center, any point $(x, y)$ is transformed
into another point $(x', y')$ so related that
\begin{eqnarray*}
 x' & = & \frac{a}{\sqrt{a^2+b^2}} x -  \frac{b}{\sqrt{a^2+b^2}}y \mbox{,} \\
 y' & = & \frac{b}{\sqrt{a^2+b^2}} x + \frac{a}{\sqrt{a^2+b^2}}y \mbox{.}
\end{eqnarray*}

Since the number
\[
 \sqrt{a^2+b^2} = a \sqrt{ 1 + \left( \frac{b}{a} \right)^2 }
\]
belongs to the domain $\Omega$, it follows that, under the
conventions which we have made, the axioms of congruence
(group IV) are all fulfilled. The same is true
of the axiom of Archimedes.

\begin{figure}[htb]
\begin{center}
\includegraphics*[width=2.5in]{images/f014.png}\\
{\small Fig. 14.}
\end{center}
\end{figure}

From these considerations, it follows that every
contradiction resulting from our system of axioms
must also appear in the arithmetic related to the
domain $\Omega$.

The corresponding considerations for the geometry
of space present no difficulties.

If, in the preceding development, we had selected
the domain of all real numbers instead of the domain
%%-----File: 039.png-----%%
$\Omega$, we should have obtained likewise a geometry in
which all of the axioms of groups I---V are valid. For
the purposes of our demonstration, however, it was
sufficient to take the domain $\Omega$, containing on an
enumerable set of elements.

\rfa{\S\!~10.\ {\small INDEPENDENCE OF THE AXIOMS OF PARALLELS. (NON-EUCLIDEAN GEOMETRY.)}\footnote{The
mutual independence of Hilbert's system of
axioms has also been discussed recently by Schur and Moore. Schur's paper,
entitled ``Ueber die Grundlagen der Geometrie'' appeared in
\textit{Math. Annalem}, Vol. 55, p. 265, and that of
Moore, ``On the Projective Axioms of Geometry,'' is to be
found in the Jan. (1902) number of the
\textit{Transactions of the Amer. Math. Society.---Tr.}}}\label{p10}


Having shown that the axioms of the above system are not
contradictory to one another, it is of interest to
investigate the question of their mutual independence.
In fact, it may be shown that none of them can be
deduced from the remaining ones by any logical process
of reasoning.

First of all, so far as the particular axioms of groups
I, II, and IV are concerned, it is easy to show that
the axioms of these groups are each independent of the
other of the same group.\footnote{See my lectures upon
Euclidean Geometry, winter semester of 1898--1899, which
were reported by Dr. Von Schaper and manifolded for the
members of the class.}

According to our presentation, the axioms of groups I and II form
the basis of the remaining axioms. It is sufficient, therefore, to
show that each of the groups II, IV, and V is independent of the
others.

The first statement of the axiom of parallels can be demonstrated
by aid of the axioms of groups I, II, and IV\@. In order to do this,
join the given point $A$ with any arbitrary point $B$ of the
straight line $a$. Let $C$ be any other point of the given straight
line. At
%%-----File: 040.png-----%%
the point $A$ on $AB$, construct the angle $ABC$ so that it shall
lie in the same plane $\alpha$ as the point $C$, but upon the
opposite side of $AB$ from it. The straight line thus obtained
through $A$ does not meet the give straight line $\alpha$; for, if
it should cut it, say in the point $D$, and if we suppose $B$ to be
situated between $C$ and $D$, we could then find on $\alpha$ a
point $D'$ so situated that $B$ would lie between $D$ and $D'$,
and, moreover, so that we should have
\[
  AD \equiv BD'
\]
Because of the congruence of the two triangles $ABD$ and $BAD'$,
we have also
\[
  \angle ABD \equiv \angle BAD',
\]
and since the angles $ABD'$ and $ABD$ are supplementary,
it follows from theorem 12 that the angles $BAD$ and $BAD'$ are
also supplementary. This, however, cannot be true, as,
by theorem 1, two straight lines cannot intersect in more than one point,
which would be the case if $BAD$ and $BAD'$ were supplementary.

The second statement of the axiom of parallels is independent of
all the other axioms. This may be most easily shown in the
following well known manner. As the individual elements of a
geometry of space, select the points, straight lines, and planes of
the ordinary geometry as constructed in \S~9, and regard these
elements as restricted in extent to the interior of a fixed sphere.
Then, define the congruences of this geometry by aid of such linear
transformations of the ordinary geometry as transform the fixed
sphere into itself. By suitable conventions, we can make this
``\textit{non-euclidean geometry}'' obey all of the axioms of our
system except the axiom of Euclid (group III).
Since the possibility of the ordinary geometry has
%%-----File: 039.png-----%%
already been established, that of the non-euclidean
geometry is now an immediate consequence of the
above considerations.

\rfa{\S\!~11.\ {\small INDEPENDENCE OF THE AXIOMS OF CONGRUENCE.}}\label{p11}

We shall show the independence of the axioms of
congruence by demonstrating that axiom IV, 6, or
what amounts to the same thing, that the first theorem
of congruence for triangles (theorem 10) cannot
be deduced from the remaining axioms I, II, III, IV
1--5, V by any logical process of reasoning.

Select, as the points, straight lines, and planes of
our new geometry of space, the points, straight lines,
and planes of ordinary geometry, and define the laying
off of an angle as in ordinary geometry, for example,
as explained in \S~9. We will, however, define the laying
off of segments in another manner. Let $A_1$, $A_2$ be
two points which, in ordinary geometry, have the co-ordinates
$x_1$, $y_1$, $z_1$ and $x_2$, $y_2$, $z_2$, respectively. We
will now define the length of the segment $A_1 A_2$ as the
positive value of the expression
\[
\sqrt{(x_1-x_2+y_1-y_2)^2 + (y_1-y_2)^2 + (z_1-z_2)^2}
\]
and call the two segments $A_1A_2$ and $A'_1A'_2$ congruent
when they have equal lengths in the sense just defined.

It is at once evident that, in the geometry of space
thus defined, the axioms I, II, III, IV 1--2, 4--5, V are
all fulfilled.

In order to show that axiom IV, 3 also holds, we
select an arbitrary straight line $a$ and upon it three
points $A_1$, $A_2$, $A_3$ so that $A_2$ shall lie between $A_1$ and
$A_3$. Let the points $x$, $y$, $z$ of the straight line $a$ be
given by means of the equations
%%-----File: 040.png-----%%
\begin{eqnarray*}
x & = & \lambda t + \lambda', \\
y & = & \mu t + \mu', \\
z & = & \nu t + \nu',
\end{eqnarray*}
where $\lambda$, $\lambda'$, $\mu$, $\mu'$, $\nu$, $\nu'$ represent certain constants and
$T$ is a parameter. If $t_1$, $t_2\ (<t_1)$, $t_3\ (<t_2)$ are the values
of the parameter corresponding to the points $A_1$, $A_2$,
$A_3$ we have as the lengths of the three segments $A_1A_2$
$A_2A_3$ and $A_1A_3$ respectively, the following values:
\[
(t_1-t_2) \left| \sqrt{(\lambda + \mu)^2 + \mu^2 + \nu^2} \right|
\]
\[
(t_2-t_3) \left| \sqrt{(\lambda + \mu)^2 + \mu^2 + \nu^2} \right|
\]
\[
(t_1-t_3) \left| \sqrt{(\lambda + \mu)^2 + \mu^2 + \nu^2} \right|
\]
Consequently, the length of $A_1A_3$ is equal to the sum of
the lengths of the segments $A_1A_2$ and $A_2A_3$. But this
result is equivalent to the existence of axiom IV, 3.

Axiom IV, 6, or rather the first theorem of congruence
for triangles, is not always fulfilled in this
geometry. Consider, for example, in the plane $z = 0$,
the four points

\begin{center}
\begin{tabular}{cccccc}
$O$, & \rfb having& \rfb the& \rfb co-ordinates& \rfb $x=0$, & \rfb $y = 0$ \\
$A$, & \rfb  `` &  \rfb  `` &  \rfb  `` & \rfb $x=1$, & \rfb $y=0$ \\
$B$, & \rfb  `` &  \rfb  `` &  \rfb  `` & \rfb $x=0$, & \rfb $y=1$ \\
$C$, & \rfb  `` &  \rfb  `` &  \rfb  `` & \rfb $x=\frac{1}{2}$, & \rfb $y=\frac{1}{2}$
\end{tabular}
\end{center}

\begin{figure}[htb]
\begin{center}
\includegraphics*[width=2in]{images/f015.png}\\
{\small Fig. 15.}
\end{center}
\end{figure}

\noindent Then, in the right triangles
$OAC$ and $OBC$, the angles at
$C$ as also the adjacent sides
$AC$ and $BC$ are respectively
congruent; for, the side $OC$ is
common to the two triangles
and the sides $AC$ and $BC$ have
the same length, namely, $\frac{1}{2}$.
However, the third sides $OA$
and $OB$ have the lengths $1$ and $\sqrt{2}$, respectively, and
are not, therefore, congruent.
%%-----File: 041.png-----%%
\label{folio34}
It is not difficult to find in this geometry two triangles
for which axiom IV, 6, itself is not valid.

\rfa{\S\!~12.\ {\small INDEPENDENCE OF THE AXIOM OF CONTINUITY. (NON-ARCHIMEDEAN GEOMETRY.)}}\label{p12}

In order to demonstrate the independence of the
axiom of Archimedes, we must produce a geometry
in which all of the axioms are fulfilled with the exception
of the one in question.\footnote{In
his very scholarly book,---\emph{Grundz\"uge der Geometrie}, German translation
by A.~Schepp, Leipzig, 1894,---G.~Veronese has also attempted the construction
of a geometry independent of the axiom of Archimedes.}

For this purpose, we construct a domain $\Omega(t)$ of
all those algebraic functions of $t$ which may be obtained
from $t$ by means of the four arithmetical operations
of addition, subtraction, multiplication, division,
and the fifth operation $\sqrt{1+\omega^2}$, where $\omega$ represents
any function arising from the application of these five
operations. The elements of $\Omega(t)$---just as was previously
the case for $\Omega$---constitute an enumerable set.
These five operations may all be performed without
introducing imaginaries, and that in only one way.
The domain $\Omega(t)$ contains, therefore, only real, single-valued
functions of $t$.

Let $c$ be any function of the domain $\Omega(t)$. Since
this function $c$ is an algebraic function of $t$, it can in
no case vanish for more than a finite number of values
of $t$, and, hence, for sufficiently large positive values of
$t$, it must remain always positive or always negative.

Let us now regard the functions of the domain
$\Omega(t)$ as a kind of complex numbers. In the system of
complex numbers thus defined, all of the ordinary
rules of operation evidently hold. Moreover, if $a$, $b$
are any two distinct numbers of this system, then $a$
%%-----File: 042.png-----%%
is said to be greater than, or less than, $b$ (written $a>b$
or $a<b$) according as the difference $c = a-b$ is always
positive or always negative for sufficiently large values
of $t$. By the adoption of this convention for the numbers
of our system, it is possible to arrange them according
to their magnitude in a manner analogous to
that employed for real numbers. We readily see also
that, for this system of complex numbers, the validity
of an inequality is not destroyed by adding the same
or equal numbers to both members, or by multiplying
both members by the same number, providing it is
greater than zero.

If $n$ is any arbitrary positive integral rational number,
then, for the two numbers $n$ and $t$ of the domain
$\Omega (t)$, the inequality $n<t$ certainly holds; for,
the difference $n-t$, considered as a function of $t$, is always
negative for sufficiently large values of $t$. We express
this fact in the following manner: The two numbers
$l$ and $t$ of the domain $\Omega (t)$, each of which is greater
than zero, possess the property that any multiple
whatever of the first always remains smaller than the
second.

From the complex numbers of the domain $\Omega (t)$,
we now proceed to construct a geometry in exactly
the same manner as in \S~9, where we took as the basis
of our consideration the algebraic numbers of the domain
$\Omega$. We will regard a system of three numbers
$(x,y,z)$ of the domain $\Omega (t)$ as defining a point, and
the ratio of any four such numbers $(u:v:w:r)$, where
$u$, $v$, $w$ are not all zero, as defining a plane. Finally,
the existence of the equation
\[
xu + yv + zw + r = 0
\]
shall express the condition that the point $(x, y, z)$ lies
in the plane $(u:v:w:r)$. Let the straight line be defined
%%-----File: 043.png-----%%
in our geometry as the totality of all the points
lying simultaneously in the same two planes. If now
we adopt conventions corresponding to those of \S~9
concerning the arrangement of elements and the laying
off of angles and of segments, we shall obtain a
\emph{``non-archimedean'' geometry} where, as the properties
of the complex number system already investigated
show, all of the axioms, with the exception of that of
Archimedes, are fulfilled. In fact, we can lay off successively
the segment 1 upon the segment $t$ an arbitrary
number of times without reaching the end point
of the segment $t$, which is a contradiction to the axiom
of Archimedes.
\newpage
%%-----File: 044.png-----%%
\label{folio37}
\begin{center}{\Large THE THEORY OF PROPORTION.\footnote{See
also Schur, \emph{Math. Annalen}, Vol.~55, p.~265.---\emph{Tr.}}}\end{center}
\rfa{\S\!~13.\ {\small COMPLEX NUMBER-SYSTEMS.}}\label{p13}

At the beginning of this chapter, we shall present
briefly certain preliminary ideas concerning complex
number systems which will later be of service to
us in our discussion.

The real numbers form, in their totality, a system
of things having the following properties:

\begin{center}
THEOREMS OF CONNECTION (1--12).
\end{center}

\begin{description}
\item[1.] From the number $a$ and the number $b$, there
is obtained by ``addition'' a definite number $c$,
which we express by writing
\[
a+b= c \mbox{ or } c=a+b.
\]

\item[2.] There exists a definite number, which we call
$0$, such that, for every number $a$, we have
\[
a+0= a \mbox{ and } 0 + a = a.
\]

\item[3.] If $a$ and $b$ are two given numbers, there exists
one and only one number $x$, and also one and
only one number $y$, such that we have respectively
\[
a+x = b,\ y+a=b.
\]

\item[4.] From the number $a$ and the number $b$, there
may be obtained in another way, namely, by
%%-----File: 045.png-----%%
``multiplication,'' a definite number $c$, which
we express by writing
\[
ab = c \mbox{ or } c = ab.
\]

\item[5.] There exists a definite number, called $1$, such
that, for every number $a$, we have
\[
a\cdot 1 = a \mbox{ and } 1 \cdot a = a.
\]

\item[6.] If $a$ and $b$ are any arbitrarily given numbers,
where $a$ is different from $0$, then there exists
one and only one number $x$ and also one and
only one number $y$ such that we have respectively
\[
ax = b,\,ya=b.
\]
\end{description}

If $a$, $b$, $c$ are arbitrary numbers, the following laws
of operation always hold:\\
\hspace*{-2em}
\begin{tabular}{lll}
  {\begin{tabular}{r} \textbf{7.} \\ \textbf{8.} \\ \textbf{9.} \\ \textbf{10.} \\ \textbf{11.} \\ \textbf{12.} \end{tabular} }
  &
  \hspace*{1in}
  &
  $\begin{array}{lll}
  a + (b + c) &  = &  (a + b) + c \\
  a+b         & = &  b+a \\
  a(bc)       & = & (ab)c \\
  a(b + c)   & = & ab+ac \\
  (a+b)c      & = & ac+bc \\
  ab          & = & ba.
  \end{array}$
\end{tabular}

\begin{center}
THEOREMS OF ORDER (13--16).
\end{center}
\medskip

\begin{description}
\item[13.] If $a$, $b$ are any two distinct numbers, one of
these, say $a$, is always greater ($>$) than the
other. The other number is said to be the
smaller of the two. We express this relation
by writing
\[
a > b \mbox{ and } b < a.
\]
\item[14.] If $a > b$ and $b > c$, then is also $a > c$.
\item[15.] If $a > b$, then is also $a+c > b+ c$ and $c+a > c+b$.
%%-----File: 046.png-----%%
\item[16.] If $a > b$ and $c>0$, then is also $ac>bc$ and
$ca > cb$.
\end{description}
\medskip

\begin{center}
THEOREM OF ARCHIMEDES (17).
\end{center}
\medskip

\begin{description}
\item[17.] If $a$, $b$ are any two arbitrary numbers, such
that $a > 0$ and $b > 0$, it is always possible to
add $a$ to itself a sufficient number of times so
that the resulting sum shall have the property
that
\[
a+a+a+ \cdots + a > b.
\]
\end{description}

A system of things possessing only a portion of the
above properties (1--17) is called a \emph{complex number
system}, or simply a \emph{number system}. A number system
is called \emph{archimedean}, or \emph{non-archimedean}, according as
it does, or does not, satisfy condition (17).

Not every one of the properties (1--17) given above
is independent of the others. The problem arises to
investigate the logical dependence of these properties.
Because of their great importance in geometry, we
shall, in \S\S~32, 33, pp.~\pageref{folio101}--\pageref{folio106}, answer two definite
questions of this character. We will here merely call
attention to the fact that, in any case, the last of these
conditions (17) is not a consequence of the remaining
properties, since, for example, the complex number
system $\Omega(t)$, considered in \S~12, possesses all of the
properties (1--16), but does not fulfil the law stated
in (17).

\rfa{\S\!~14.\ {\small DEMONSTRATION OF PASCAL'S THEOREM.}}\label{p14}

In this and the following chapter, we shall take as
the basis of our discussion all of the plane axioms
with the exception of the axiom of Archimedes; that
is to say, the axioms I, 1--2 and II--IV\@. In the present
%%-----File: 047.png-----%%
chapter, we propose, by aid of these axioms, to
establish Euclid's theory of proportion; that is, {\em we
shall establish it for the plane and that independently of
the axiom of Archimedes}.

\begin{figure}[htb]
\begin{center}
\includegraphics*[width=3in]{images/f016.png}\\
{\small Fig. 16.}
\end{center}
\end{figure}

For this purpose, we shall first demonstrate a proposition
which is a special case of the well known theorem
of Pascal usually considered in the theory of
conic sections, and which we shall hereafter, for the
sake of brevity, refer to simply as Pascal's theorem.
This theorem may be stated as follows:

\begin{itemize}
\item[] {\scshape Theorem 21.} (Pascal's theorem.) Given the two
sets of points $A$, $B$, $C$ and $A'$, $B'$, $C'$ so situated
respectively upon two intersecting straight lines
that none of them fall at the intersection of
these lines. If $CB'$ is parallel to $BC'$ and $CA'$
is also parallel to $AC'$, then $BA'$ is parallel to
$AB'$.\footnote{F. Schur has published in the \emph{Math. Ann.}, Vol. 51, a very interesting
proof of the theorem of Pascal, based upon the axioms I--II, IV.}
\end{itemize}

In order to demonstrate this theorem, we shall
first introduce the following notation. In a right
triangle, the base $a$ is uniquely determined by the
%%-----File: 048.png-----%%
hypotenuse $c$ and the base angle $\alpha$ included by $a$ and
$c$.  We will express this fact
briefly by writing
\[
a=\alpha c.
\]
\vspace*{2mm}
\begin{figure}[htb]
\begin{center}
\includegraphics*[width=2in]{images/f017.png}\\
{\small Fig. 17.}
\end{center}
\end{figure}

Hence, the symbol $\alpha c$ always
represents a definite segment,
providing $c$ is any given segment
whatever and $\alpha$ is any given acute angle.

Furthermore, if $c$ is any arbitrary segment and $\alpha$,
$\beta$ are any two acute angles whatever, then the two
segments $\alpha \beta c$ and $\beta \alpha c$ are always congruent; that is,
we have
\[
\alpha \beta c = \beta \alpha c,
\]
and, consequently, the symbols $\alpha$ and $\beta$ are interchangeable.

\begin{figure}[htb]
\begin{center}
\includegraphics*[width=3in]{images/f018.png}\\
{\small Fig. 18.}
\end{center}
\end{figure}

In order to prove this statement, we take the segment
$c = AB$, and with $A$ as a vertex lay off upon the
one side of this segment
the angle $\alpha$
and upon the other
the angle $\beta$. Then,
from the point $B$,
let fall upon the
opposite sides of
the $\alpha$ and $\beta$ the
perpendiculars $BC$
and $BD$, respectively.
Finally, join
$C$ with $D$ and let fall from $A$ the perpendicular $AE$
upon $CD$.

Since the two angles $ACB$ and $ADB$ are right angles,
the four points $A$, $B$, $C$, $D$ are situated upon a
circle. Consequently, the angles $ACD$ and $ABD$,
%%-----File: 049.png-----%%
being inscribed in the same segment of the circle,
are congruent. But the angles $ACD$ and $CAE$, taken
together, make a right angle, and the same is true of
the two angles $ABD$ and $BAD$. Hence, the two angles
$CAE$ and $BAD$ are also congruent; that is to say,
\[
\angle CAE \equiv \beta
\]
and, therefore,
\[
\angle DAE \equiv \alpha.
\]

From these considerations, we have immediately
the following congruences of segments:
\[\begin{array}{rclrcl}
\beta c  & \equiv & AD, & \alpha c & \equiv & AC, \\
\alpha \beta c & \equiv & \alpha(AD) \equiv AE, & \beta \alpha c & \equiv & \beta(AC) \equiv AE. \\
\end{array}\]
From these, the validity of the congruence in question
follows.

Returning now to the figure in connection with
Pascal's theorem, denote the intersection of the two
given straight lines by $O$ and the segments $OA$, $OB$,
$OC$, $OA'$, $OB'$, $OC'$, $CB'$, $BC'$, $CA'$, $AC'$, $BA'$, $AB'$
by $a$, $b$, $c$, $a'$, $b'$, $c'$, $l$, $l^{\ast}$, $m$, $m^{\ast}$, $n$, $n^{\ast}$, respectively.

\begin{figure}[htb]
\begin{center}
\includegraphics*[width=4in]{images/f019.png}\\
{\small Fig. 19.}
\end{center}
\end{figure}

Let fall from the point $O$ a perpendicular upon each
of the segments $l$, $m$, $n$. The perpendicular to $l$ will
%%-----File: 050.png-----%%
form with the straight lines $OA$ and $OA'$ acute angles,
which we shall denote by $\lambda'$ and $\lambda$, respectively.
Likewise, the perpendiculars to $m$ and $n$ form with
these same lines $OA$ and $OA'$ acute angles, which we
shall denote by $\mu'$, $\mu$ and $\nu'$, $\nu$, respectively. If we
now express, as indicated above, each of these perpendiculars
in terms of the hypotenuse and base angle,
we have the three following congruences of segments:\\
\begin{tabular}{lll}
  \begin{tabular}{l} (1) \\ (2) \\ (3) \end{tabular} &
  \hspace*{10em} &
  $\begin{array}{rcl}
     \lambda b' & \equiv & \lambda' c \\
     \mu a'     & \equiv & \mu' c     \\
     \nu a'     & \equiv & \nu' b.
  \end{array}$
\end{tabular}\\
But since, according to our hypothesis, $l$ is parallel to
$l^{\ast}$ and $m$ is parallel to $m^{\ast}$, the perpendiculars from $O$
falling upon $l^{\ast}$ and $m^{\ast}$ must coincide with the perpendiculars
from the same point falling upon $l$ and $m$,
and consequently, we have\\
\begin{tabular}{lll}
   \begin{tabular}{l} (4) \\ (5) \end{tabular}
   &
   \hspace*{10em}
   &
   $\begin{array}{rcl}
    \lambda c' & \equiv & \lambda' b,\\
    \mu c'     & \equiv & \mu' a.
   \end{array}$
\end{tabular}\\
Multiplying both members of congruence (3) by
the symbol $\lambda' \mu$, and remembering that, as we have
already seen, the symbols in question are commutative,
we have
\[
   \nu \lambda' \mu a' \equiv \nu' \mu \lambda' b.
\]
In this congruence, we may replace $\mu a'$ in the first
member by its value given in (2) and $\lambda' b$ in the second
member by its value given in (4), thus obtaining as a
result
\[
   \nu \lambda' \mu' c \equiv \nu' \mu \lambda c',
\]
or
\[
   \nu \mu' \lambda' c \equiv \nu' \lambda \mu c'.
\]
Here again in this congruence we can, by aid of (1),
%%-----File: 051.png-----%%
replace $\lambda' c$ by $\lambda b'$, and, by aid of (5), we may replace
in the second member $\mu c'$ by $\mu' a$. We then have
\[
\nu\mu'\lambda b' \equiv \nu'\lambda\mu' a,
\]
or
\[
\lambda\mu'\nu b' \equiv \lambda\mu'\nu' a.
\]
Because of the significance of our symbols, we can
conclude at once from this congruence that
\[
\mu'\nu b' \equiv \mu'\nu' a,
\]
and, consequently, that\\
\begin{tabular}{lll}
   \begin{tabular}{l}(6)\end{tabular}
   &
   \hspace*{10em}
   &
   $\begin{array}{l}
         \nu b' \equiv \nu' a.
         \end{array}$
\end{tabular}\\
If now we consider the perpendicular let fall from
$O$ upon $n$ and draw perpendiculars to this same line
from the points $A$ and $B'$, then congruence (6) shows
that the feet of the last two perpendiculars must coincide;
that is to say, the straight line $n^{\ast} = AB'$ makes
a right angle with the perpendicular to $n$ and, consequently,
is parallel to $n$. This establishes the truth
of Pascal's theorem.

Having given any straight line whatever, together
with an arbitrary angle and a point lying outside of
the given line, we can, by constructing the given angle
and drawing a parallel line, find a straight line passing
through the given point and cutting the given
straight line at the given angle. By means of this
construction, we can demonstrate Pascal's theorem in
the following very simple manner, for which, however,
I am indebted to another source.

\begin{figure}[htb]
\begin{center}
\includegraphics*[width=4in]{images/f020.png}\\
{\small Fig. 20.}
\end{center}
\end{figure}

Through the point $B$, draw a straight line cutting
$OA'$ in the point $D'$ and making with it the angle
$OCA'$, so that the congruence\\
(1$\ast$)\hspace*{10em} $\angle OCA' \equiv \angle OD'B$\\
%%-----File: 052.png-----%%
is fulfilled. Now, according to a well known property
of circles, $CBD'A'$ is an inscribed quadrilateral, and,
consequently, by aid of the theorem concerning the
congruence of angles inscribed in the same segment
of a circle, we have the congruence\\
(2$\ast$)\hspace*{10em} $\angle OBA' \equiv \angle OD'C$.\\
Since, by hypothesis, $CA'$ and $AC'$ are parallel, we
have\\
(3$\ast$)\hspace*{10em} $\angle OCA' \equiv \angle OAC'$,\\
and from (1$\ast$) and (3$\ast$) we obtain the congruence
\[
\angle OD'B \equiv \angle OAC'.
\]
However, $BAD'C'$ is also an inscribed quadrilateral,
and, consequently, by virtue of the theorem relating
to the angles of such a quadrilateral, we have the congruence\\
(4$\ast$)\hspace*{10em} $\angle OAD' \equiv \angle OC'B$.\\
But as $CB'$ is, by hypothesis, parallel to $BC'$, we
have also\\
%%-----File: 053.png-----%%
\label{folio46}
(5$\ast$)\hspace*{10em}     $\angle OB'C \equiv \angle OC'B$.\\
From $(4\ast)$ and $(5\ast)$, we obtain the congruence
\[
 \angle OAD' \equiv \angle OB'C,
\]
which shows that $CAD'B'$ is also an inscribed quadrilateral,
and, hence, the congruence\\
(6$\ast$)\hspace*{10em}  $\angle OAB' \equiv \angle OD'C$,\\
is valid. From $(2\ast)$ and $(6\ast)$, it follows that
\[
 \angle OBA' \equiv \angle OAB',
\]
and this congruence shows that $BA'$ and $AB'$ are parallel
as Pascal's theorem demands.
In case $D'$ coincides with one of the points $A'$, $B'$,
$C'$, it is necessary to make a modification of this
method, which evidently is not difficult to do.

\rfa{\S\!~15.\ {\small AN ALGEBRA OF SEGMENTS, BASED UPON PASCAL'S THEOREM.}}\label{p15}

Pascal's theorem, which was demonstrated in the
last section, puts us in a position to introduce into
geometry a method of calculating with segments, in
which all of the rules for calculating with real numbers
remain valid without any modification.

Instead of the word ``congruent'' and the sign $\equiv$,
we make use, in the algebra of segments, of the word
``equal'' and the sign $=$.

\begin{figure}[htb]
\begin{center}
\includegraphics*[width=4in]{images/f021.png}\\
{\small Fig. 21.}
\end{center}
\end{figure}

If $A, B, C$ are three points of a straight line and
if $B$ lies between $A$ and $C$, then we say that $c = AC$ is
%%-----File: 054.png-----%%
the \emph{sum} of the two segments $a=AB$ and $b=BC$. We
indicate this by writing
\[
  c=a+b.
\]
The segments $a$ and $b$ are said to be smaller than c,
which fact we indicate by writing
\[
  a<c,\ b<c.
\]
On the other hand, $c$ is said to be larger than $a$ and $b$,
and we indicate this by writing
\[
   c>a,\ c>b.
\]

From the linear axioms of congruence (axioms
IV, 1--3), we easily see that, for the above definition
of addition of segments, the associative law
\[
  a+(b+c)=(a+b)+c,
\]
as well as the commutative law
\[
  a+b = b+a
\]
is valid.

\begin{wrapfigure}[12]{R}{6.0cm}
\begin{center}
\includegraphics*[width=2.5in]{images/f022.png}\\
{\small Fig. 22.}
\end{center}
\end{wrapfigure}
In order to define geometrically the product of two
segments $a$ and $b$, we shall make use of the following
construction. Select any convenient segment, which,
having been selected, shall remain constant throughout
the discussion, and denote the same by $1$.
Upon the one side of a right angle, lay off
from the vertex $O$ the segment $1$ and
also the segment $b$. Then, from $O$ lay off
upon the other side of the right angle the
segment $a$. Join the extremities of the segments $1$
and $a$ by a straight line,
and from the extremity of $b$ draw a line parallel to
%%-----File: 055.png-----%%
this straight line. This parallel will cut off from the
other side of the right angle a segment $c$. We call
this segment $c$ the \emph{product} of the
segments $a$ and $b$,
and indicate this relation by writing
\[
  c=ab.
\]
\newpage % required for correct placement of Figure 23
\begin{wrapfigure}[18]{L}{6.5cm}
\begin{center}
\includegraphics*[width=2.5in]{images/f023.png}\\
{\small Fig. 23.}
\end{center}
\end{wrapfigure}
We shall now demonstrate that, for this definition
of the multiplication of segments, the
\emph{commutative} law
\[
  ab=ba
\]
holds. For this purpose, we construct in the above
manner the product $ab$. Furthermore, lay off from $0$
upon the first side (I)
of the right angle the
segment $a$ and upon
the other side (II) the
segment $b$. Connect by
a straight line the extremity of the segment
$1$ with the extremity of
$b$, situated on II, and
draw through the endpoint of $a$, on I, a line
parallel to this straight
line. This parallel will
determine, by its intersection with the side II, the
segment $ba$. But, because the two dotted lines are,
by Pascal's theorem, parallel, the segment $ba$ just
found coincides with the segment $ab$ previously
constructed, and our proposition is established.
In order to show that the \emph{associative} law
\[
  a(bc)=(ab)c
\]
holds for the multiplication of segments, we construct
first of all the segment $d=be$,
then $da$, after that the
%%-----File: 056.png-----%%
segment $e=ba$, and finally $ec$. By virtue of Pascal's
theorem, the extremities of the segments $da$ and $ec$
coincide, as may be clearly seen from figure 24.

\begin{figure}[htb]
\begin{center}
\includegraphics*[width=3in]{images/f024.png}\\
{\small Fig. 24.}
\end{center}
\end{figure}

If now we apply the commutative law which we have
just demonstrated, we obtain the above formula, which
expresses the associative law for the multiplication of
two segments.

\begin{figure}[htb]
\begin{center}
\includegraphics*[width=3in]{images/f025.png}\\
{\small Fig. 25.}
\end{center}
\end{figure}

Finally, \textit{the distributive} law
\[
 a(b+c) = ab+ac
\]
also holds for our algebra of segments. In order to
demonstrate this, we construct the segments, $ab$, $ac$,
%%-----File: 057.png-----%%
\label{folio50}
and $a(b+c)$, and draw through the extremity of the
segment $c$ (Fig. 25) a straight line parallel to the other
side of the right angle. From the congruence of the
two right-angled triangles which are shaded in the
figure and the application of the theorem relating to
the equality of the opposite sides of a parallelogram,
the desired result follows. If $b$ and $c$ are any two arbitrary
segments, there is always a segment $a$ to be
found such that $c=ab$. This segment $a$ is denoted
by $\frac{c}{b}$ and is called the \emph{quotient} of $c$ by $b$.
\bigskip

\rfa{\S\!~16.\ {\small PROPORTION AND THE THEOREMS OF SIMILITUDE.}}\label{p16}
\bigskip

By aid of the preceding algebra of segments, we
can establish Euclid's theory of proportion in a manner
free from objections and without making use of
the axiom of Archimedes.

If $a,b,a',b'$ are any four segments whatever, the
proportion
\[
   a:b=a':b'
\]
expresses nothing else than the validity of equation
\[
   ab'=ba'.
\]

{\scshape Definition}. Two triangles are called \emph{similar} when
the corresponding angles are congruent.

\begin{itemize}
\item[ ] {\scshape Theorem 22.} If $a,b$ and $a',b'$ are homologous
sides of two similar triangles, we have the proportion
\[
 a:b=a':b'
\]
\end{itemize}

{\scshape Proof.} We shall first consider the special case
where the angle included between $a$ and $b$ and the
one included between $a'$ and $b'$ are right angles.
\begin{wrapfigure}[16]{R}{6.5cm}
\begin{center}
\includegraphics*[width=2.5in]{images/f026.png}\\
{\small Fig. 26.}
\end{center}
\end{wrapfigure}
Moreover,
%%-----File: 058.png-----%%
we shall assume that the two triangles are laid
off in one and the same right angle. Upon one of the
sides of this right angle, we lay off
from the vertex $0$ the segment $1$, and
through the extremity of this segment,
we draw a straight line parallel to the
hypotenuses of the two triangles.

This parallel determines upon the other side of the right
angle a segment $e$.
Then, according to our definition of the product of
two segments, we have
\[
  b = ea,\ b' = ea',
\]
from which we obtain
\[
  ab'=ba',
\]
that is to say,
\[
  a:b=a':b'.
\]

\begin{wrapfigure}[16]{R}{6.5cm}
\begin{center}
\includegraphics*[width=2.5in]{images/f027.png}\\
{\small Fig. 27.}
\end{center}
\end{wrapfigure}

Let us now return to the general case. In each of
the two similar triangles, find the point of intersection
of the bisectors of
the three angles. Denote these points by
$S$ and $S'$. From
these points let fall
upon the sides of the
triangles the perpendiculars
$r$ and $r'$, respectively.
Denote
the segments thus
determined upon the sides of the triangles by
%%-----File: 059.png-----%%
\[
  a_b, a_c, b_c, b_a, c_a, c_b
\]
and
\[
  a'_b, a'_c, b'_c, b'_a, c'_a, c'_b,
\]
respectively. The special case of our proposition,
demonstrated above, gives us then the following
proportions:
\begin{eqnarray*}
a_b:r = a'_b:r',  & \ &  b_c:r=b'_c:r', \\
a_c:r = a'_c:r',  & \ &  b_a:r=b'_a:r'.
\end{eqnarray*}
By aid of the distributive law, we obtain from these
proportions the following:
\[
  a: r = a': r, \quad  b: r = b': r'.
\]
Consequently, by virtue of the commutative law of
multiplication, we have
\[
a:b=a':b'.
\]

From the theorem just demonstrated, we can easily
deduce the fundamental theorem in the theory of
proportion. This theorem may be stated as follows:

\begin{itemize}
\item[]{\scshape Theorem 23.} If two parallel lines cut from the
sides of an arbitrary angle the segments $a,b$
and $a',b'$ respectively, then we have always the
proportion
\[
  a: b = a': b'.
\]
Conversely, if the four segments $a,b,a',b$
fulfill this proportion and if $a,a'$ and $b,b'$ are
laid off upon the two sides respectively of an
arbitrary angle, then the straight lines joining
the extremities of $a$ and $b$ and of $a'$ and $b'$ are
parallel to each other.
\end{itemize}
%%-----File: 060.png-----%%
\label{folio53}
\bigskip
\rfa{\S\!~17.\ {\small EQUATIONS OF STRAIGHT LINES AND OF PLANES.}}\label{p17}
\bigskip

To the system of segments already discussed, let
us now add a second system. We will distinguish the
segments of the new system from those of the former
one by means of a special sign, and will call them
``\emph{negative}'' segments in contradistinction to the ``\emph{positive}''
segments already considered. If we introduce
also the segment $O$, which is determined by a single
point, and make other appropriate conventions, then
all of the rules deduced in \S~13 for calculating with
real numbers will hold equally well here for
calculating with segments. We call special attention to
the following particular propositions:

\begin{description}
\item[] We have always $a\cdot 1 = 1\cdot a=a$.
\item[] If $a\cdot b=0$, then either $a=0$, or $b=0$.
\item[] If $a>b$ and $c>0$, then $ac>bc$.
\end{description}

In a plane $\alpha$, we now take two straight lines cutting
each other in $O$ at right angles as the fixed axes
of rectangular co-ordinates, and lay off from $O$ upon
these two straight lines the arbitrary segments $x$ and
$y$. We lay off these segments upon the one side or
upon the other side of $O$, according as they are
positive or negative. At the extremities of $x$ and $y$, erect
perpendiculars and determine the point $P$ of their
intersection. The segments $x$ and $y$ are called the
co-ordinates of $P$. Every point of the plane $\alpha$ is uniquely
determined by its co-ordinates $x$, $y$, which may be
positive, negative, or zero.

Let $l$ be a straight line in the plane $\alpha$, such that it
%%-----File: 061.png-----%%
shall pass through $O$ and also through a point $C$
having the co-ordinates $a,b$. If $x,y$ are
the co-ordinates


\begin{figure}[htb]
\begin{center}
\includegraphics*[width=4in]{images/f028.png}\\
{\small Fig. 28.}
\end{center}
\end{figure}

of any point on $l$, it follows at once from theorem 22
that
\[
  a: b = x: y,
\]
or
\[
  bx-ay = 0,
\]
is the equation of the straight line $l$. If $l'$ is a straight
line parallel to $l$ and cutting off from the $x$-axis the
segment $c$, then we may obtain the equation of the
straight line $l'$ by replacing, in the equation for $l$, the
segment $x$ by the segment $x-c$. The desired equation
will then be of the form
\[
  bx-ay-bc=0.
\]

From these considerations, we may easily
conclude, independently of the axiom of Archimedes, that
every straight line of a plane is represented by an
equation which is linear in the co-ordinates $x$, $y$, and,
conversely, every such linear equation represents a
straight line when the co-ordinates are segments
appertaining to the geometry in question.
%%-----File: 062.png-----%%
The corresponding results for the geometry of
space may be easily deduced.

The remaining parts of geometry may now be
developed by the usual methods of analytic geometry.

So far in this chapter, we have made absolutely
no use of the axiom of Archimedes. If now we
assume the validity of this axiom, we can arrange a
definite correspondence between the points on any
straight line in space and the real numbers. This
may be accomplished in the following manner.

We first select on a straight line any two points,
and assign to these points the numbers $0$ and $1$.
Then, bisect the segment $(0,1)$ thus determined and
denote the middle point by the number $\frac{1}{2}$. In the
same way, we denote the middle of $(0, \frac{1}{2})$ by $\frac{1}{4}$, etc.
After applying this process $n$ times, we obtain a point
which corresponds to $\frac{1}{2^n}$.  Now, lay off $m$ times in
both directions from the point $O$ the segment $(O,\frac{1}{2^n})$.
We obtain in this manner a point corresponding to
the numbers $\frac{m}{2^n})$ and $-\frac{m}{2^n})$. From the axiom of Archimedes,
we now easily see that, upon the basis of this
association, to each arbitrary point of a straight line
there corresponds a single, definite, real number, and,
indeed, such that this correspondence possesses the
following property: If $A$, $B$, $C$ are any three points
on a straight line and $\alpha$, $\beta$, $\gamma$
are the corresponding
real numbers, and, if $B$ lies between $A$ and $C$, then
one of the inequalities,
\[
 \alpha<\beta<\gamma\ \mbox{or}\ \alpha>\beta>\gamma,
\]
is always fulfilled.

From the development given in \S~9, p.~\pageref{folio27}, it is
evident, that to every number belonging to the field of
%%-----File: 063.png-----%%
\label{folio56}
algebraic numbers $\Omega$, there must exist a corresponding
point upon the straight line. Whether to every
real number there corresponds a point cannot in general
be established, but depends upon the geometry to
which we have reference.

However, it is always possible to generalize the
original system of points, straight lines, and planes
by the addition of ``ideal'' or ``irrational'' elements,
so that, upon any straight line of the corresponding
geometry, a point corresponds without exception to
every system of three real numbers. By the adoption
of suitable conventions, it may also be seen that, in
this generalized geometry, all of the axioms I--V are
valid. This geometry, generalized by the addition of
irrational elements, is nothing else than the ordinary
analytic geometry of space.
\newpage
%%-----File: 064.png-----%%
\begin{center}{\Large THE THEORY OF PLANE AREAS.}\end{center}
\rfa{\S\!~18.\ {\small EQUAL AREA AND EQUAL CONTENT OF POLYGONS.}\footnote{In
connection with the theory of areas, we desire to call attention to
the following works of M. G\'erard: \textit{Th\`ese de Doctorat sur la g\'eom\'etrie non
euclidienne} (1892) and \textit{G\'eom\'etrie plane} (Paris, 1898). M. G\'erard has developed
a theory concerning the measurement of polygons analogous to that presented
in \S~20 of the present work. The difference is that M. G\'erard makes use of
parallel transversals, while I use transversals emanating from the vertex.
The reader should also compare the following works of F. Schur, where he
will find a similar development: \textit{Sitzungsberichte der Dorpater Naturf. Ges.},
1892, and \textit{Lehrbuch der analytischen Geometrie}, Leipzig, 1898 (introduction).
Finally, let me refer to an article by O. Stolz in \textit{Monatshefte f\"ur Math, und
Phys.}, 1894. (Note by Professor Hilbert in French ed.)\newline
M. G\'erard has also treated the subject of areas in various ways in the
following journals: \textit{Bulletin de Math, spciales} (May, 1895), \textit{Bulletin de la Soci\'et\'e
math\'ematique de France} (Dec., 1895), \textit{Bulletin Math, \'el\'ementaires} (January,
1896, June, 1897, June, 1898). (Note in French ed.)}}\label{p18}

We shall base the investigations of the present
chapter upon the same axioms as were made
use of in the last chapter, \S\S~13--17, namely, upon the
plane axioms of all the groups, with the single exception
of the axiom of Archimedes. This involves then
the axioms I, 1--2 and II--IV.

The theory of proportion as developed in \S\S~13--17
together with the algebra of segments introduced in
the same chapter, puts us now in a position to establish
Euclid's theory of areas by means of the axioms
already mentioned; that is to say, \emph{for the plane geometry,
and that independently of the axiom of Archimedes.}

Since, by the development given in the last chapter,
pp. \pageref{folio37}--\pageref{folio56}, the theory of proportion was made to depend
%%-----File: 065.png-----%%
essentially upon Pascal's theorem (theorem 21),
the same may then be said here of the theory of areas.
This manner of establishing the theory of areas seems
to me a very remarkable application of Pascal's theorem
to elementary geometry.

If we join two points of a polygon $P$ by any arbitrary
broken line lying entirely within the polygon,
we shall obtain two new polygons $P_1$ and $P_2$ whose
interior points all lie within $P$. We say that $P$ is \textit{decomposed
into} $P_1$ and $P_2$, or that the polygon $P$ is \textit{composed
of} $P_1$ and $P_2$.

{\scshape Definition.} Two polygons are said to be of \textit{equal
area} when they can be decomposed into a finite number
of triangles which are respectively congruent to
one another in pairs.

{\scshape Definition.} Two polygons are said to be of \textit{equal
content} when it is possible, by the addition of other
polygons having equal area, to obtain two resulting
polygons having equal area.

From these definitions, it follows at once that by
combining polygons having equal area, we obtain as
a result polygons having equal area. However, if
from polygons having equal area we take polygons
having equal area, we obtain polygons which are of
equal content.

Furthermore, we have the following propositions:

\begin{itemize}
\item[ ] {\scshape Theorem 24.} If each of two polygons $P_1$ and $P_2$
is of equal area to a third polygon $P_3$ then $P_1$
and $P_2$ are themselves of equal area. If each
of two polygons is of equal content to a third,
then they are themselves of equal content.
\end{itemize}

{\scshape Proof.} By hypothesis, we can so decompose each
of the polygons $P_1$ and $P_2$ into such a system of triangles
%%-----File: 066.png-----%%
that any triangle of either of these systems
will be congruent to the corresponding triangle of a
system into which $P_3$ may be decomposed. If we consider
simultaneously the two decompositions of $P_3$
we see that, in general, each triangle of the one decomposition
is broken up into polygons by the segments
which belong to the other decomposition. Let
ms add to these segments a sufficient number of others
to reduce each of these polygons to triangles, and
apply the two resulting methods of decompositions to
$P_1$ and $P_2$, thus breaking them up into corresponding
triangles. Then, evidently the two polygons $P_1$ and
$P_2$ are each decomposed into the same number of triangles,
which are respectively congruent by pairs.
Consequently, the two polygons are, by definition, of
equal area.

\begin{figure}[htb]
\begin{center}
\includegraphics*[width=5in]{images/f029.png}\\
{\small Fig. 29.}
\end{center}
\end{figure}

The proof of the second part of the theorem follows
without difficulty.

We define, in the usual manner, the terms: \emph{rectangle},
\emph{base} and \emph{height of a parallelogram}, \emph{base} and \emph{height
of a triangle}.

\newpage  % required for correct placement of Figures 30 and 31
\rfa{\S\!~19.\ {\small PARALLELOGRAMS AND TRIANGLES HAVING EQUAL BASES AND EQUAL ALTITUDES.}}\label{p19}

The well known reasoning of Euclid, illustrated
by the accompanying figure, furnishes a proof for the
following theorem:
%%-----File: 067.png-----%%
\begin{itemize}
\item[] {\scshape Theorem 25.} Two parallelograms having equal
bases and equal altitudes are also of equal content.
\end{itemize}
\medskip
\begin{figure}[!htb]
\begin{center}
\includegraphics*[width=4in]{images/f030.png}\\
{\small Fig. 30.}
\end{center}
\end{figure}
\medskip
We have also the following well known proposition:
\begin{itemize}
\item[]{\scshape Theorem 26.} Any triangle $ABC$ is always of
equal area to a certain parallelogram having
an equal base and an altitude half as great as
that of the triangle.
\end{itemize}

\begin{figure}[lhtb]
\begin{center}
\includegraphics*[width=2in]{images/f031.png}\\
{\small Fig. 31.}
\end{center}
\end{figure}

{\scshape Proof.} Bisect $AC$ in $D$
and $BC$ in $E$, and extend
the line $DE$ to $F$, making
$EF$ equal to $DE$. Then, the
triangles $DEC$ and $FBE$
are congruent to each other,
and, consequently, the triangle
$ABC$ and the parallelogram
$ABFD$ are of
equal area.

From theorems 25 and 26, we have at once, by aid
of theorem 24, the following proposition.

\begin{itemize}
\item[] {\scshape Theorem 27.} Two triangles having equal bases
and equal altitudes have also equal content.
\end{itemize}

It is usual to show that two triangles having equal
bases and equal altitudes are always of equal area. It
is to be remarked, however, that \emph{this demonstration
cannot be made without the aid of the axiom of
Archimedes.}
%%-----File: 068.png-----%%
In fact, we may easily construct in our non-archimedean
geometry (see \S~12, p.~\pageref{folio34}) two triangles
so that they shall have equal bases and equal altitudes
and, consequently, by theorem 27, must be of
equal content, but which are not, however, of equal
area. As such an example, we may take the two triangles
$ABC$ and $ABD$ having each the base $AB=1$
and the altitude $1$, where the vertex of the first triangle
is situated perpendicularly above $A$, and in the second
triangle the foot $F$ of the perpendicular let fall from
the vertex $D$ upon the base is so situated that $AF=t$.
The remaining propositions of elementary geometry
concerning the equal content of polygons, and
in particular the pythagorean theorem, are all simple
consequences of the theorems which we have already
given. In the further development of the theory of
area, we meet, however, with an essential difficulty.
In fact, the discussion so far leaves it still in doubt
whether all polygons are not, perhaps, of equal content.
In this case, all of the propositions which we
have given would be devoid of meaning and hence of
no value. Furthermore, the more general question
also arises as to whether two rectangles of equal content
and having one side in common, do not also have
their other sides congruent; that is to say, whether a
rectangle is not definitely determined by means of a
side and its area. As a closer investigation shows,
in order to answer this question, we need to make use
of the converse of theorem 27. This may be stated as
follows:

\begin{itemize}
\item[ ] {\scshape Theorem 28.} If two triangles have equal content
and equal bases, they have also equal altitudes.
\end{itemize}

%%-----File: 069.png-----%%
This fundamental theorem is to be found in the
first book of Euclid's Elements as proposition 39. In
the demonstration of this theorem, however, Euclid
appeals to the general proposition relating to magnitudes:
``$K \alpha \grave{\iota}
\ \tau \grave{o}
\ \check{o} \lambda o \upsilon
\ \tau o \hat{\nu}
\ \mu \acute{\epsilon} \rho o \upsilon \varsigma
\ \mu \epsilon \hat{\iota} \zeta \acute{o} \nu
\ \grave{\epsilon} \sigma \tau \iota \nu$''---a method
of procedure which amounts to the same thing as introducing
a new geometrical axiom concerning areas.

It is now possible to establish the above theorem
and hence the theory of areas in the manner we have
proposed, that is to say, with the help of the plane
axioms and without making use of the axiom of Archimedes.
In order to show this, it is necessary to introduce
the idea of the measure of area.

\bigskip
\rfa{\S\!~20.\ {\small THE MEASURE OF AREA OF TRIANGLES AND POLYGONS.}}\label{p20}
\bigskip

{\scshape Definition.}  If in a triangle $ABC$, having the
sides $a$, $b$, $c$, we construct the two altitudes $h_a = AD$,
$h_b = BE$, then, according to theorem 22, it follows
from the similarity of the triangles $BCE$ and $ACD$,
that we have the proportion
\[
a: h_b = b: ha;
\]

that is, we have
\[
a \cdot h_a = b \cdot h_b.
\]

\begin{figure}[htb]
\begin{center}
\includegraphics*[width=2in]{images/f032.png}\\
{\small Fig. 32.}
\end{center}
\end{figure}

This shows that the product of the base and the corresponding
altitude of a triangle is the same whichever
side is selected as the base. The half of this
product of the base and the altitude of a triangle $\Delta$ is
called the \emph{measure of area of the triangle $\Delta$} and we denote
it by $F(\Delta)$.
%%-----File: 070.png-----%%
A segment joining a vertex of a triangle with a
point of the opposite side is called \emph{a transversal.} A
transversal divides the given triangle into two others
having the same altitude and having bases which lie
in the same straight line. Such a decomposition is
called a \emph{transversal decomposition of the triangle.}

\begin{itemize}
\item[ ] {\scshape Theorem 29.} If a triangle $\Delta$ is decomposed by
means of arbitrary straight lines into a finite
number of triangles $\Delta_k$, then the measure of
area of $\Delta$ is equal to the sum of the measures
of area of the separate triangles $\Delta_k$.
\end{itemize}

{\scshape Proof.} From the distributive law of our algebra
of segments, it follows immediately that the measure
of area of an arbitrary triangle is equal to the sum of
the measures of area of two such triangles as arise
from any transversal decomposition
of the given
triangle. The repeated
application of this proposition
shows that the
measure of area of any
triangle is equal to the
sum of the measures of
area of all the triangles arising by applying the transversal
decomposition an arbitrary number of times in
succession.

\begin{wrapfigure}[12]{L}{5.4cm}
\begin{center}
\includegraphics*[width=2in]{images/f033.png}\\
{\small Fig. 33.}
\end{center}
\end{wrapfigure}

In order to establish the corresponding proof for
an arbitrary decomposition of the triangle $\Delta$ into the
triangles $\Delta_k$, draw from the vertex $A$ of the given triangle
$\Delta$ a transversal through each of the points of
division of the required decomposition; that is to say,
draw a transversal through each vertex of the triangles
$\Delta_k$. By means of these transversals, the given triangle
%%-----File: 071.png-----%%
$\Delta$ is decomposed into certain triangles $\Delta_t$. Each
of these triangles $\Delta_t$ is broken up by the segments
which determined this decomposition
into certain triangles
and quadrilaterals. If, now, in
each of the quadrilaterals, we
draw a diagonal, then each triangle
$\Delta_t$ is decomposed into
certain other triangles $\Delta_{ts}$. We
shall now show that the decomposition
into the triangles
$\Delta_{ts}$ is for the triangles $\Delta_t$, as
well as for the triangles $\Delta_k$,
nothing else than a series of
transversal decompositions. In fact, it is at once evident
that any decomposition of a triangle into partial
triangles may always be affected by a series of transversal
decompositions, providing, in this decomposition,
points of division do not exist within the triangle,
and further, that at least one side of the triangle remains
free from points of division.

\begin{wrapfigure}[17]{L}{5.4cm}
\begin{center}
\includegraphics*[width=2in]{images/f034.png}\\
{\small Fig. 34.}
\end{center}
\end{wrapfigure}

We easily see that these conditions hold for the
triangles $\Delta_t$. In fact, the interior of each of these triangles,
as also one side, namely, the side opposite the
point $A$, contains no points of division.

Likewise, for each of the triangles $\Delta_k$, the decomposition
into $\Delta_{ts}$ is reducible to transversal decompositions.
Let us consider a triangle $\Delta_k$. Among the
transversals of the triangle $\Delta$ emanating from the
point $A$, there is always a definite one to be found
which either coincides with a side of $\Delta_k$, or which itself
divides $\Delta_k$ into two triangles. In the first case,
the side in question always remains free from further
points of division by the decomposition into the triangles
%%-----File: 072.png-----%%
$\Delta_{ts}$. In the second case, the segment of the
transversal contained within the triangle $\Delta_k$ is a side
of the two triangles arising from the division, and this
side certainly remains free from further points of division.

According to the considerations set forth at the beginning
of this demonstration, the measure of area
$F(\Delta)$ of the triangle $\Delta$ is equal to the sum of the
measures of area $F(\Delta_t)$ of all the triangles $\Delta_t$ and this
sum is in turn equal to the sum of all the measures of
area $F(\Delta_{ts})$. However, the sum of the measures of
area $F(\Delta_k)$ of all the triangles $\Delta_k$ is also equal to the
sum of the measures of area $F(\Delta_{ts})$. Consequently,
we have finally that the measure of area $F(\Delta)$ is also
equal to the sum of all the measures of area $F(\Delta_k)$,
and with this conclusion our demonstration is completed.

{\scshape Definition.} If we define the measure of area $F(P)$
of a polygon as the sum of the measures of area of all
the triangles into which the polygon is, by a definite
decomposition, divided, then upon the basis of theorem
29 and by a process of reasoning similar to that
which we have employed in �~18 to prove theorem 24,
we know that the measure of area of a polygon is independent
of the manner of decomposition into triangles
and, consequently, is definitely determined by the polygon
itself. From this we obtain, by aid of theorem
29, the result that \emph{polygons of equal area have also equal
measures of area}.

Furthermore, if $P$ and $Q$ are two polygons of equal
content, then there must exist, according to the above
definition, two other polygons $P'$ and $Q'$ of equal area,
such that the polygon composed of $P$ and $P'$ shall be
%%-----File: 073.png-----%%
of equal area with the polygon formed by combining
the polygons $Q$ and $Q'$. From the two equations
\begin{eqnarray*}
F(P+P') & = & F(Q + Q') \\
F(P') & = & F(Q'),
\end{eqnarray*}
we easily deduce the equation
\[
F(P) = F(Q);
\]
that is to say, \emph{polygons of equal content have also equal
measure of area}.

From this last statement, we obtain immediately
the proof of theorem 28. If we denote the equal bases
of the two triangles by $g$ and the corresponding altitudes
by $h$ and $h'$, respectively, then we may conclude
from the assumed equality of content of the two triangles
that they must also have equal measures of
area; that is to say, it follows that
\[
\frac{1}{2}gh = \frac{1}{2}gh'
\]
and, consequently, dividing by $\frac{1}{2}g$, we get
\[
h = h',
\]
which is the statement made in theorem 28.

\rfa{\S\!~21.\ {\small EQUALITY OF CONTENT AND THE MEASURE OF AREA.}}\label{p21}

In \S~20 we have found that polygons having equal
content have also equal measures of area. The converse
of this is also true.

In order to prove the converse, let us consider two
triangles $ABC$ and $AB'C'$ having a common right
angle at $A$. The measures of area of these two triangles
are expressed by the formul{\ae}
\begin{eqnarray*}
F(ABC) & = & \frac{1}{2}AB \cdot AC, \\
F(AB'C') & = & \frac{1}{2}AB' \cdot AC'.
\end{eqnarray*}
%%-----File: 074.png-----%%
We now assume that these measures of area are equal
to each other, and consequently we have
\[
AB \cdot AC = AB' \cdot AC',
\]
or
\[
AB:AB' = AC':AC.
\]

\begin{figure}[htb]
\begin{center}
\includegraphics*[width=3in]{images/f035.png}\\
{\small Fig. 35.}
\end{center}
\end{figure}

From this proposition, it follows, according to theorem
23, that the two straight lines $BC'$ and $B'C$ are
parallel, and hence, by theorem 27, the two triangles
$BC'B'$ and $BC'C$ are of equal content. By the addition
of the triangle $ABC'$, it follows that the two triangles
$ABC$ and $AB'C'$ are of equal content. We
have then shown that two right triangles having the
same measure of area are always of equal content.

Take now any arbitrary triangle having the base $g$
and the altitude $h$. Then, according to theorem 27,
it has equal content with a right triangle having the
two sides $g$ and $h$. Since the original triangle had
evidently the same measure of area as the right triangle,
it follows that, in the above consideration, the
restriction to right triangles was not necessary. Hence,
two \textit{arbitrary triangles with equal measures of area are
also of equal content}.

Moreover, let us suppose $P$ to be any polygon
%%-----File: 075.png-----%%
having the measure of area $g$ and let $P$ be decomposed
into $n$ triangles having respectively the measures of
area $g_1$, $g_2$, $g_3$, $\ldots$, $g_n$. Then, we have
\[
g = g_1 + g_2 + g_3 + \cdots + g_n.
\]

Construct now a triangle $ABC$ having the base
$AB=g$ and the altitude $h = 1$. Take, upon the base
of this triangle, the points $A_1$, $A_2$, $\ldots$, $A_{n-1}$ so that
$g_1=AA_1$, $g_1=A_1A_2$, $\ldots$, $g_{n-1}=A_{n-2}A_{n-1}$,
$g_n=A_{n-1}B$.

\begin{figure}[htb]
\begin{center}
\includegraphics*[width=4in]{images/f036.png}\\
{\small Fig. 36.}
\end{center}
\end{figure}

Since the triangles composing the polygon $P$ have respectively
the same measures of area as the triangles
$AA_1C$, $A_1A_2C$, $\ldots$, $A_{n-2}A_{n-1}C$, $A_{n-1}BC$, it follows
from what has already been demonstrated that they
have also the same content as these triangles. Consequently,
the polygon $P$ and a triangle, having the
base $g$ and the altitude $h = 1$ are of equal content.
From this, it follows, by the application of theorem
24, that two polygons having equal measures of area
are always of equal content.

We can now combine the proposition of this section
with that demonstrated in the last, and thus obtain
the following theorem:

\begin{itemize}
\item[]{\scshape Theorem 30.} Two polygons of equal content
have always equal measures of area. Conversely,
%%-----File: 076.png-----%%
two polygons having equal measures
of area are always of equal content.
\end{itemize}

In particular, if two rectangles are of equal content
and have one side in common, then their remaining
sides are respectively congruent. Hence, we have the
following proposition:

\begin{itemize}
\item[]{\scshape Theorem 31.} If we decompose a rectangle into
several triangles by means of straight lines and
then omit one of these triangles, we can no
longer make up completely the rectangle from
the triangles which remain.
\end{itemize}

This theorem has been demonstrated by F. Schur\footnote{\textit{Sitzungsberichte
der Dorpater Naturf. Ges.} 1892.}
and by W. Killing,\footnote{\textit{Grundlagen der Geometrie},
Vol. 2, Chapter 5, \S~5, 1898.}
but by making use of the axiom
of Archimedes. By O. Stolz,\footnote{\textit{Monatshefte f\"ur Math, und Phys.} 1894.}
it has been regarded
as an axiom. In the foregoing discussion, it has been
shown that it is completely independent of the axiom
of Archimedes. However, when we disregard the axiom
of Archimedes, this theorem (31) is not sufficient
of itself to enable us to demonstrate Euclid's theorem
concerning the equality of altitudes of triangles
having equal content and equal bases. (Theorem 28.)

In the demonstration of theorems 28, 29, and 30,
we have employed essentially the algebra of segments
introduced in \S~15, p.~\pageref{folio46}, and as this depends substantially
upon Pascal's theorem (theorem 21), we see
that this theorem is really the corner-stone in the theory
of areas. We may, by the aid of theorems 27 and
28, easily establish the converse of Pascal's theorem.

Of two polygons $P$ and $Q$, we call $P$ the smaller
or larger in content according as the measure of area
%%-----File: 077.png-----%%
\label{folio70}
$F(P)$ is less or greater than the measure of area $F(Q)$.
From what has already been said, it is clear that the
notions, equal content, smaller content, larger content,
are mutually exclusive. Moreover, we easily see
that a polygon, which lies wholly within another polygon,
must always be of smaller content than the exterior
one.

With this we have established the important theorems
in the theory of areas.
\newpage
%%-----File: 078.png-----%%
\begin{center}{\Large DESARGUES'S THEOREM.}\end{center}
\rfa{\S\!~22.\ {\small DESARGUES'S THEOREM AND ITS DEMONSTRATION FOR PLANE GEOMETRY BY AID OF THE AXIOMS OF CONGRUENCE.}}\label{p22}

Of the axioms given in \S\S~1--8, pp.~\pageref{folio1}--\pageref{folio26}, those
of groups II--V are in part linear and in part
plane axioms. Axioms 3--7 of group I are the only
space axioms. In order to show clearly the significance
of these axioms of space, let us assume a plane
geometry and investigate, in general, the conditions
for which this plane geometry may be regarded as a
part of a geometry of space in which at least the axioms
of groups I--III are all fulfilled.

Upon the basis of the axioms of groups I--III, it is
well known that the so-called theorem of Desargues
may be easily demonstrated. This theorem relates to
points of intersection in a plane. Let us assume in
particular that the straight line, upon which are situated
the points of intersection of the homologous
sides of the two triangles, is the straight line which
we call the straight line at infinity. We will designate
the theorem which arises in this case, together
with its converse, as the theorem of Desargues. This
theorem is as follows:

\begin{itemize}
\item[]{\scshape Theorem 32.} (Desargues's theorem.) When two
triangles are so situated in a plane that their
homologous sides are respectively parallel, then
%%-----File: 079.png-----%%
the lines joining the homologous vertices pass
through one and the same point, or are parallel
to one another.
\end{itemize}

Conversely, if two triangles are so situated
in a plane that the straight lines joining the
homologous vertices intersect in a common
point, or are parallel to one another, and, furthermore,
if two pairs of homologous sides are
parallel to each other, then the third sides of
the two triangles are also parallel to each other.

\begin{figure}[htb]
\begin{center}
\includegraphics*[width=3in]{images/f037.png}\\
{\small Fig. 37.}
\end{center}
\end{figure}

As we have already mentioned, theorem 32 is a
consequence of the axioms I--III\@. Because of this
fact, the validity of Desargues's theorem in the plane
is, in any case, a necessary condition that the geometry
of this plane may be regarded as a part of a geometry
of space in which the axioms of groups I--III are
all fulfilled.

Let us assume, as in \S\S~13--21, pp.~\pageref{folio37}--\pageref{folio70}, that we
have a plane geometry in which the axioms I, 1--2 and
II--IV all hold and, also, that we have introduced in
this geometry an algebra of segments conforming to
\S~15.

Now, as has already been established in \S~17, there
%%-----File: 080.png-----%%
may be made to correspond to each point in the plane
a pair of segments $(x, y)$ and to each straight line a
ratio of three segments $(u:v:w)$, so that the linear
equation
\[
  ux+vy+w=0
\]
expresses the condition that the point is situated upon
the straight line. The system composed of all the
segments in our geometry forms, according to \S~17, a
domain of numbers for which the properties (1--16),
enumerated in \S~13, are valid. We can, therefore, by
means of this domain of numbers, construct a
geometry of space in a manner similar to that already
employed in \S~9 or in \S~12, where we made use of the
systems of numbers $\Omega$ and $\Omega(t)$, respectively. For
this purpose, we assume that a system of three
segments $(x, y, z)$ shall represent a point, and that the
ratio of four segments $(u:v:w:r)$ shall represent a
plane, while a straight line is defined as the
intersection of two planes. Hence, the linear equation
\[
ux+vy+wz+r = 0
\]
expresses the fact that the point $(x, y, z)$ lies in the
plane $(u: v: w: r)$. Finally, we determine the
arrangement of the points upon a straight line, or the points
of a plane with respect to a straight line situated in
this plane, or the arrangement of the points in space
with respect to a plane, by means of inequalities in a
manner similar to the method employed for the plane
in \S~9.

Since we obtain again the original plane geometry
by putting $z=0$, we know that our plane geometry
can be regarded as a part of geometry of space. Now,
the validity of Desargues's theorem is, according to the
above considerations, a necessary condition for this
%%-----File: 081.png-----%%
result.  Hence, in the assumed plane geometry, it
follows that Desargues's theorem must also hold.

It will be seen that the result just stated may also
be deduced without difficulty from theorem 23 in the
theory of proportion.

\rfa{\S\!~23.\ {\small THE IMPOSSIBILITY OF DEMONSTRATING DESARGUES'S THEOREM FOR THE
PLANE WITHOUT THE HELP OF THE AXIOMS OF CONGRUENCE.}\footnote{See
also a recent paper by F. R. Moulton on ``Simple Non-desarguesian Geometry,''
\textit{Transactions of the Amer. Math. Soc.}, April, 1902.---\textit{Tr.}}}\label{p23}

We shall now investigate the question whether or
no in plane geometry Desargues's theorem may be
deduced without the assistance of the axioms of congruence.
This leads us to the following result:

\begin{itemize}
\item[]{\scshape Theorem 33.} A plane geometry exists in which
the axioms I 1--2, II--III, IV 1--5, V, that is to
say, all linear and all plane axioms with the
exception of axiom IV, 6 of congruence, are
fulfilled, but in which the theorem of Desargues
(theorem 32) is not valid. Desargues's theorem
is not, therefore, a consequence solely of the
axioms mentioned; for, its demonstration necessitates
either the space axioms or all of the
axioms of congruence.
\end{itemize}


{\scshape Proof.} Select in the ordinary plane geometry (the
possibility of which has already been demonstrated in
\S~9, pp.~\pageref{folio27}--\pageref{folio30}) any two straight lines perpendicular
to each other as the axes of $x$ and $y$. Construct about
the origin $O$ of this system of co-ordinates an ellipse
having the major and minor axes equal to 1 and $\frac{1}{2}$, respectively.
Finally, let $F$ denote the point situated
upon the positive $x$-axis at the distance $\frac{3}{2}$ from $O$.
%%-----File: 082.png-----%%
Consider all of the circles which cut the ellipse in
four real points. These points may be either distinct
or in any way coincident. Of all the points situated
upon these circles, we shall attempt to determine the
one which lies upon the $x$-axis farthest from the origin.
For this purpose, let us begin with an arbitrary
circle cutting the ellipse in four distinct points and
intersecting the positive $x$-axis in the point $C$. Suppose
this circle then turned about the point $C$ in
such a manner that two or more of the four points of
intersection with the ellipse finally coincide in a single
point $A$, while the rest of them remain real. Increase
now the resulting tangent circle in such a way that $A$
always remains a point of tangency with the ellipse.
In this way we obtain a circle which is either tangent
to the ellipse in also a second point $B$, or which has
with the ellipse a four-point contact in $A$. Moreover,
this circle cuts the positive $x$-axis in a point more remote
than $C$. The desired farthest point will accordingly
be found among those points of intersection of
the positive $x$-axis by circles lying exterior to the
ellipse and being doubly tangent to it. All such circles
must lie, as we can easily see, symmetrically with
respect to the $y$-axis. Let $a$, $b$ be the co-ordinates of
any point on the ellipse. Then an easy calculation
shows that the circle, which is symmetrical with respect
to $y$-axis and tangent to the ellipse at this point,
cuts off from the positive $x$-axis the segment
\[
x = \mid \sqrt{1 + 3b^2} \mid.
\]
The greatest possible value of this expression occurs
for $b = \frac{1}{2}$ and, hence, is equal to $\frac{1}{2}\mid\sqrt{7}\mid$. Since the
point on the $x$-axis which we have denoted by $F$ has
for its abscissa the value $\frac{3}{2} > \frac{1}{2} \mid\sqrt{7}\mid$, it follows that
%%-----File: 083.png-----%%
\emph{among the circles cutting the ellipse four times there is
certainly none which passes through the point $F$.}

We will now construct a new plane geometry in
the following manner. As points in this new geometry,
let us take the points of the $(xy)$-plane. We
will define a straight line of our new geometry in the
following manner. Every straight line of the $(xy)$-plane
which is either tangent to the fixed ellipse, or
does not cut it at all, is taken unchanged as a straight
line of the new geometry. However, when any straight
line $g$ of the $(xy)$-plane cuts the ellipse, say in the
points $P$ and $Q$, we will then define the corresponding

\begin{figure}[htb]
\begin{center}
\includegraphics*[width=4in]{images/f038.png}\\
{\small Fig. 38.}
\end{center}
\end{figure}

straight line of the new geometry as follows. Construct
a circle passing through the points $P$ and $Q$
and the fixed point $F$. From what has just been said,
this circle will have no other point in common with
%%-----File: 084.png-----%%
the ellipse. We will now take the broken line, consisting
of the arc $PQ$ just mentioned and the two
parts of the straight line $g$ extending outward indefinitely
from the points $P$ and $Q$, as the required straight
line in our new geometry. Let us suppose all of the
broken lines constructed which correspond to straight
lines of the $(xy)$-plane. We have then a system of
broken lines which, considered as straight lines of our
new geometry, evidently satisfy the axioms I, 1--2 and
III\@. By a convention as to the actual arrangement
of the points and the straight lines in our new geometry,
we have also the axioms II fulfilled.

Moreover, we will call two segments $AB$ and $A'B'$
congruent in this new geometry, if the broken line
extending between $A$ and $B$ has equal length, in the
ordinary sense of the word, with the broken line extending
from $A'$ to $B'$.

Finally, we need a convention concerning the congruence
of angles. So long as neither of the vertices
of the angles to be compared lies upon the ellipse, we
call the two angles congruent to each other, if they
are equal in the ordinary sense. In all other cases
we make the following convention. Let $A, B, C$ be
points which follow one another in this order upon a
straight line of our new geometry, and let $A', B', C'$
be also points which lie in this order upon another
straight line of our new geometry. Let $D$ be a point
lying outside of the straight line $ABC$ and $D'$ be a
point outside of the straight $A'B'C'$. We will now
say that, in our new geometry, the angles between
these straight lines fulfill the congruences
\[
 \angle ABD \equiv \angle A'B'D' \mbox{ and } \angle CBD \equiv \angle C'B'D'
\]
whenever the natural angles between the corresponding
%%-----File: 085.png-----%%
broken lines of the ordinary geometry fulfill the
proportion
\[
\angle ABD: \angle CBD = \angle A'B'D' : \angle C'B'D'.
\]
These conventions render the axioms IV, 1--5 and V
valid.

\begin{figure}[htb]
\begin{center}
\includegraphics*[width=4in]{images/f039.png}\\
{\small Fig. 39.}
\end{center}
\end{figure}

In order to see that Desargues's theorem does not
hold for our new geometry, let us consider the following
three ordinary straight lines of the $(xy)$-plane;
viz., the axis of $x$, the axis of $y$, and the straight line
joining the two points of the ellipse $(\frac{3}{5},\frac{2}{5})$ and
$(-\frac{3}{5},-\frac{2}{5})$.
Since these three ordinary straight lines pass
through the origin, we can easily construct two triangles
so that their vertices shall lie respectively upon
these three straight lines and their homologous sides
shall be parallel and all three sides shall lie exterior to
the ellipse. As we may see from figure 40, or show
%%-----File: 086.png-----%%
\label{folio79}
by an easy calculation, the broken lines arising from
the three straight lines in question do not intersect in
a common point. Hence, it follows that Desargues's
theorem certainly does not hold for this particular
plane geometry in which we have constructed the two
triangles just considered.
\begin{figure}[htb]
\begin{center}
\includegraphics*[width=4in]{images/f040.png}\\
{\small Fig. 40.}
\end{center}
\end{figure}

This new geometry serves at the same time as an
example of a plane geometry in which the axioms I,
1--2, II--III, IV, 1--5, V all hold, but which cannot be
considered as a part of a geometry of space.
\vspace{2cm}
\rfa{\S\!~24.\ {\small INTRODUCTION OF AN ALGEBRA OF SEGMENTS
BASED UPON DESARGUES'S THEOREM AND INDEPENDENT
OF THE AXIOMS OF CONGRUENCE.\footnote{Discussed
also by Moore in a paper before the Am. Math. Soc., Jan,
1902. See \textit{Trans. Am. Math. Soc.---Tr.}}}}\label{p24}
\bigskip

In order to see fully the significance of Desargues's
theorem (theorem 32), let us take as the basis of our
consideration a plane geometry where all of the axioms
%%-----File: 087.png-----%%
I 1--2, II--III are valid, that is to say, where all
of the plane axioms of the first three groups hold, and
then introduce into this geometry, in the following
manner, a new algebra of segments independent of
the axioms of congruence.

Take in the plane two fixed straight lines intersecting
in $O$, and consider only such segments as have
$O$ for their origin and their other extremity in one of
the fixed lines. We will regard the point $O$ itself as
a segment and call it the segment $O$. We will indicate
this fact by writing
\[
 OO = 0, \mbox{ or } 0 = OO.
\]

Let $E$ and $E'$ be two definite points situated respectively
upon the two fixed straight lines through
$O$. Then, define the two segments $OE$ and $OE'$ as
the segment 1 and write accordingly
\begin{equation}
OE = OE' = 1 \mbox{\ or\ } 1 = OE = OE'.
\end{equation}
We will call the straight line $EE'$, for brevity, the
unit-line. If, furthermore, $A$ and $A'$ are points upon

\begin{figure}[htb]
\begin{center}
\includegraphics*[width=3in]{images/f041.png}\\
{\small Fig. 41.}
\end{center}
\end{figure}

the straight lines $OE$ and $OE'$, respectively, and, if
the straight line $AA'$ joining them is parallel to $EE'$,
%%-----File: 088.png-----%%
then we will say that the segments OA and OA' are
equal to one another, and write
\[
                OA=OA' \mbox{ or } OA'=OA.
\]
In order now to define the sum of the segments
$a=OA$ and $b= OB$, we construct $AA'$ parallel to the
unit-line $EE'$ and draw through $A'$ a parallel to $OE$
and through $B$ a parallel to $OE'$. Let these two
parallels intersect in $A''$. Finally, draw through $A''$ a
straight line parallel to the unit-line $EE'$. Let this
parallel cut the two fixed lines $OE$ and $OE'$ in $C$ and
$C'$ respectively. Then $c=OC= OC'$ is called the
\emph{sum} of the segments $a=OA$ and $b=OB$. We indicate
this by writing
\[
                   c=a+b, \mbox{ or } a+b=c.
\]
In order to define the product of a segment $a=OA$
by a segment $b=OB$, we make use of exactly the
same construction as employed in \S~15, except that,
in place of the sides of a right angle, we make use
here of the straight lines $OE$ and $OE'$. 
\begin{figure}[htb]
\begin{center}
\includegraphics*[width=4in]{images/f042.png}\\
{\small Fig. 42.}
\end{center}
\end{figure}
The construction
is consequently as follows. Determine upon $OE'$
a point $A'$ so that $AA'$ is parallel to the unit-line $EE'$,
and join $E$ with $A'$. Then draw through $B$ a straight
%%-----File: 089.png-----%%
line parallel to $EA'$. This parallel will intersect the
fixed straight line $OE'$ in the point $C'$, and we call
$c=OC'$ the product of the segment $a = OA$ by the
segment $b = OB$. We indicate this relation by writing
\[
 c = ab, \mbox{ or } ab = c.
\]

\rfa{\S\!~25.\ {\small THE COMMUTATIVE AND THE ASSOCIATIVE LAW OF ADDITION FOR OUR NEW ALGEBRA OF SEGMENTS.}}\label{p25}

In this section, we shall investigate the laws of
operation, as enumerated in \S~13, in order to see which
of these hold for our new algebra of segments, when
we base our considerations upon a plane geometry in
which axioms I 1--2, II--III are all fulfilled, and, moreover,
in which Desargues's theorem also holds.

First of all, we shall show that, for the addition of
segments as defined in \S~24, the commutative law
\[
 a+ b = b+a
\]
holds. Let
\begin{eqnarray*}
 & a = OA = OA' & \\
 & b = OB = OB' &
\end{eqnarray*}

\begin{wrapfigure}[12]{L}{5.4cm}
\begin{center}
\includegraphics*[width=2in]{images/f043.png}\\
{\small Fig. 43.}
\end{center}
\end{wrapfigure}
Hence, $AA'$ and $BB'$ are, according to our convention, parallel to the unit-line.
Construct the points $A''$ and $B''$ by drawing $A'A''$ and
$B'B''$ parallel to $OA$ and also $AB''$ and $BA'$ parallel to $OA$.
We see at once that the line $A''B''$ is parallel to $AA'$ as the commutative
law requires. We shall show the validity of this statement by the aid of
Desargues's theorem in
%%-----File: 090.png-----%%
the following manner. Denote the point of
intersection of $AB''$ and $A'A''$ by $F$ and that of $BA''$ and $B'B''$
by $D$. Then, in the triangles $AA'F$ and $BB'D$, the
homologous sides are parallel to each other. By
Desargues's theorem, it follows that the three points
$O$, $F$, $D$ lie in a straight line. In consequence of this
condition, the two triangles $OAA'$ and $DB''A''$ lie in
such a way that the lines joining the corresponding
vertices pass through the same point $F$, and since the
homologous sides $OA$ and $DB''$, as also $OA'$ and $DA''$,
are parallel to each other, then, according to the
second part of Desargues's theorem (theorem 32), the
third sides $AA'$ and $B''A''$ are parallel to each other.

To prove the associative law of addition
\[
  a+(b+c)=(a+b)+c,
\]
we shall make use of figure 44. In consequence of
the commutative law of addition just demonstrated,
the above formula states that the straight line $A''B''$
must be parallel to the unit-line. 
\begin{figure}[htb]
\begin{center}
\includegraphics*[width=3in]{images/f044.png}\\
{\small Fig. 44.}
\end{center}
\end{figure}
The validity of this statement is evident, since the shaded part of figure
44 corresponds exactly with figure 43.

%%-----File: 091.png-----%%
\bigskip
\rfa{\S\!~26.\ {\small THE ASSOCIATIVE LAW OF MULTIPLICATION AND THE TWO DISTRIBUTIVE LAWS FOR THE NEW ALGEBRA OF SEGMENTS.}}\label{p26}
\bigskip

The associative law of multiplication
\[
  a(bc)=(ab)c
\]
has also a place in our new algebra of segments.

Let there be given upon the first of the two fixed
straight lines through $O$ the segments
\[
  1 = OA, \quad b = OC, \quad c=OA'
\]

\begin{figure}[htb]
\begin{center}
\includegraphics*[width=4in]{images/f045.png}\\
{\small Fig. 45.}
\end{center}
\end{figure}

and upon the second of these straight lines,
the segments
\[
  a=OG,\quad  b=OB.
\]

In order to construct the segments
\begin{eqnarray*}
bc & = & OB',\ \mbox{and}\ bc=OC', \\
ab & = & OD, \\
(ab) c &=& OD',
\end{eqnarray*}
%%-----File: 092.png-----%%
in accordance with \S 24, draw $A'B'$ parallel to $AB$,
$B'C'$ parallel to $BC$, $CD$ parallel to $AG$, and $A'D'$
parallel to $AD$. We see at once that the given law
amounts to the same as saying that $CD$ must also be
parallel to $C'D'$. Denote the point of intersection of
the straight lines $A'D'$ and $B'C'$ by $F'$ and that of the
straight lines $AD$ and $BC$ by $F$. Then the triangles
$ABF$ and $A'B' F'$ have their homologous sides
parallel to each other, and, according to Desargues's
theorem, the three points $O$, $F$, $F'$ must lie in a
straight line. Because of these conditions, we can
apply the second part of Desargues's theorem to the
two triangle $CDF$ and $C' D'F'$, and hence show that,
in fact, $CD$ is parallel to $C'D'$.

Finally, upon the basis of Desargues's theorem,
we shall show that the two distributive laws
\[
  a (b+c) = ab+ac
\]
and
\[
  (a+b)c = ac+bc
\]
hold for our algebra of segments.

In the proof of the first one of these laws, we shall
make use of
figure 46.\footnote{Figures 46, 47, and 48 were designed
by Dr. Von Schaper, as have also
the details of the demonstrations relating to
these figures.}
In this figure, we have
\[
\begin{array}{c}
b = OA', c=OC', \\
ab=OB', ab=OA'', ac=OC'', etc.
\end{array}
\]
In the same figure, $B'' D_2$ is parallel to $C''D_1$ which
is parallel to the fixed straight line $OA'$, and $B'D_1$ is
parallel to $C'D_2$, which is parallel to the fixed straight
line $OA''$. Moreover, we have $A' A''$ parallel to $C'C''$,
and $A'B''$ parallel to $B' A''$, parallel to $F'D_2$, parallel
to $F''D_1$.

%%-----File: 093.png-----%%
Our proposition amounts to asserting that we must
necessarily have also
\begin{center}
$F'F''$ parallel to $A'A''$ and to $C'C''$.
\end{center}

We construct the following auxiliary lines:

\begin{center}
\begin{tabular}{cccccccc}
$F''J$ & \rfb parallel & \rfb to & \rfb the & \rfb fixed & \rfb straight & \rfb line & \rfb $OA'$, \\
\vspace{-0.2cm} \\
$F'J$ & \rfb`` & \rfb`` & \rfb`` & \rfb`` & \rfb`` & \rfb`` & \rfb $OA''$. \\
\end{tabular}
\end{center}

Let us denote the points of intersection of the straight
lines $C''D_1$ and $C'D_2$, $C''D_1$ and $F'J$, $C'D_2$ and $F''J$
by $G$, $H_1$, $H_2$, respectively. Finally, we obtain the
other auxiliary lines indicated in the figure by joining
the points already constructed.

\begin{figure}[htb]
\begin{center}
\includegraphics*[width=4in]{images/f046.png}\\
{\small Fig. 46.}
\end{center}
\end{figure}

In the two triangles $A' B'' C''$ and $F'D_2G$, the straight
lines joining homologous vertices are parallel to each
other. According to the second part of Desargues's
theorem, it follows, therefore, that
\begin{center}
$A'C''$ is parallel to $F' G$.
\end{center}
In the two triangles $A'C''F''$ and $F'GH_2$, the straight
lines joining the homologous vertices are also
parallel to each other. From the properties already demonstrated,
%%-----File: 094.png-----%%
it follows by virtue of the second part of
Desargues's theorem that we must have
\begin{center}
 $A'F''$ parallel to $F'H_2$.
\end{center}

Since in the two horizontally shaded triangles $OA'F'$
and $JH_2F'$ the homologous sides are parallel, Desargues's
theorem shows that the three straight lines
joining the homologous vertices, viz.:
\begin{equation}
 OJ,\ A'H_2,\ F''F'
\end{equation}
all intersect in one and the same point, say in $P$.

In the same way, we have necessarily
\[
 A''F' \mbox{ parallel to } F''H_1
\]
and since, in the two obliquely shaded triangles $OA''F'$
and $JH_1F''$, the homologous sides are parallel, then,
in consequence of Desargues's theorem, the three
straight lines joining the homologous vertices, viz.:
\[
 OJ, A''H_1, F'F''
\]
all intersect likewise in the same point, namely, in
point $P$.

Moreover, in the triangles $OA'A''$ and $JH_2H_1$, the
straight lines joining the homologous vertices all pass
through this same point $P$, and, consequently, it follows
that we have
\[
 H_1H_2 \mbox{ parallel to } A'A'',
\]
and, therefore,
\[
 H_1H_2 \mbox{ parallel to } C'C'',
\]
Finally, let us consider the figure $F''H_2C'C''H_1F'F''$.
Since, in this figure, we have
\begin{eqnarray*}
 F''H_2 \mbox{ parallel to} & C'F',       & \mbox{parallel to } C''H_1, \\
 GH_2C' \quad \mbox{`` ``}\quad & F''C'', & \mbox{`` ``}\quad H_1F', \\
 C'C''  \quad \mbox{`` ``}\quad & H_1H_2, &
\end{eqnarray*}
%%-----File: 095.png-----%%
we recognize here again figure 43, which we have
already made use of in \S~25 to prove the commutative
law of addition. The conclusions, analogous to those
which we reached there, show that we must have
\[
 F'F'' \mbox{ parallel to } H_1H_2
\]
and, consequently, we must have also
\[
 F'F'' \mbox{ parallel to} A'A'',
\]
which result concludes our demonstration.

To prove the second formula of the distributive
law, we make use of an entirely different figure,----figure 47.
In this figure, we have

\begin{figure}[htb]
\begin{center}
\includegraphics*[width=4in]{images/f047.png}\\
{\small Fig. 47.}
\end{center}
\end{figure}

\[
 1 = OD,\ a = OA,\ a = OB,\ b = OG,\ c = OD',
\]
\[
 ac = OA',\ ac = OB',\ BC = OG',\mbox{ etc.},
\]
and, furthermore, we have

\begin{center}
\begin{tabular}{cccccccccc}
$GH$ & \rfb parallel & \rfb to & \rfb $G'H'$, & \rfb parallel & \rfb to & \rfb the & \rfb fixed & \rfb line & \rfb $OA$, \\
\vspace{-0.2cm} \\
$GH$ & \rfb `` & \rfb `` & \rfb $A'H'$, & \rfb `` & \rfb `` & \rfb `` & \rfb `` & \rfb `` & \rfb $OB$, \\
\end{tabular}
\end{center}

We have also
\begin{eqnarray*}
 AB & \mbox{parallel to} & A'B' \\
 BD & \mbox{`` ``} & B'D' \\
 DG & \mbox{`` ``} & D'G' \\
 HJ & \mbox{`` ``} & H'J'.
\end{eqnarray*}

%%-----File: 096.png-----%%
\label{folio89}
That which we are to prove amounts, then, to demonstrating that
\[
DJ \mbox{ must be parallel to } D'J'.
\]

Denote the points in which $BD$ and $GD$ intersect
the straight line $AH$ by $C$ and $F$, respectively, and
the points in which $B'D'$ and $G'D'$ intersect the straight
line $A'H'$ by $C'$ and $F'$, respectively. Finally, draw
the auxiliary lines $FJ$ and $F'J'$, indicated in the figure
by dotted lines.

In the triangles $ABC$ and $A'B'C'$, the homologous
sides are parallel and, consequently, by Desargues's
theorem the three points $O$, $C$, $C'$ lie on a straight
line. Then, by considering in the same way the triangles
$CDF$ and $CD'F'$, it follows that the points
$O$, $F$, $F'$ lie upon the same straight line and likewise,
from a consideration of the triangles $FGH$ and
$F'G'H'$, we find the points $O$, $H$, $H'$ to be situated
on a straight line. Now, in the triangles $FHJ$ and
$F'H'J'$, the straight lines joining the homologous
vertices all pass through the same point $O$, and,
hence, as a consequence of the second part of Desargues's
theorem, the straight lines $FJ$ and $F'J'$ must
also be parallel to each other. Finally, a consideration
of the triangles $DFJ$ and $D'F'J'$ shows that the
straight lines $DJ$ and $D'J'$ are parallel to each other
and with this our proof is completed.

\rfa{\S\!~27.\ {\small EQUATION OF THE STRAIGHT LINE, BASED UPON THE NEW ALGEBRA OF SEGMENTS.}}\label{p27}

In \S\S~24--26, we have introduced into the plane
geometry an algebra of segments in which the commutative
law of addition and that of multiplication,
as well as the two distributive laws, hold. This was
%%-----File: 097.png-----%%
done upon the assumption that the axioms cited in
\S~24, as also the theorem of Desargues, were valid.
In this section, we shall show how an analytical representation
of the point and straight line in the plane
is possible upon the basis of this algebra of segments.

{\scshape Definition.} Take the two given fixed straight
lines lying in the plane and intersecting in $O$ as the
axis of $x$ and of $y$, respectively. Let us suppose any
point $P$ of the plane determined by the two segments
$x$, $y$ which we obtain upon the $x$-axis and $y$-axis, respectively,
by drawing through $P$ parallels to these
axes. These segments are called the \emph{co-ordinates} of
the point $P$. Upon the basis of this new algebra of
segments and by aid of Desargues's theorem, we shall
deduce the following proposition.

\begin{itemize}
\item[]{\scshape Theorem 34.} The co-ordinates $x$, $y$ of a point on
an arbitrary straight line always satisfy an equation
in these segments of the form
\[
ax+by+c = 0.
\]
In this equation, the segments $a$ and $b$ stand
necessarily to the left of the co-ordinates $x$ and
$y$. The segments $a$ and $b$ are never both zero
and $c$ is an arbitrary segment.

Conversely, every equation in these segments
and of this form represents always a straight
line in the plane geometry under consideration.
\end{itemize}

{\scshape Proof.} Suppose that the straight line $l$ passes
through the origin $O$. Furthermore, let $C$ be a definite
point upon $l$ different from $O$, and $P$ any arbitrary
point of $l$. Let $OA$ and $OB$ be the co-ordinates of $C$
and $x$, $y$ be the co-ordinates of $P$. We will denote
the straight line joining the extremities of the segments
$x$, $y$ by $g$. Finally, through the extremity of
%%-----File: 098.png-----%%
the segment $1$, laid off on the $x$-axis, draw a straight
line $h$ parallel to $AB$. This parallel cuts off upon the
$y$-axis the segment $e$. From the second part of

\begin{figure}[htb]
\begin{center}
\includegraphics*[width=4in]{images/f048.png}\\
{\small Fig. 48.}
\end{center}
\end{figure}

Desargues's theorem, it follows that the straight line $g$
is also always parallel to $AB$. Since $g$ is always
parallel to $h$, it follows that the co-ordinates $x$, $y$ of the
point $P$ must satisfy the equation
$$
  ex=g.
$$

Moreover, in figure 49 let $l'$ be any arbitrary
straight line in our plane. This straight line will cut
off on the $x$-axis the segment $c=OO'$. Now, in the
same figure, draw through $O$ the straight line $l$ parallel
to $l'$. Let $P'$ be an arbitrary point on the line $l'$.
The straight line through $P'$, parallel to the $x$-axis,
intersects the straight line $l$ in $P$ and cuts off upon
the $y$-axis the segment $y = OB$. Finally, through $P$
and $P'$ let parallels to the $y$-axis cut off on the $x$-axis
the segments $x=OA$ and $x'= OA'$.

We shall now undertake to show that the equation
\[
  x'=x+c
\]
%%-----File: 099.png-----%%
is fulfilled by the segments in question. For this purpose,
draw $O'C$ parallel to the unit-line and likewise
$CD$ parallel to the $x$-axis and $AD$ parallel to the $y$-axis.

\begin{figure}[htb]
\begin{center}
\includegraphics*[width=4in]{images/f049.png}\\
{\small Fig. 49.}
\end{center}
\end{figure}

Then, to prove our proposition amounts to showing
that we must have necessarily
\[
A'D \mbox{ parallel to } O'C.
\]
Let $D'$ be the point of intersection of the straight
lines $CD$ and $A'P'$ and draw $O'C'$ parallel to the $y$-axis.

Since, in the triangles $OCP$ and $O'C'P'$, the straight
lines joining the homologous vertices are parallel, it
follows, by virtue of the second part of Desargues's
theorem, that we must have
\[
CP \mbox{ parallel to } C'P'.
\]
In a similar way, a consideration of the triangles $ACP$
and $A'C'P'$ shows that we must have
\[
AC \mbox{ parallel to } A'C'.
\]
Since, in the triangles $ACD$ and $C'A'O'$, the homologous
sides are parallel to each other, it follows that
the straight lines $AC'$, $CA'$ and $DO'$ intersect in a
common point. A consideration of the triangles $C'A'D$
and $ACO'$ then shows that $A'D$ and $CO'$ are parallel
to each other.
%%-----File: 100.png-----%%
From the two equations already obtained, viz.:

\begin{equation}
ex = y  \mbox{\ and\ } x' = x + c,
\end{equation}

follows at once the equation

\begin{equation}
ex' = y + ec.
\end{equation}

If we denote, finally, by $n$ the segment which added
to the segment 1 gives the segment 0, then, from this
last equation, we may easily deduce the following

\begin{equation}
ex' + ny + nec = 0,
\end{equation}

and this equation is of the form required by theorem
34.

We can now show that the second part of the theorem
is equally true; for, every linear equation

\begin{equation}
ax + by + c = 0
\end{equation}

may evidently be brought into the required form

\begin{equation}
ex + ny + nec = 0
\end{equation}

by a left-sided multiplication by a properly chosen
segment.

It must be expressly stated, however, that, by our
hypothesis, an equation of segments of the form
\begin{equation}
xa + yb + c = 0,
\end{equation}
where the segments $a$, $b$ stand to the right of the co-ordinates
$x$, $y$ does not, in general, represent a straight
line.

In Section 30, we shall make an important application
of theorem 34.

\rfa{\S\!~28.\ {\small THE TOTALITY OF SEGMENTS, REGARDED AS A COMPLEX NUMBER SYSTEM.}}\label{p28}

We see immediately that, for the new algebra of
segments established in Section 24, theorems 1--6 of Section 13 are
fulfilled.
%%-----File: 101.png-----%%
Moreover, by aid of Desargues's theorem, we have
already shown in Sections 25 and 26 that the laws 7--11 of
operation, as given in Section 13, are all valid in this algebra
of segments. With the single exception of the commutative
law of multiplication, therefore, all of the
theorems of connection hold.

Finally, in order to make possible an order of magnitude
of these segments, we make the following convention.
Let $A$ and $B$ be any two distinct points of
the straight line $OE$. Suppose then that the four
points $O$, $E$, $A$, $B$ stand, in conformity with axiom II,
4, in a certain sequence. If this sequence is one of
the following six possible ones, viz.:

\[ ABOE, AOBE, AOEB, OABE, OAEB, OEAB, \]

then we will call the segment $a = OA$ {\itshape smaller} than the
segment $b = OB$ and indicate the same by writing

\[ a < b. \]

On the other hand, if the sequence is one of the six
following ones, viz.:

\[ BAOE, BOAE, BOEA, OBAE, OBEA, OEBA, \]

then we will call the segment $a = OA$ greater than the
segment $b = OB$, and we write accordingly

\[ a > b. \]

This convention remains in force whenever $A$ or $B$
coincides with $O$ or $E$, only then the coinciding points
are to be regarded as a single point, and, consequently,
we have only to consider the order of three points.

Upon the basis of the axioms of group II, we can
easily show also that, in our algebra of segments, the
laws 13--16 of operation given in Section 13 are fulfilled.
Consequently, the totality of all the different segments
forms a complex number system for which the laws
%%-----File: 102.png-----%%
1--11, 13--16 of Section 13 hold; that is to say, all of the
usual laws of operation except the commutative law
of multiplication and the theorem of Archimedes. We
will call such a system, briefly, a \textit{desarguesian number system.}

\rfa{\S\!~29.\ {\small CONSTRUCTION OF A GEOMETRY OF SPACE BY AID OF A DESARGUESIAN NUMBER SYSTEM.}}\label{p29}

Suppose we have given a desarguesian number
system $D$. Such a system makes possible the construction
of a geometry of space in which axioms I,
II, III are all fulfilled.

In order to show this, let us consider any system
of three numbers $(x, y, z)$ of the desarguesian number
system $D$ as a point, and the ratio of four such numbers
$(u:v:w:r)$, of which the first three are not 0,
as a plane. However, the systems $(u:v:w:r)$ and
$(au:av:aw:ar)$, where $a$ is any number of $D$ different
from 0, represent the same plane. The existence of
the equation
\[
ux + vy + wz + r = 0
\]
expresses the condition that the point $(x, y, z)$ shall
lie in the plane $(u:v:w:r)$. Finally, we define a
straight line by the aid of a system of two planes
$(u':v':w':r')$ and $(u'':v'':w'':r'')$, where we impose the
condition that it is impossible to find in $D$ two numbers
$a'$, $a''$ different from zero, such that we have
simultaneously the relations
\[
a'u' = a''u'', a'v' = a''v'', a'w' = a''w'',
\]
A point $(x, y, z)$ is said to be situated upon this
straight line $[(u':v':w':r'), (u'':v'':w'':r'')]$, if it is
common to the two planes $(u':v':w':r')$ and $(u'':v'':w'':r'')$.
%%-----File: 103.png-----%%
Two straight lines which contain the same
points are not regarded as being distinct.

By application of the laws 1--11 of \S\! 13, which by
hypothesis hold for the numbers of $D$, we obtain without
difficulty the result that the geometry of space
which we have just constructed satisfies all of the axioms
of groups I and III\@.

In order that the axioms (II) of order may also be
valid, we adopt the following conventions. Let
\[
(x_1, y_1, z_1), (x_2, y_2, z_2), (x_3, y_3, z_3)
\]
be any three points of a straight line
\[
[(u':v':w':r'), (u'':v'':w'':r'')].
\]
Then, the point $(x_2, y_2, z_2)$ is said to lie between the
other two, if we have fulfilled at least one of the six
following double inequalities:

\begin{eqnarray}
&x_1 < x_2 < x_3, &x_1 > x_2 > x_3,
 \\% \tag{1}
&y_1 < y_2 < y_3, &y_1 > y_2 > y_3,
 \\% \tag{2}
&z_1 < z_2 < z_3, &z_1 > z_2 > z_3,% \tag{3}
\end{eqnarray}

If one of the two double inequalities (1) exists, then
we can easily conclude that either $y_1 = y_2 = y_3$ or one
of the two double inequalities (2) exists, and, consequently,
either $z_1 = z_2 = z_3$ or one of the double inequalities
(3) must exist. In fact, from the equations
\begin{align*}
u'x_i +  v'y_i +  w'z_i +  r' & =  0,\\
u''x_i +  v''y_i +  w''z_i +  r'' & =  0,\\
(i = 1,2,3)
\end{align*}
we may obtain, by a left-sided multiplication of these
equations by numbers suitably chosen from $D$ and
then adding the resulting equations, a system of equations
of the form
\begin{eqnarray}
u'''x_i + v'''y_i + r''' = 0, (i = 1, 2, 3).
\end{eqnarray}
%%-----File: 104.png-----%%
In this system, the coefficient $v'''$ is certainly different
from zero, since otherwise the three numbers $x_1$, $x_2$, $x_3$
would be mutually equal.

From
\[
x_1 \lessgtr x_2 \lessgtr x_3,
\]
it follows that
\[
u'''x_1 \lesseqqgtr x'''u_2 \lesseqqgtr x'''u_3,
\]
and, hence, as a consequence of (4), we have
\[
v'''y_1 + r''' \lesseqqgtr v'''y_2 + r''' \lesseqqgtr
v'''y_3 + r'''
\]
and, therefore,
\[
v'''y_1 \lesseqqgtr v'''y_2 \lesseqqgtr v'''y_3.
\]
Since $v'''$ is different from zero, we have
\[
y_1 \lesseqqgtr y_2 \lesseqqgtr y_3.
\]
In each of these double inequalities, we must take
either the upper sign throughout, or the middle sign
throughout, or the lower sign throughout.

The preceding considerations show, that, in our
geometry, the linear axioms II, 1--4 of order are all
valid. However, it remains yet to show that, in this
geometry, the plane axiom II, 5 is also valid.

For this purpose let a plane $(u:v:w:r)$ and a
straight line $[(u:v:w:r), (u':v':w':r')]$ in this plane
be given. Let us assume that all the points $(x, y, z)$
of the plane $(u:v:w:r)$, for which we have the expression
$u'x + v'y + w'z + r'$ greater than or less than
zero, lie respectively upon the one side or upon the
other side of the given straight line. We have then
only to show that this convention is in accordance
with the preceding statements. This, however, is
easily done.
%%-----File: 105.png-----%%
We have thus shown that all of the axioms of
groups I, II, III are fulfilled in the geometry of space
which we have obtained in the above indicated manner
from the desarguesian number system \textit{D}. Remembering
now that the theorem of Desargues is a
consequence of the axioms I, II, III, we see that the
proposition just stated is exactly the converse of the
result reached in Section 28.

\rfa{\S\!~30.\ {\small SIGNIFICANCE OF DESARGUES'S THEOREM.}}\label{p30}

If, in a plane geometry, axioms I, 1--2, II, III are
all fulfilled and, moreover, if the theorem of Desargues
holds, then, according to \S\S~24--28, it is always
possible to introduce into this geometry an algebra of
segments to which the laws 1--11, 13--16 of \S\! 13 are
applicable. We will now consider the totality of these
segments as a complex number system and construct,
upon the basis of this system, a geometry of space, in
accordance with \S\! 29, in which all of the axioms I, II,
III hold.

In this geometry of space, we shall consider only
the points $(x, y, 0)$ and those straight lines upon
which only such points lie. We have then a plane geometry
which must, if we take into account the proposition
established in \S\! 27, coincide exactly with the
plane geometry proposed at the beginning. Hence,
we are led to the following proposition, which may be
regarded as the objective point of the entire discussion
of the present chapter.

\begin{itemize}
\item[]{\scshape Theorem 35.} If, in a plane geometry, axioms I,
1--2, II, III are all fulfilled, then the existence of
Desargues's theorem is the necessary and sufficient
condition that this plane geometry may
%%-----File: 106.png-----%%
be regarded as a part of a geometry of space in
which all of the axioms I, II, III are fulfilled.
\end{itemize}

The theorem of Desargues may be characterized
for plane geometry as being, so to speak, the result
of the elimination of the space axioms.

The results obtained so far put us now in the position
to show that every geometry of space in which
axioms I, II, III are all fulfilled may be always regarded
as a part of a ``geometry of any number of dimensions
whatever.'' By a geometry of an arbitrary
number of dimensions is to be understood the totality
of all points, straight lines, planes, and other linear
elements, for which the corresponding axioms of connection
and of order, as well as the axiom of parallels,
are all valid.
\newpage
%%-----File: 107.png-----%%
\begin{center}{\Large PASCAL's THEOREM.}\end{center}
\rfa{\S\!~31.\ {\small TWO THEOREMS CONCERNING THE POSSIBILITY OF PROVING PASCAL'S THEOREM.}}\label{p31}

As is well known, Desargues's theorem (theorem 32)
may be demonstrated by the aid of axioms I, II,
III; that is to say, by the use, essentially, of the axioms
of space. In \S\! 23, we have shown that the demonstration
of this theorem without the aid of the space
axioms of group I and without the axioms of congruence
(group IV) is impossible, even if we make use
of the axiom of Archimedes.

Upon the basis of axioms I, 1--2, II, III, IV and,
hence, by the exclusion of the axioms of space but
with the assistance, essentially, of the axioms of congruence,
we have, in \S\! 14, deduced Pascal's theorem
and, consequently, according to \S\! 22, also Desargues's
theorem. The question arises as to whether Pascal's
theorem can be demonstrated without the assistance
of the axioms of congruence. Our investigation will
show that in this respect Pascal's theorem is very different
from Desargues's theorem; for, in the demonstration
of Pascal's theorem, the admission or exclusion
of the axiom of Archimedes is of decided influence.
We may combine the essential results of our investigation
in the two following theorems.

\begin{itemize}
\item[]{\scshape Theorem 36.} Pascal's theorem (theorem 21) may
be demonstrated by means of the axioms I, II,
%%-----File: 108.png-----%%
\label{folio101}
III, V; that is to say, without the assistance
of the axioms of congruence and with the aid
of the axiom of Archimedes.
\end{itemize}

\begin{itemize}
\item[]{\scshape Theorem 37.} Pascal's theorem (theorem 21) cannot
be demonstrated by means of the axioms I,
II, III alone; that is to say, by exclusion of
the axioms of congruence and also the axiom
of Archimedes.
\end{itemize}

In the statement of these two theorems, we may,
by virtue of the general theorem 35, replace the space
axioms I, 3--7 by the plane condition that Desargues's
theorem (theorem 32) shall be valid.

\rfa{\S\!~32.\ {\small THE COMMUTATIVE LAW OF MULTIPLICATION FOR AN ARCHIMEDEAN NUMBER SYSTEM.}}\label{p32}

The demonstration of theorems 36 and 37 rests
essentially upon certain mutual relations concerning
the laws of operation and the fundamental propositions
of arithmetic, a knowledge of which is of itself
of interest. We will state the two following theorems.

\begin{itemize}
\item[]{\scshape Theorem 38.} For an archimedean number system,
the commutative law of multiplication is a
necessary consequence of the remaining laws of
operation; that is to say, if a number system
possesses the properties 1--11, 13--17 given in
\S\! 13, it follows necessarily that this system satisfies
also formula 12.
\end{itemize}

{\scshape Proof}. Let us observe first of all that, if $a$ is an
arbitrary number of the system, and, if
\[
n = 1+1+\cdots+1
\]
is a positive integral rational number, then for $n$ and
$a$ the commutative law of multiplication always holds.
In fact, we have
%%-----File: 109.png-----%%
\begin{align*}
 an &= a(1+ 1 + \cdots +1) \\
    &= a \cdot 1 + a\cdot 1 +\cdots +a\cdot 1\\
    &= a + a +\cdots +a
\end{align*}
and likewise
\begin{align*}
 na &= (1+ 1 + \cdots +1)a \\
    &= 1 \cdot a + 1\cdot a + \cdots + 1\cdot a\\
    &= a + a +\cdots +a.
\end{align*}
Suppose now, in contradiction to our hypothesis,
$a, b$ to be numbers of this system, for which the commutative
law of multiplication does not hold. It is
then at once evident that we may make the assumption
that we have
\[
 a> 0 \text{, }  b > 0 \text{, }  ab-ba > 0.
\]
By virtue of condition 6 of \S\! 13, there exists a number
$c(> 0)$, such that
\[
 (a+b+1) c = ab-ba.
\]
Finally, if we select a number $d$, satisfying simultaneously
the inequalities
\[
 d> 0 \text{, } d < 1 \text{, }  d<c,
\]
and denote by $m$ and $n$ two such integral rational
numbers $\ge 0$ that we have respectively
\[
 md < a \le (m+1) d
\]
and
\[
 nd < b \le (n+1) d
\]
then the existence of the numbers $m$ and $n$ is an immediate
consequence of the theorem of Archimedes
(theorem 17, \S\! 13). Recalling now the remark made
at the beginning of this proof, we have by the multiplication
of the last inequalities
\begin{align*}
 ab &\leqq mnd^2 + (m+n+1)d^2\\
 ba &> mnd^2,
\end{align*}
%%-----File: 110.png-----%%
and, hence, by subtraction
\[
ab-ba \leqq (m+n+1)d^2.
\]
We have, however,
\[
md<a \text{, } nd<b, \text{, } d<1
\]
and, consequently,
\[
(m+n+1)d<a+b+1;
\]
i.e.,
\[
ab-ba<(a+b+1)d,
\]
or, since $d<c$, we have
\[
ab-ba<(a+b+1)c.
\]
This inequality stands in contradiction to the definition
of the number $c$, and, hence, the validity of the
theorem 38 follows.

\rfa{\S\!~33.\ {\small THE COMMUTATIVE LAW OF MULTIPLICATION FOR A NON-ARCHIMEDEAN NUMBER SYSTEM.}}\label{p33}

\begin{itemize}
\item[]{\scshape Theorem 39.} For a non-archimedean number
system, the commutative law of multiplication
is not a necessary consequence of the remaining
laws of operation; that is to say, there exists
a system of numbers possessing the properties
1--11, 13--16 mentioned in \S\! 13, but for
which the commutative law (12) of multiplication
is not valid. A desarguesian number system,
in the sense of \S\! 28, is such a system.
\end{itemize}

{\scshape Proof}. Let $t$ be a parameter and $T$ any expression
containing a finite or infinite number of terms,
say of the form
\[
T=r_0t^n + r_1t^{n+3} + r_2t^{n+2} + r_3t^{n+3} + \ldots,
\]
where $r_0(\neq 0), r_1, r_2 \ldots$ are arbitrary rational numbers
and $n$ is an arbitrary integral rational number
%%-----File: 111.png-----%%
$\gtreqqless 0$. Moreover, let $s$ be another parameter and $S$ any
expression having a finite or infinite number of terms,
say of the form
\begin{equation*}
 S = s^mT_0 + s^{m+1}T_1 + s^{m+2}T_2 + \ldots ,
\end{equation*}
where $T_0 (\ne 0), T_1, T_2, \ldots $ denote arbitrary expressions
of the form $T$ and $m$ is again an arbitrary integral
rational number $\lesseqqgtr 0$. We will regard the totality of
all the expressions of the form $S$ as a complex number
system $\Omega(s,\!t)$, for which we will assume the following
laws of operation; namely, we will operate
with $s$ and $t$ according to the laws 7--11 of \S\! 13, as
with parameters, while in place of rule 12 we will apply
the formula
\begin{equation}
ts = 2st.
\end{equation}
If, now, $S'$, $S''$ are any two expressions of the form
$S$, say
\begin{align*}
 S' &= s^{m'}T'_0 + s^{m'+1}T'_1 + s^{m'+2}T'_2 + \ldots,\\
 S'' &= s^{m''}T''_0 + s^{m''+1}T''_1 + s^{m''+2}T''_2 + \ldots,
\end{align*}
then, by combination, we can evidently form a new expression
$S'+S''$ which is of the form $S$, and is, moreover,
uniquely determined. This expression $S' + S''$
is called the sum of the numbers represented by $S'$
and $S''$.

By the multiplication of the two expressions $S'$ and
$S''$ term by term, we obtain another expression of the
form
\begin{equation*}
\begin{split}
 &S'S'' = s^{m'}T'_0 s^{m''} T''_0
 + (s^{m'} T'_0 s^{m''+1} T''_1
 - s^{m'+1} T'_1 s^{m''} T_0'')\\
 &+ (s^{m'} T'_0 s^{m''+2} T''_2
 + s^{m'+1} T'_1 s^{m''+1}T''_1
 + s^{m'+2} T'_2 s^{m''} T''_0) + \ldots
\end{split}
\end{equation*}
This expression, by the aid of formula (1), is evidently
a definite single-valued expression of the form $S$ and
%%-----File: 112.png-----%%
we will call it the product of the numbers represented
by $S'$ and $S''$.

This method of calculation shows at once the validity
of the laws 1--5 given in Section 13 for calculating with
numbers. The validity of law 6 of that section is also
not difficult to establish. To this end, let us assume
that
\[
S'=s^{m'}T'_0+s^{m'+1}T'_1+s^{m'+2}T'_2+ \cdots
\]
and
\[
S'''=s^{m'''}T'''_0+s^{m'''+1}T'''_1+s^{m'''+2}T'''_2 \cdots
\]
are two expressions of the form $S$, and let us suppose,
further, that the coefficient $r'_0$ of $T'_0$ is different from
zero. By equating the like powers of $s$ in the two
members of the equation
\[
S'S''=S''',
\]
we find, first of all, in a definite manner an integral
number $m''$ as exponent, and then such a succession
of expressions
\[
T''_0 \text{, } T''_1 \text{, } T''_2 \ldots
\]
that, by aid of formula (1), the expression
\[
S''=s^{m''}T''_0+s^{m''+1}T''_1+s^{m''+2}T''_2 \dots
\]
satisfies equation (2). With this our theorem is established.

In order, finally, to render possible an order of sequence
of the numbers of our system $\Omega(s, t)$, we make
the following conventions. Let a number of this system
be called greater or less than 0 according as in
the expression $S$, which represents it, the first coefficient
$r_0$ of $T_0$ is greater or less than zero. Given any
two numbers $a$, $b$ of the complex number system under
consideration, we say that $a<b$ or $a>b$ according as
%%-----File: 113.png-----%%
\label{folio106}
we have $a-b < 0$ or $> 0$. It is seen immediately that,
with these conventions, the laws 13--16 of \S\! 13 are
valid; that is to say, $\Omega (s, t)$ is a desarguesian number
system (see \S\! 28).

As equation (1) shows, law 12 of \S\! 13 is not
fulfilled by our complex number system and,
consequently, the validity of theorem 39 is fully established.

In conformity with theorem 38, Archimedes's theorem
(theorem 17, \S\! 13) does not hold for the number
system $\Omega (s, t)$ which we have just constructed.

We wish also to call attention to the fact that the
number system $\Omega (s, t)$, as well as the systems $\Omega$ and
$\Omega (t)$ made use of in \S\! 9 and \S\! 12, respectively, contains
only an enumerable set of numbers.

\rfa{\S\!~34.\ {\small PROOF OF THE TWO PROPOSITIONS CONCERNING PASCAL'S THEOREM. (NON-PASCALIAN GEOMETRY.)}}\label{p34}

If, in a geometry of space, all of the axioms I, II,
III are fulfilled, then Desargues's theorem (theorem
32) is also valid, and, consequently, according to
\S\S~24--26, pp. \pageref{folio79}--\pageref{folio89}, it is possible to introduce into this
geometry an algebra of segments for which the rules
1--11, 13--16 of \S\! 13 are all valid. If we assume now
that the axiom (V) of Archimedes is valid for our
geometry, then evidently Archimedes's theorem (theorem
17 of \S\! 13) also holds for our algebra of segments,
and, consequently, by virtue of theorem 38,
the commutative law of multiplication is valid. Since,
however, the definition of the product of two segments,
as introduced in \S\! 24 (figure 42) and which is
the definition here also under discussion, agrees with
the definition in \S\! 15 (figure 22), it follows from the
construction made in \S\! 15 that the commutative law
%%-----File: 114.png-----%%
of multiplication is here nothing else than Pascal's
theorem. Consequently, the validity of theorem 36 is
established.

In order to demonstrate theorem 37, let us consider
again the desarguesian number system $\Omega(s, t)$
introduced in $\S$ 33, and construct, in the manner described
in $\S$ 29, a geometry of space for which all of
the axioms I, II, III are fulfilled. However, Pascal's
theorem will not hold for this geometry; for, the commutative
law of multiplication is not valid in the desarguesian
number system $\Omega(s, t)$. According to theorem 36,
the non-pascalian geometry is then necessarily
also a non-archimedean geometry.

By adopting the hypothesis we have, it is evident
that we cannot demonstrate Pascal's theorem, providing
we regard our geometry of space as a part of
a geometry of an arbitrary number of dimensions in
which, besides the points, straight lines, and planes,
still other linear elements are present, and providing
there exists for these elements a corresponding system
of axioms of connection and of order, as well as
the axiom of parallels.

\rfa{\S\!~35.\ {\small THE DEMONSTRATION, BY MEANS OF THE
THEOREMS OF PASCAL AND DESARGUES, OF ANY THEOREM RELATING TO POINTS
OF INTERSECTION.}}\label{p35}

Every proposition relating to points of intersection
in a plane has necessarily The form: Select,
first of all, an arbitrary system of points and straight
lines satisfying respectively the condition that certain
ones of these points are situated on certain ones of
the straight lines. If, in some known manner, we
construct the straight lines joining the given points
%%-----File: 115.png-----%%
and determine the points of intersection of the given
lines, we shall obtain finally a definite system of three
straight lines, of which our proposition asserts that
they all pass through the same point.

Suppose we now have a plane geometry in which
all of the axioms I 1--2, II \ldots, V are valid. According
to $\S$ 17, pp. \pageref{folio53}--\pageref{folio56}, we may now find, by making
use of a rectangular pair of axes, for each point a corresponding
pair of numbers $(x,y)$ and for each straight
line a ratio of three definite numbers $(u:v:w)$. Here,
the numbers $x,\ y,\ u,\ v,\ w$ are all \emph{real} numbers, of
which $u,\ v$ cannot both be zero. The condition showing
that the given point is situated upon the given
straight line, viz.:
\begin{equation}ux+vy+w=0\end{equation}
is an equation in the ordinary sense of the word. Conversely,
in case $x,\ y,\ u,\ v,\ w$ are numbers of the algebraic
domain $\Omega$ of $\S$ 9, and $u,\ v$ are not both zero, we
may certainly assume that each pair of numbers $(x,y)$
gives a point and that each ratio of three numbers
$(u:v:w)$ gives a straight line in the geometry in question.

If, for all the points and straight lines which occur
in connection with any theorem relating to intersections
in a plane, we introduce the corresponding pairs
and triples of numbers, then such a theorem asserts
that a definite expression $A(p_1,p_2,\ldots,p_r)$ with real
coefficients and depending rationally upon certain
parameters $p_1,p_2,\ldots,p_r$ always vanishes as soon as
we put for each of these parameters a number of the
main $\Omega$ considered in $\S$ 9. We conclude from this
that the expression $A(p_1,p_2,\dots,p_r)$ must also vanish
identically in accordance with the laws 7--12 of
$\S$ 13.
%%-----File: 116.png-----%%

Since, according to $\S$ 32, Desargues's theorem holds
for the geometry in question, it follows that we certainly
can make use of the algebra of segments introduced
in $\S$ 24, and because Pascal's theorem is equally
valid in this case, the commutative law of multiplication
is also. Hence, for this algebra of segments, all
of the laws 7--12 of $\S$ 13 are valid.

If we take as our axes in this new algebra of segments
the co-ordinate axes already used and consider
the unit points $E$, $E'$ as suitably established, we see
that the new algebra of segments is nothing else than
the system of co-ordinates previously employed.

In order to show that, for the new algebra of segments,
the expression $A(p_1,\ p_2,\ \ldots,\ p_r)$ vanishes
identically, it is sufficient to apply the theorems of
Pascal and Desargues. Consequently we see that:

\textit{Every proposition relative to points of intersection in
the geometry in question must always, by the aid of suitably
constructed auxiliary points and straight lines, turn
out to be a combination of the theorems of Pascal and
Desargues. Hence for the proof of the validity of a theorem
relating to points of intersection, we need not have
resource to the theorems of congruence.}
\newpage
%%-----File: 117.png-----%%
\begin{center}{\Large GEOMETRICAL CONSTRUCTIONS BASED UPON THE AXIOMS I--V.}\end{center}
\rfa{\S\!~36.\ {\small GEOMETRICAL CONSTRUCTIONS BY MEANS OF A STRAIGHT-EDGE AND A TRANSFERER OF SEGMENTS.}}\label{p36}

Suppose we have given a geometry of space, in
which all of the axioms I--V are valid. For the
sake of simplicity, we shall consider in this chapter a
a plane geometry which is contained in this geometry
of space and shall investigate the question as to what
elementary geometrical constructions may be carried
out in such a geometry.

Upon the basis of the axioms of group I, the following
constructions are always possible.

{\scshape Problem} 1. To join two points with a straight
line and to find the intersection of two straight lines,
the lines not being parallel.

Axiom III renders possible the following construction:

{\scshape Problem} 2. Through a given point to draw a parallel
to a given straight line.

By the assistance of the axioms (IV) of congruence,
it is possible to lay off segments and angles;
that is to say, in the given geometry we may solve the
following problems:

{\scshape Problem} 3. To lay off from a given point upon a
given straight line a given segment.
%%-----File: 118.png-----%%

{\scshape Problem} 4. To lay off on a given straight line a
given angle; or what is the same thing, to construct
a straight line which shall cut a given straight line at
a given angle.

It is impossible to make any new constructions by
the addition of the axioms of groups II and V. Consequently,
when we take into consideration merely the
axioms of groups I--V, all of those constructions and
only those are possible, which may be reduced to the
problems 1--4 given above.

We will add to the fundamental problems 1--4 also
the following:

{\scshape Problem} 5. To draw a perpendicular to a given
straight line.

We see at once that this construction can be made
in different ways by means of the problems 1--4.

In order to carry out the construction in problem
1, we need to make use of only a \textit{straight edge}. An
instrument which enables us to make the construction
in problem 3, we will call a \textit{transferer of segments}. We
shall now show that problems 2, 4, and 5 can be reduced
to the constructions in problems 1 and 3 and,
consequently, all of the problems 1--5 can be completely
constructed by means of a straight-edge and a
transferer of segments. We arrive, then, at the following
result:


\begin{itemize}
\item[]{\scshape Theorem 40.} Those problems in geometrical
construction, which may be solved by the assistance
of only the axioms I--V, can always be
carried out by the use of the straight-edge and
the transferer of segments.
\end{itemize}

{\scshape Proof.} In order to reduce problem 2 to the solution
of problems 1 and 3, we join the given point $P$
%%-----File: 119.png-----%%
with any point $A$ of the given straight line and produce
$PA$ to $C$, making $AC=PA$. Then, join $C$ with
any other point $B$ of the given straight line and
produce $CB$ to $Q$, making, $BQ=CB$. The straight
line $PQ$ is the desired parallel.

\begin{figure}[htb]
\begin{center}
\includegraphics*[width=2in]{images/f050.png}\\
{\small Fig. 50.}
\end{center}
\end{figure}

We can solve problem 5 in the following manner.
Let $A$ be an arbitrary point of the given straight line.
Then upon this straight line, lay off in both directions
from $A$ the two equal segments $AB$ and $AC$. Determine,
upon any two straight lines passing through the
point $A$, the points $E$ and $D$ so that the segments
$AD$ and $AE$ will equal $AB$ and $AC$.

\begin{figure}[htb]
\begin{center}
\includegraphics*[width=4in]{images/f051.png}\\
{\small Fig. 51.}
\end{center}
\end{figure}

Suppose the
straight lines $BD$ and $CE$ intersect in $F$ and the
straight lines $BE$ and $CD$ intersect in $H$. $FH$ is
then the desired perpendicular. In fact, the angles
$BDC$ and $BEC$, being inscribed in a semicircle having
%%-----File: 120.png-----%%
the diameter $BC$, are both right angles, and, hence,
according to the theorem relating to the point of intersection
of the altitudes of a triangle, the straight
lines $FH$ and $BC$ are perpendicular to each other.

\begin{figure}[htb]
\begin{center}
\includegraphics*[width=3in]{images/f052.png}\\
{\small Fig. 52.}
\end{center}
\end{figure}

Moreover, we can easily solve problem 4 simply by
the drawing of straight lines and the laying off of segments.
We will employ the following method which
requires only the drawing of parallel lines and the
erection of perpendiculars. Let $\beta$ be the angle to be
laid off and $A$ its vertex. Draw through $A$ a straight
line $l$ parallel to the given straight line, upon which
we are to lay off the given angle $\beta$. From an arbitrary
point $B$ of one side of the angle $\beta$, let fall a perpendicular
upon the other side of this angle and also
one upon $l$. Denote the feet of these perpendiculars
by $D$ and $C$ respectively. The construction of these
perpendiculars is accomplished by means of problems
2 and 5. Then, let fall from $A$ a perpendicular upon
$CD$, and let its foot be denoted by $E$. According to
the demonstration given in Section \S~14, the angle $CAE$ equals
$\beta$. Consequently, the construction in 4 is made to
depend upon that of 1 and 3 and with this our proposition
is demonstrated.
%%-----File: 121.png-----%%
\rfa{\S\!~37.\ {\small ANALYTICAL REPRESENTATION OF THE CO-ORDINATES OF POINTS WHICH CAN BE SO CONSTRUCTED.}}\label{p37}

Besides the elementary geometrical problems considered
in \S~36, there exists a long series of other
problems whose solution is possible by the drawing
of straight lines and the laying off of segments. In
order to get a general survey of the scope of the problems
which may be solved in this manner, let us take
as the basis of our consideration a system of axes in
rectangular co-ordinates and suppose that the co-ordinates
of the points are, as usual, represented by real
numbers or by functions of certain arbitrary parameters.
In order to answer the question in respect to
all the points capable of such a construction, we employ
the following considerations.

Let a system of definite points be given. Combine
the co-ordinates of these points into a domain $R$.
This domain contains, then, certain real numbers and
certain arbitrary parameters $p$. Consider, now, the
totality of points capable of construction by the drawing
of straight lines and the laying off of definite segments,
making use of the system of points in question.
We will call the domain formed from the co-ordinates
of these points $\Omega(R)$, which will then contain real
numbers and functions of the arbitrary parameters $p$.

The discussion in \S~17 shows that the drawing of
straight lines and of parallels amounts, analytically,
to the addition, subtraction, multiplication, and division
of segments. Furthermore, the well known formula
given in \S~9 for a rotation shows that the laying
off of segments upon a straight line does not necessitate
%%-----File: 122.png-----%%
any other analytical operation than the extraction
of the square root of the sum of the squares of two
segments whose bases have been previously constructed.
Conversely, in consequence of the pythagorean
theorem, we can always construct, by the aid
of a right triangle, the square root of the sum of the
squares of two segments by the mere laying off of
segments.

From these considerations, it follows that the domain
$\Omega(R)$ contains all of those and only those real
numbers and functions of the parameters $p$, which arise
from the numbers and parameters in $R$ by means of a
finite number of applications of the five operations;
viz., the four elementary operations of arithmetic and,
in addition, the fifth operation of extracting the square
root of the sum of two squares. We may express this
result as follows:

\begin{itemize}
\item[]{\scshape Theorem 41.} A problem in geometrical construction
is, then, possible of solution by the drawing
of straight lines and the laying off of segments,
that is to say, by the use of the straight-edge
and a transferer of segments, when and only
when, by the analytical solution of the problem,
the co-ordinates of the desired points are
such functions of the co-ordinates of the given
points as may be determined by the rational
operations and, in addition, the extraction of
the square root of the sum of two squares.
\end{itemize}

From this proposition, we can at once show that
not every problem which can be solved by the use of
a compass can also be solved by the aid of a transferer
of segments and a straight-edge. For the purpose
of showing this, let us consider again that geometry
%%-----File: 123.png-----%%
which was constructed in \S~9 by the help of the
domain $\Omega$ of algebraic numbers. In this geometry,
there exist only such segments as can be constructed
by means of a straight-edge and a transferer of segments,
namely, the segments determined by the numbers
of the domain $\Omega$.

Now, if $\omega$ is a number of the domain $\Omega$, we easily
see from the definition of $\Omega$ that every algebraic number
conjugate to $\omega$ must also lie in $\Omega$. Since the numbers
of the domain $\Omega$ are evidently all real, it follows
that it can contain only such real algebraic numbers
as have their conjugates also real.

Let us now consider the following problem; viz.,
to construct a right triangle having the hypotenuse
1 and one side $|\sqrt{2}| - 1$. The algebraic number
$\sqrt{2|\sqrt{2}| - 2}$, which expresses the numerical value of
the other side, does not occur in the domain $\Omega$, since
the conjugate number $\sqrt{-2|\sqrt{2}| - 2}$ is imaginary.
This problem is, therefore, not capable of solution in
the geometry in question and, hence, cannot be constructed
by means of a straight-edge and a transferer
of segments, although the solution by means of a compass
is possible.

\rfa{\S\!~38.\ {\small THE REPRESENTATION OF ALGEBRAIC NUMBERS AND OF INTEGRAL RATIONAL FUNCTIONS AS SUMS OF SQUARES.}}\label{p38}

The question of the possibility of geometrical constructions
by the aid of a straight-edge and a transferer
of segments necessitates, for its complete treatment,
particular theorems of an arithmetical and algebraic
character, which, it appears to me, are themselves of
interest.
%%-----File: 124.png-----%%
Since the time of Fermat, it has been known that
every positive integral rational number can be represented
as the sum of four squares. This theorem of
Fermat permits the following remarkable generalization:

{\scshape Definition}. Let $k$ be an arbitrary number field
and let $m$ be its degree. We will denote by $k^{\prime}$, $k^{\prime\prime}$,
\ldots, $k^{(m-1)}$ the $m-1$ number fields conjugate to $k$.
If, among the $m$ fields $k$, $k^{\prime}, k^{\prime\prime}, \ldots, k^{(m-1)}$ there is
one or more formed entirely of real numbers, then we
call these fields real. Suppose that the fields $k$, $k^\prime$,
\ldots, $k^{(s-1)}$ are such. A number $\alpha$ of the field $k$ is called
in this case {\em totally positive in $k$}, whenever the $s$ numbers
conjugate to $\alpha$, contained respectively in $k$, $k^\prime$,
$k^{\prime\prime}$, \ldots, $k^{(s-1)}$ are all positive. However, if in each
of the $m$ fields $k$, $k^\prime$, $k^{\prime\prime}$, \ldots, $k^{(m-1)}$ there are also imaginary
numbers present, we call every number $\alpha$ in
$k$ {\em totally positive}.

We have, then, the following proposition:
\begin{itemize}
\item[]{\scshape Theorem 42.} Every totally positive number in $k$
may be represented as the sum of four squares,
whose bases are integral or fractional numbers
of the field $k$.
\end{itemize}
The demonstration of this theorem presents serious
difficulty. It depends essentially upon the theory
of relatively quadratic number fields, which I have
recently developed in several papers.\footnote{``Ueber
die Theorie der relativquadratischen Zahlk\"orper,'' {\em Jahresbericht
der Deutschen Math. Vereinigung}, Vol. 6, 1899, and {\em Math. Annalen}, Vol.
51. See, also, ``Ueber die Theorie der relativ-Abelschen Zahlk\"orper'' {\em Nachr.\ der
K. Ges.\ der Wiss.\ zu G\"ottingen}, 1898.
}
 We will here
call attention only to that proposition in this theory
which gives the condition that a ternary diophantine
equation of the form
%%-----File: 125.png-----%%
\[
\alpha\xi^2 + \beta\eta^2 + \gamma\zeta^2 = 0
\]

can be solved when the coefficients $\alpha$, $\beta$, $\gamma$ are given
numbers in $k$ and $\xi$, $\eta$, $\zeta$ are the required numbers in
$k$. The demonstration of theorem 42 is accomplished
by the repeated application of the proposition just
mentioned.

From theorem 42 follow a series of propositions
concerning the representation of such rational functions
of a variable, with rational coefficients, as never
have negative values. I will mention only the following
theorem, which will be of service in the following
sections.

\begin{itemize}
\item[]{\scshape Theorem 43.} Let, $f(x)$ be an integral rational function
of $x$ whose coefficients are rational numbers
and which never becomes negative for any
real value of $x$. Then $f(x)$ can always be represented
as the quotient of two sums of squares
of which the bases are all integral rational functions
of $x$ with rational coefficients.
\end{itemize}

{\scshape Proof}. We will denote the degree of the function
$f(x)$ by $m$, which, in any case, must evidently be even.
When $m = 0$, that is to say, when $f(x)$ is a rational
number, the validity of theorem 43 follows immediately
from Fermat's theorem concerning the representation
of a positive number as the sum of four
squares. We will assume that the proposition is already
established for functions of degree $2$, $4$, $6$, \ldots,
$m-2$, and show, in the following manner, its validity
for the case of a function of the $m^{th}$ degree.

Let us, first of all, consider briefly the case where
$f(x)$ breaks up into the product of two or more integral
functions of $x$ with rational coefficients. Suppose
$p(x)$ to be one of those functions contained in
%%-----File: 126.png-----%%
$f(x)$, which itself cannot be further decomposed into
a product of integral functions having rational coefficients.
It then follows at once from the ``definite''
character which we have given to the function $f(x)$,
that the factor $p(x)$ must either appear in $f(x)$ to an
even degree or $p(x)$ must be itself ``definite''; that
is to say, must be a function which never has negative
values for any real values of $x$. In the first case,
the quotient $\frac{f(x)}{\{p(x)\}^2}$ and, in the second case, both $p(x)$
and $\frac{f(x)}{p(x)}$, are ``definite,'' and these functions have an
even degree $< m$. Hence, according to our hypothesis,
in the first case, $\frac{f(x)}{\{p(x)\}^2}$ and, in the last case, $p(x)$
and $\frac{f(x)}{p(x)}$ may be represented as the quotient of the
sum of squares of the character mentioned in theorem
43. Consequently, in both of these cases, the function
$f(x)$ admits of the required representation.

Let us now consider the case where $f(x)$ cannot
be broken up into the product of two integral functions
having rational coefficients. The equation $f(\theta)=0$ defines,
then, a field of algebraic numbers $k(\theta)$ of the
$m^\text{th}$ degree, which, together with all their conjugate
fields, are imaginary. Since, according to the definition
given just before the statement of theorem 42,
each number given in $k(\theta)$, and hence also $-1$ is totally
positive in $k(\theta)$, it follows from theorem 42 that
the number $-1$ can be represented as a sum of the
squares of four definite numbers in $k(\theta)$. Let, for example
\[
\tag{1}
-1 = \alpha^2 + \beta^2 + \gamma^2 + \delta^2,
\]
where $\alpha$, $\beta$, $\gamma$, $\delta$ are integral or fractional numbers in
$k(\theta)$. Let us put
%%-----File: 127.png-----%%
\begin{equation*}
\begin{array}{rcccl}
\alpha & = & a_1 \theta^{m-1} + a_2 \theta^{m-2} + \cdots + a_m  & = & \phi(\theta), \\
\beta  & = & b_1 \theta^{m-1} + b_2 \theta^{m-2} + \cdots + b_m  & = & \psi(\theta), \\
\gamma & = & c_1 \theta^{m-1} + c_2 \theta^{m-2} + \cdots + c_m  & = & \chi(\theta), \\
\theta & = & d_1 \theta^{m-1} + d_2 \theta^{m-2} + \cdots + d_m  & = & \rho(\theta);
\end{array}
\end{equation*}

where $a_1$, $a_2$, \ldots, $a_m$, \ldots, $d_1$, $d_2$, \ldots, $d_m$ are the
rational numerical coefficients and $\phi(\theta)$, $\psi(\theta)$, $\chi(\theta)$,
$\rho(\theta)$ the integral rational functions in question, having
the degree $(m-1)$ in $\theta$.

From (1), we have
\[
1 + \{ \phi(\theta) \}^2 + \{\psi(\theta)\}^2 + \{\chi(\theta)\}^2 + \{\rho(\theta)\}^2 = 0
\]
Because of the irreducibility of the equation $f(x) = 0$,
the expression
\[
F(x) = 1 + \{ \phi(\theta) \}^2 + \{\psi(\theta)\}^2 + \{\chi(\theta)\}^2 + \{\rho(\theta)\}^2
\]
represents, necessarily, an integral rational function
of $x$ which is divisible by $f(x)$. $F(x)$ is, then, a
``definite'' function of the degree $(2m-2)$ or lower.
Hence, the quotient $\frac{F(x)}{f(x)}$ is a ``definite'' function of
the degree $(m-2)$ or lower in $x$, having rational coefficients.
Consequently, by the hypothesis we have
made, $\frac{F(x)}{f(x)}$  can be represented as the quotient of two
sums of squares of the kind mentioned in theorem 43
and, since $F(x)$ is itself such a sum of squares, it follows
that, $f(x)$ must also be a quotient of two sums of
squares of the required kind. The validity of theorem 43
is accordingly established.

It would be perhaps difficult to formulate and to
demonstrate the corresponding proposition for integral
functions of two or more variables. However, I will
here merely remark that I have demonstrated in an
entirely different manner the possibility of representing
any ``definite'' integral rational function of two
%%-----File: 128.png-----%%
variables as the quotient of sums of squares of integral
functions, upon the hypothesis that the functions
represented may have as coefficients not only
rational but {\textit any} real numbers.\footnote{See
``Ueber tern\"are definite Formen,'' {\em Acta mathematica}, Vol. 17.
}

\rfa{\S\!~39.\ {\small CRITERION FOR THE POSSIBILITY OF A GEOMETRICAL
CONSTRUCTION BY MEANS OF A STRAIGHT-EDGE AND A TRANSFERER OF SEGMENTS.}}\label{p39}

Suppose we have given a problem in geometrical
construction which can be affected by means of a compass.
We shall attempt to find a criterion which will
enable us to decide, from the analytical nature of the
problem and its solutions, whether or not the construction
can be carried out by means of only a straight-edge
and a transferer of segments. Our investigation
will lead us to the following proposition.

\begin{itemize}
\item[]{\scshape Theorem 44.} Suppose we have given a problem
in geometrical construction, which is of such a
character that the analytical treatment of it
enables us to determine uniquely the co-ordinates
of the desired points from the co-ordinates
of the given points by means of the rational
operations and the extraction of the square root.
Let $n$ be the smallest number of square roots
which suffice to calculate the co-ordinates of
the points. Then, in order that the required
construction shall be possible by the drawing
of straight lines and the laying off of segments,
it is necessary and sufficient that the given geometrical
problem shall have exactly $2^n$ real solutions
for every position of the given points;
that is to say, for all values of the arbitrary
%%-----File: 129.png-----%%
parameter expressed in terms of the co-ordinates
of the given points.
\end{itemize}

{\scshape Proof}. We shall demonstrate this proposition
merely for the case where the co-ordinates of the
given points are rational functions, having rational
coefficients, of a single parameter $p$.

It is at once evident that the proposition gives a
necessary condition. In order to show that it is also
sufficient, let us assume that it is fulfilled and then,
among the $n$ square roots, consider that one which,
in the calculation of the co-ordinates of the desired
points, is first to be extracted. The expression under
this radical is a rational function $f_1(p)$, having rational
coefficients, of the parameter $p$. This rational function
cannot have a negative value for any real value
of the parameter $p$\,; for, otherwise the problem must
have imaginary solutions for certain values of $p$, which
is contrary to the given hypothesis. Hence, from
theorem 43, we conclude that $f_1(p)$ can be represented
as a quotient of the sums of squares of integral rational
functions.

Moreover, the formul{\ae}
\begin{eqnarray*}
\sqrt{a^2 + b^2 + c^2} & = & \sqrt{(\sqrt{a^2 + b^2})^2 + c^2}\\
\sqrt{a^2 + b^2 + c^2 + d^2} & = & \sqrt{(\sqrt{a^2 + b^2 + c^2})^2 + d^2}
\end{eqnarray*}
\[
\cdot \quad \cdot \quad \cdot \quad \cdot \quad \cdot \quad \cdot \quad
\cdot \quad \cdot \quad \cdot \quad \cdot \quad \cdot \quad
\cdot \quad \cdot \quad \cdot \quad \cdot \quad \cdot
\]
show that, in general, the extraction of the square root
of a sum of any number of squares may always be reduced
to the repeated extraction of the square root of
the sum of two squares.

If now we combine this conclusion with the preceding
results, it follows that the expression $\sqrt{f_1(p)}$
can certainly be constructed by means of a straight-edge
and a transferer of segments.
%%-----File: 130.png-----%%
Among the $n$ square roots, consider now the second
one to be extracted in the process of calculating
the co-ordinates of the required points. The expression
under this radical is a rational function $f_2(p, \sqrt{f_1})$
of the parameter $p$ and the square root first considered.
This function $f_2$ can never be negative for any real
arbitrary value of the parameter $p$ and for either sign
of $\sqrt{f_1}$; for, otherwise among the $2^n$ solutions of our
problem, there would exist for certain values of $p$ also
imaginary solutions, which is contrary to our hypothesis.
It follows, therefore, that $f_2$ must satisfy a quadratic
equation of the form
\[
 f_2^2 - \phi_2(p)f_2 + \psi_1(p) = 0,
\]
where $\phi_1(p)$ and $\psi_1(p)$ are, necessarily, such rational
functions of $p$ as have rational coefficients and for real
values of $p$ never become negative. From this equation, we have
\[
 f_2 = \frac{f_2^2 + \psi_1(p)}{\phi_1(p)}.
\]

Now, according to theorem 43, the functions $\phi_1(p)$ and
$\psi_1(p)$ must again be the quotient of the sums of squares
of rational functions, and, on the other hand, the expression
$f_2$ may be, from the above considerations,
constructed by means of a straight-edge and a transferer
of segments. The expression found for $f_2$ shows,
therefore, that $f_2$ is a quotient of the sum of squares
of functions which may be constructed in the same
way. Hence, the expression $\sqrt{f_2}$ can also be constructed
by means of a straight-edge and a transferer
of segments.

Just as with the expression $f_2$, any other rational
function $\phi_2(p, \sqrt{f_1})$ of $p$ and $\sqrt{f_1}$ may be shown to be
the quotient of two sums of squares of functions which
%%-----File: 131.png-----%%
may be constructed, provided this rational function
$\phi_2$ possesses the property that for real values of the
parameter $p$ and for either sign of $\sqrt{f_1}$, it never becomes
negative.

This remark permits us to extend the above method
of reasoning in the following manner.

Let $f_3(p, \sqrt{f_1}, \sqrt{f_2})$ be such an expression as depends
in a rational manner upon the three arguments
$p, \sqrt{f_1}, \sqrt{f_2}$ and of which, in the analytical calculation
of the co-ordinates of the desired points, the square
root is the third to be extracted. As before, it follows
that $f_3$ can never have negative values for real
values of $p$ and for either sign of $\sqrt{f_1}$ and $\sqrt{f_2}$. This
condition of affairs shows again that $f_3$ must satisfy a
quadratic equation of the form
\[
 f_3^2 - \phi_2(p, \sqrt{f_1}) f_3 - \psi_2(p, \sqrt{f_1}) = 0,
\]
where $\phi_2$ and $\psi_2$ are such rational functions of $p$ and
$\sqrt{f_1}$ as never become negative for any real value of $p$
and either sign of $\sqrt{f_1}$. But, according to the preceding
remark, the functions $\phi_2$ and $\psi_2$ are the quotients
of two sums of squares of functions which may be constructed
and, hence, it follows that the expression
\[
f_3 = \frac{f_3^2 + \psi_2(p, \sqrt{f_1})}{\phi_2(p, \sqrt{f_1})}
\]
is likewise possible of construction by aid of a straight-edge
and a transferer of segments.

The continuation of this method of reasoning leads
to the demonstration of theorem 44 for the case of a
single parameter $p$.

The truth of theorem 44 for the general case depends
upon whether or not theorem 43 can be generalized
in a similar manner to cover the case of two
or more variables.
%%-----File: 132.png-----%%

As an example of the application of theorem 44,
we may consider the regular polygons which may be
constructed by means of a compass. In this case, the
arbitrary parameter, $p$ does not occur, and the expressions
to be constructed all represent algebraic numbers.
We easily see that the criterion of theorem 44
is fulfilled, and, consequently, it follows that the above-mentioned
regular polygons can be constructed by the
drawing of straight lines and the laying off of segments.
We might deduce this result also directly from the
theory of the division of the circle (\textit{Kreisteilung}).

Concerning the other known problems of construction
in the elementary geometry, we will here only
mention that the problem of Malfatti may be constructed
by means of a straight-edge and a transferer
of segments. This is, however, not the case with the
contact problems of Appolonius.

%%-----File: 133.png-----%%
\newpage
\rfa{CONCLUSION.}\label{pc}

The preceding work treats essentially of the problems
of the euclidean geometry only; that is to
say, it is a discussion of the questions which present
themselves when we admit the validity of the axiom
of parallels. It is none the less important to discuss
the principles and the fundamental theorems when we
disregard the axiom of parallels. We have thus excluded
from our study the important question as to
whether it is possible to construct a geometry in a
logical manner, without introducing the notion of the
plane and the straight line, by means of only points
as elements, making use of the idea of groups of transformations,
or employing the idea of distance. This
last question has recently been the subject of considerable
study, due to the fundamental and prolific works
of Sophus Lie. However, for the complete elucidation
of this question, it would be well to divide into
several parts the axiom of Lie, that space is a numerical
multiplicity. First of all, it would seem to me
desirable to discuss thoroughly the hypothesis of Lie,
that functions which produce transformations are not
only continuous, but may also be differentiated. As
to myself, it does not seem to me probable that the
geometrical axioms included in the condition for the
possibility of differentiation are all necessary.

In the treatment of all questions of this character,
%%-----File: 134.png-----%%
I believe the methods and the principles employed in
the preceding work will be of value. As an example,
let me call attention to an investigation undertaken at
my suggestion by Mr. Dehn, and which has already
appeared.\footnote{\textit{Math. Annalen}, Vol. 53 (1900).}
In this article, he has discussed the known
theorems of Legendre concerning the sum of the angles
of a triangle, in the demonstration of which that
geometer made use of the idea of continuity.

The investigation of Mr. Dehn rests upon the axioms
of connection, of order, and of congruence; that
is to say, upon the axioms of groups I, II, IV\@.  However,
the axiom of parallels and the axiom of Archimedes
are excluded.  Moreover, the axioms of order
are stated in a more general manner than in the present
work, and in substance as follows: Among four
points $A, B, C, D$ of a straight line, there are always
two, for example $A, C$, which are separated from the
other two and conversely.  Five points $A, B, C, D,
E$ upon a straight line may always be so arranged that
$A, C$ shall be separated from $B, E$ and from $B, D$.
Consequently, $A, D$ are always separated from $B, E$
and from $C, E$, etc.  The (elliptic) geometry of Riemann,
which we have not considered in the present
work, is in this way not necessarily excluded.

Upon the basis of the axioms of connection, order,
and congruence, that is to say, the axioms I, II, IV,
we may introduce, in the well known manner, the elements
called ideal,----ideal points, ideal straight lines,
and ideal planes.  Having done this, Mr. Dehn demonstrates
the following theorem.

\begin{itemize}
\item[] If, with the exception of the straight line $t$ and
the points lying upon it, we regard all of the
%%-----File: 135.png-----%%
straight lines and all of the points (ideal or
real) of a plane as the elements of a new geometry,
we may then define a new kind of congruence
so that all of the axioms of connection,
order, and congruence, as well as the axiom of
Euclid, shall be fulfilled. In this new geometry,
the straight line $t$ takes the place of the
straight line at infinity.
\end{itemize}

This euclidean geometry, superimposed upon the
non-euclidean plane, may be called a \textit{pseudo-geometry}
and the new kind of congruence a \textit{pseudo-congruence.}

By aid of the preceding theorem, we may now introduce
an algebra of segments relating to the plane
and depending upon the developments made in \S 15,
pp. \pageref{folio46}--\pageref{folio50}. This algebra of segments permits the
demonstration of the following important theorem:

\begin{itemize}
\item[ ]If, in any triangle whatever, the sum of the angles
is greater than, equal to, or less than, two
right angles, then the same is true for all triangles.
\end{itemize}

The case where the sum of the angles is equal to
two right angles gives the well known theorem of
Legendre. However, in his demonstration, Legendre
makes use of continuity.

Mr. Dehn then discusses the connection between
the three different hypotheses relative to the sum of
the angles and the three hypotheses relative to parallels.

He arrives in this manner at the following remarkable
propositions.

\begin{itemize}
\item[ ]Upon the hypothesis that through a given point
we may draw an infinity of lines parallel to a
given straight line, it does not follow, when we
%%-----File: 136.png-----%%
exclude the axiom of Archimedes, that the sum
of the angles of a triangle is less than two right
angles, but on the contrary, this sum may be

(\emph{a}) greater than two right angles, or

(\emph{b}) equal to two right angles.
\end{itemize}

In order to demonstrate part (\emph{a}) of this theorem,
Mr. Dehn constructs a geometry where we may draw
through a point an infinity of lines parallel to a given
straight line and where, moreover, all of the theorems
of Riemann's (elliptic) geometry are valid. This geometry
may be called non-legendrian, for it is in contradiction
with that theorem of Legendre by virtue of
which the sum of the angles a triangle is never greater
than two right angles. From the existence of this
non-legendrian geometry, it follows at once that it is
impossible to demonstrate the theorem of Legendre
just mentioned without employing the axiom of Archimedes,
and in fact, Legendre made use of continuity
in his demonstration of this theorem.

For the demonstration of case (\emph{b}), Mr. Dehn constructs
a geometry where the axiom of parallels does
not hold, but where, nevertheless, all of the theorems
of the euclidean geometry are valid. Then, we have
the sum of the angles of a triangle equal to two right
angles. There exist also similar triangles, and the extremities
of the perpendiculars having the same length
and their bases upon a straight line all lie upon the
same straight line, etc. The existence of this geometry
shows that, if we disregard the axiom of Archimedes,
the axiom of parallels cannot be replaced by
any of the propositions which we usually regard as
equivalent to it.

This new geometry may be called a \textit{semi-euclidean}
%%-----File: 137.png-----%%
\textit{geometry.} As in the case of the non-legendrian geometry,
it is clear that the semi-euclidean geometry is at
the same time a non-archimedean geometry.

Mr. Dehn finally arrives at the following surprising
theorem:

\begin{itemize}
\item[] Upon the hypothesis that there exists no parallel,
it follows that the sum of the angles of a triangle
is greater than two right angles.
\end{itemize}

This theorem shows that, with respect to the axiom
of Archimedes, the two non-euclidean hypotheses concerning
parallels act very differently.

We may combine the preceding results in the following table.
\vspace*{4mm}

\begin{tabular}{c|c|c|c}
\hline
\hline
&\multicolumn{3}{c}{{\tiny THOUGH A GIVEN POINT, WE MAY DRAW}} \\
\cline{2-4}
{\tiny THE SUM OF}&{\tiny NO PARALLELS}&{\tiny ONE PARALLEL}&{\tiny AN INFINITY OF PARALLELS}\\[-1ex]
{\tiny THE ANGLES}&{\tiny TO A}&{\tiny TO A}&{\tiny TO A STRAIGHT LINE}\\[-1ex]
{\tiny OF A TRIANGLE IS}&{\tiny STRAIGHT LINE}&{\tiny STRAIGHT LINE}&\\
\hline
{\tiny $>2$ right}&{\tiny Riemann's}&{\tiny This case is}&{\tiny Non-legendrian geometry}\\[-1ex]
{\tiny angles}&{\tiny (elliptic) geometry}&{\tiny impossible}& \\
\hline
{\tiny $<2$ right}&{\tiny This case is}&{\tiny Euclidean}&{\tiny Semi-euclidean geometry}\\[-1ex]
{\tiny angles}&{\tiny impossible}&{\tiny (parabolic) geometry} & \\
\hline
{\tiny $= 2$ right}&{\tiny This case is}&{\tiny This case is}&{\tiny Geometry of Lobatschewski}\\[-1ex]
{\tiny angles}&{\tiny impossible}&{\tiny impossible}&{\tiny(hyperbolic)}\\
\hline
\hline
\end{tabular}

\vspace*{4mm}
However, as I have already remarked, the present
work is rather a critical investigation of the principles
of the euclidean geometry. In this investigation, we
have taken as a guide the following fundamental principle;
viz., to make the discussion of each question
of such a character as to examine at the same time
%%-----File: 138.png-----%%
whether or not it is possible to answer this question
by following out a previously determined method and
by employing certain limited means. This fundamental
rule seems to me to contain a general law and to
conform to the nature of things. In fact, whenever
in our mathematical investigations we encounter a
problem or suspect the existence of a theorem, our
reason is satisfied only when we possess a complete
solution of the problem or a rigorous demonstration of
the theorem, or, indeed, when we see clearly the reason
of the impossibility of the success and, consequently,
the necessity of failure.

Thus, in the modern mathematics, the question of
the impossibility of solution of certain problems plays
an important role, and the attempts made to answer
such questions have often been the occasion of discovering
new and fruitful fields for research. We recall
in this connection the demonstration by Abel of
the impossibility of solving an equation of the fifth
degree by means of radicals, as also the discovery of
the impossibility of demonstrating the axiom of parallels,
and, finally, the theorems of Hermite and Lindeman
concerning the impossibility of constructing
by algebraic means the numbers $e$ and $\pi$.

This fundamental principle, which we ought to bear
in mind when we come to discuss the principles underlying
the impossibility of demonstrations, is intimately
connected with the condition for the ``purity'' of
methods in demonstration, which in recent times has
been considered of the highest importance by many
mathematicians. The foundation of this condition is
nothing else than a subjective conception of the fundamental
principle given above. In fact, the preceding
geometrical study attempts, in general, to explain
%%-----File: 139.png-----%%
what are the axioms, hypotheses, or means, necessary
to the demonstration of a truth of elementary geometry,
and it only remains now for us to judge from the
point of view in which we place ourselves as to what
are the methods of demonstration which we should
prefer.

%%-----File: 140.png-----%%
\newpage
\rfa{APPENDIX.\footnote{The following is a summary of a paper by Professor Hilbert
which is soon to appear in full in the \textit{Math. Annalen.---Tr.}}}

The investigations by Riemann and Helmholtz of
the foundations of geometry led Lie to take up
the problem of the \textit{axiomatic} treatment of geometry
as introductory to the study of groups. This profound
mathematician introduced a system of axioms which
he showed by means of his theory of transformation
groups to be sufficient for the complete development
of geometry.\footnote{See Lie-Engel,
\textit{Theorie der Transformationsgruppen}, Vol. 3, Chapter 5.}

As the basis of his transformation groups, Lie
made the assumption that the functions defining the
group can be differentiated. Hence in Lie's development,
the question remains uninvestigated as to
whether this assumption as to the differentiability of
the functions in question is really unavoidable in developing
the subject according to the axioms of geometry,
or whether, on the other hand, it is not a
consequence of the group-conception and of the remaining
axioms of geometry. In consequence of his
method of development, Lie has also necessitated the
express statement of the axiom that the group of displacements
is produced by infinitesimal transformations.
These requirements, as well as essential parts
%%-----File: 141.png-----%%
of Lie's fundamental axioms concerning the nature of
the equation defining points of equal distance, can be
expressed geometrically in only a very unnatural and
complicated manner. Moreover, they appear only
through the analytical method used by Lie and not
as a necessity of the problem itself.

In what follows, I have therefore attempted to set
up for plane geometry a system of axioms, depending
likewise upon the conception of a group,\footnote{By
the following investigation is answered also, as I believe, a general
question concerning the theory of groups, which I proposed in my address
on ``MathematischeProbleme,'' \textit{G\"ottinger Nachrichten}, 1900, p. 17.}
which contains
only those requirements which are simple and
easily seen geometrically. In particular they do not
require the differentiability of the functions defining
displacement. The axioms of the system which I set
up are a special division of Lie's, or, as I believe, are
at once deducible from his.

My method of proof is entirely different from Lie's
method. I make use particularly of Cantor's theory
of assemblages of points and of the theorem of C. Jordan,
according to which every closed continuous plane
curve free from double points divides the plane into
an inner and an outer region.

To be sure, in the system set up by me, particular
parts are unnecessary. However, I have turned aside
from the further investigation of these conditions to
the simple statement of the axioms, and above all because
I wish to avoid a comparatively too complicated
proof, and one which is not at once geometrically evident.

In what follows I shall consider only the axioms
relating to the plane, although I suppose that an analogous
system of axioms for space can be set up which
%%-----File: 142.png-----%%
will make possible the construction of the geometry
of space in a similar manner.

We establish the following convention, namely:
We will understand by \textit{number-plane} the ordinary plane
having a rectangular system of co-ordinates $x, y$.

A continuous curve lying in this number-plane and
being free from double points and including its end
points is called a \textit{Jordan curve}. If the Jordan curve
is closed, the interior of the region of the number-plane
bounded by it is called a \textit{Jordan region}.

For the sake of easier representation and comprehension,
I shall in the following investigation formulate
the definition of the plane in a more restricted
sense than my method of proof requires,\footnote{Concerning
the broader statement of the conception of the plane see my
note, ``Ueber die Grundlagen der Geometrie,'' \textit{G\"ottinger Nachrichten}, 1901.}
namely: I
shall assume that it is possible to map\footnote{\textit{Abbilden}.}
in a reversible, single-valued manner all of the points of our geometry
at the same time upon the points lying in the
finite region of the number-plane, or upon a definite
partial system of the same. Hence, each point of our
geometry is characterized by a definite pair of numbers
$x, y$. We formulate this statement of the idea
of the plane as follows:

{\scshape Definitions of the Plane.} The plane is a system
of points which can be mapped in a reversible, single-valued
manner upon the points lying in the finite region
of the number-plane, or upon a certain partial
system of the same. To each point $A$ of our geometry,
there exists a Jordan curve in whose interior the
map of $A$ lies and all of whose points likewise represent
points of our geometry. This Jordan region is
called the domain of the point $A$. Each Jordan region
%%-----File: 143.png-----%%
contained in a Jordan region which includes the point
$A$ is likewise called a domain of $A$. If $B$ is any point
in a domain of $A$, then this domain is at the same
time called also a domain of $B$.

If $A$ and $B$ are any two points of our geometry,
then there always exists a domain which contains at
the same time both of the points $A$ and $B$.

We will define a \textit{displacement} as a reversible, single-valued
transformation of a plane into itself. Evidently
we may distinguish two kinds of reversible,
single-valued, continuous transformations of the number-plane
into itself. If we take any closed Jordan
curve in the number-plane and think of its being traversed
in a definite sense, then by such a transformation
this curve goes over into another closed Jordan
curve which is also traversed in a certain sense. We
shall assume in the present investigation that it is
traversed in the same sense as the original Jordan
curve, when we apply a transformation of the number-plane
into itself, which defines a displacement. This
assumption\footnote{Lie makes this assumption to contain the condition that the group of
displacements be generated by infinitesimal transformations. The opposite
assumption would assist essentially the demonstration in so far as the ``true
straight line'' could then be defined as the locus of those points which remain
unchanged by a displacement changing the sense in which the curve is
traversed (\textit{Umklappung}).}
necessitates the following statement of
the conception of a displacement.

{\scshape Definition of Displacement.} A displacement is
a reversible, single-valued, continuous transformation
of the maps of the given points upon the number-plane
into themselves in such a manner that a closed Jordan
curve is traversed in the same sense after the transformation
as before. A displacement by which the
%%-----File: 144.png-----%%
point $M$ remains unchanged is called
a \textit{rotation}\footnote{The term ``rotation'' is used here in the
sense of a rotatory displacement;
that is to say, only the initial and final stages and not the aggregate of
the intermediate stages of the transition enter into consideration.---\textit{Tr.}}
about
the point $M$.

In accordance with the conventions setting forth
the notions ``plane'' and ``displacement,'' we set up
the three following axioms:

{\scshape Axiom I.} \textit{If two displacements are followed out one
after the other, then the resulting map of the plane
upon itself is again a displacement.}

We say briefly:

{\scshape Axiom I\@. The Displacements Form a Group.}

{\scshape Axiom II.} \textit{If $A$ and $M$ are two arbitrary points
distinct from each other, then by a rotation about
$M$ we can always bring $A$ into an infinite number
of different positions.}

If in our geometry we define a true circle as the totality
of those points which arise by rotating about $M$
a point different from $M$, then we can express the
statement made in axiom II as follows:

{\scshape Axiom II\@. Every True Circle Consists of an
Infinite Number of Points.}

As preliminary to axiom III, we make the following
explanations:

Let $A$ be a definite point in our geometry and $A_1$,
$A_2$, $A_3$, \ldots any infinite system of points. With the
same letters we will also denote the maps of these
points upon the number-plane. About the point $A$
in the number-plane take an arbitrarily small domain
$\alpha$. If then any of the map-points $A_i$ fall within the
domain $\alpha$, we say that there are points $A_i$ arbitrarily
near the point $A$.

%%-----File: 145.png-----%%
Let $A$, $B$ be a definite pair of points in our geometry,
and let $A_1B_1$, $A_2B_2$, $A_3B_3$, $\ldots$ be any infinite system
of pairs of points. With the same letters we will
denote the maps of these pairs of points upon the
number-plane. Select about each of the points $A$
and $B$ in the number-plane an arbitrarily small domain
$\alpha$ and $\beta$, respectively. If then there are pairs of
points $A_iB_i$ such that $A_i$ falls within the domain $\alpha$ and
at the same time $B_i$ falls within the domain $\beta$, we say
that there are segments $A_iB_i$ lying arbitrarily near the
segment $AB$.

Let $ABC$ a definite triad of points in our geometry,
and let $A_1B_1C_1$, $A_2B_2C_2$, $A_3B_3C_3$, $\ldots$ be any infinite
system of triads of points. With the same letters
we will also denote the maps of these triads of
points upon the number-plane. About each of the
points $A$, $B$, $C$ in the number-plane take an arbitrarily
small domain $\alpha$, $\beta$, $\gamma$, respectively. If then there are
triads of points $A_iB_iC_i$ such that $A_i$ falls in the domain
$\alpha$, and likewise  $B_i$ in the domain $\beta$ and $C_i$ in the
domain $\gamma$, then we say that there are triangles $A_iB_iC_i$
lying arbitrarily near to the triangle $ABC$.

{\scshape Axiom III.} \textit{If there are displacements of such a
kind that triangles arbitrarily near the triangle
$ABC$ can be brought arbitrarily near to the triangle
$A'B'C'$, then there always exists a displacement
by which the triangle $ABC$ goes over exactly
into the triangle $A'B'C'$.}\footnote{It
is sufficient to assume that axiom III holds for sufficiently small domains
as Lie has done. My method of proof may be so changed as to make
use of only this narrower assumption.}

The content of this axiom can be briefly expressed
as follows:

%%-----File: 146.png-----%%
{\scshape Axiom III\@. The Displacements Form a Closed System.}

We call special attention to the following particular
cases of axiom III.

If there are rotations about a point $M$ of the kind
that segments lying arbitrarily near the segment $AB$
can be brought arbitrarily near the segment $A'B'$, then
there is always such a rotation about $M$ possible by
which the segment $AB$ goes over exactly into the segment $A'B'$.

If there are displacements of the kind that segments
arbitrarily near the segment $AB$ can be brought
arbitrarily near to the segment $A'B'$, then there is
always a displacement possible by which the segment
$AB$ goes over exactly into the segment $A'B'$.

If there are rotations about the point $M$ of the
kind that points arbitrarily near the point A can be
brought arbitrarily near the point $A'$, then there is
always such a rotation about $M$ possible by which $A$
goes over exactly into the point $A'$.

I now prove the following proposition:

\begin{quote}
A geometry in which axioms I--III are fulfilled is
either the euclidean or the bolyai-lobatchefskian
geometry.
\end{quote}

If we wish to obtain only the euclidean geometry,
it is necessary merely to make in connection with
axiom I the additional statement that the groups of
displacements shall possess an invariant sub-group.
This additional statement takes the place of the axioms
of parallels.

In what follows, I will briefly outline the general
idea of my method of proof.\footnote{The
complete proof will appear later in the \textit{Math. Annalen}.}
%%-----File: 147.png-----%%
Within the domain of a certain point $M$ construct
in a particular manner a certain point-configuration
$kk$, and upon this configuration construct a certain
point $K$. We then base our investigation upon the
true circle $k$ about $M$ and passing through $K$. It may
be easily shown that the true circle $k$ is an assemblage
of points which is closed and in itself dense. It constitutes,
therefore, a perfect assemblage of points.

The next objective point in our demonstration is
to show that the true circle $k$ is a closed Jordan curve.
We do this in that we first show the possibility of a
cyclical arrangement of the points of the true circle
$k$, from which it follows that we may map in a reversible,
single-valued manner the points of $k$ upon
the points of an ordinary circle. Finally, we show
that this map must necessarily be a continuous one.
Furthermore, it follows also that the originally constructed
point-configuration $kk$ is identical with the
true circle $k$. Moreover, the law holds that each true
circle inside of $k$ is likewise a closed Jordan curve.

We turn now to the investigation of the group of
all the displacements which by the rotation of the
plane about $M$ transforms a definite true circle $k$ into
itself. This group possesses the following properties: (1) Every
displacement which leaves one point
of $k$ undisturbed, leaves all points of $k$ undisturbed.
(2) There always exists a displacement which changes
any given point of $k$ into any other given point of $k$.
(3) The group of displacements is a continuous one.
These three properties determine completely the construction
of the group of transformations of all the
displacements of the true circle into itself. We set up
the following proposition: The group of all the displacements
of the true circle into itself, which are rotations
%%-----File: 148.png-----%%
about M, is holoedric, isomorphic with the
group of ordinary rotations of the ordinary circle into
itself.

Moreover, we investigate the group of displacements
of all the points of our plane by a rotation about
$M$. The law holds that, aside from the identity, there
is no rotation of the plane about $M$ which leaves every
point of the true circle undisturbed. We now see that
every true circle is a Jordan curve and deduce formul{\ae}
for the transformation of that group of all the rotations.
Finally, the proposition easily follows that: If
any two points remain fixed by a displacement of the
plane, then all points remain fixed; that is to say, the
displacement is the identity. Each point of the plane
may be indeed made to go over into any other point
of the plane by means of a displacement.

Our further important objective point is to define
the idea of the true straight line in our geometry and
deduce those properties of it which are necessary in
the further development of geometry. First of all, the
notions ``semi-rotation'' and ``middle of a segment''
are defined. A segment has at most one middle, and,
when we know the middle of one segment, then every
smaller segment possesses a middle.

In order to pass judgment as to the position of the
middle of a segment, we need particular propositions
concerning true circles which are mutually tangent,
and indeed the question depends upon the construction
of two congruent circles tangent to each other externally
in one and only one point. We derive also
a more general proposition concerning circles which
are tangent to each other internally and consequently
a theorem covering the special case where the circle
%%-----File: 149.png-----%%
which is tangent internally to a second passes through
the centre of that circle.

Moreover, a sufficiently small definite segment is
taken as a unit segment, and from this by continued
bisection and semi-rotation a system of points is constructed
of the kind that to each point of this system
a definite number $a$ corresponds, which is rational
and has as denominator some power of $2$. By setting
up a law concerning this correspondence, the points of
the above system are so arranged that the above laws
concerning mutually tangent circles are valid. It is
now shown that the points corresponding to the
numbers $\frac{1}{2}, \frac{1}{4}, \frac{1}{8}, \dots$ converge toward the point 0. This
result is generalized step by step until it is finally
shown that every series of points of our system converges,
so soon as the corresponding series of numbers
converges.

From what has been said, the definition of the true
straight line follows as a system of points which arise
from two fundamental points, if we apply repeatedly
a semi-rotation, take the middle point, and add to the
assemblage the points of condensation of the system of
points which arises. We can then prove that the true
straight line is a continuous curve, possessing no
double points and having with any other true straight
line at most one point in common. Furthermore, it
can be shown that the true straight line cuts each
circle drawn about one of its points, and from this it
follows that any two arbitrary points of the plane can
always be joined by a true straight line. We see also
that in our geometry the laws of congruence hold, by
which however two triangles are proven to be congruent
if they are traversed in the same sense.

With regard to the position of the systems of all
%%-----File: 150.png-----%%
the true straight lines with respect to one another,
there are two cases to distinguish, according as the
axiom of parallels holds, or through each point there
exists two straight lines which separate the straight
lines which cut the given straight line from those
which do not cut it. In the first case we have the
euclidean and in the second the bolyai-lobatschefskian
geometry.

\newpage

\small
\pagenumbering{gobble}
\begin{verbatim}

End of Project Gutenberg's The Foundations of Geometry,
by David Hilbert

*** END OF THIS PROJECT GUTENBERG EBOOK FOUNDATIONS OF GEOMETRY ***

*** This file should be named 17384-t.tex or 17384-t.zip ***
*** or                    17384-pdf.pdf or 17384-pdf.pdf ***
This and all associated files of various formats will be found in:
        http://www.gutenberg.org/1/7/3/8/17384/

Produced by Joshua Hutchinson, Roger Frank, David Starner and
the Online Distributed Proofreading Team at http://www.pgdp.net


Updated editions will replace the previous one--the old editions
will be renamed.

Creating the works from public domain print editions means that no
one owns a United States copyright in these works, so the Foundation
(and you!) can copy and distribute it in the United States without
permission and without paying copyright royalties.  Special rules,
set forth in the General Terms of Use part of this license, apply to
copying and distributing Project Gutenberg-tm electronic works to
protect the PROJECT GUTENBERG-tm concept and trademark.  Project
Gutenberg is a registered trademark, and may not be used if you
charge for the eBooks, unless you receive specific permission.  If
you do not charge anything for copies of this eBook, complying with
the rules is very easy.  You may use this eBook for nearly any
purpose such as creation of derivative works, reports, performances
and research.  They may be modified and printed and given away--you
may do practically ANYTHING with public domain eBooks.
Redistribution is subject to the trademark license, especially
commercial redistribution.



*** START: FULL LICENSE ***

THE FULL PROJECT GUTENBERG LICENSE PLEASE READ THIS BEFORE YOU
DISTRIBUTE OR USE THIS WORK

To protect the Project Gutenberg-tm mission of promoting the free
distribution of electronic works, by using or distributing this work
(or any other work associated in any way with the phrase "Project
Gutenberg"), you agree to comply with all the terms of the Full
Project Gutenberg-tm License (available with this file or online at
http://gutenberg.net/license).


Section 1.  General Terms of Use and Redistributing Project
Gutenberg-tm electronic works

1.A.  By reading or using any part of this Project Gutenberg-tm
electronic work, you indicate that you have read, understand, agree
to and accept all the terms of this license and intellectual
property (trademark/copyright) agreement.  If you do not agree to
abide by all the terms of this agreement, you must cease using and
return or destroy all copies of Project Gutenberg-tm electronic
works in your possession. If you paid a fee for obtaining a copy of
or access to a Project Gutenberg-tm electronic work and you do not
agree to be bound by the terms of this agreement, you may obtain a
refund from the person or entity to whom you paid the fee as set
forth in paragraph 1.E.8.

1.B.  "Project Gutenberg" is a registered trademark.  It may only be
used on or associated in any way with an electronic work by people
who agree to be bound by the terms of this agreement.  There are a
few things that you can do with most Project Gutenberg-tm electronic
works even without complying with the full terms of this agreement.
See paragraph 1.C below.  There are a lot of things you can do with
Project Gutenberg-tm electronic works if you follow the terms of
this agreement and help preserve free future access to Project
Gutenberg-tm electronic works.  See paragraph 1.E below.

1.C.  The Project Gutenberg Literary Archive Foundation ("the
Foundation" or PGLAF), owns a compilation copyright in the
collection of Project Gutenberg-tm electronic works.  Nearly all the
individual works in the collection are in the public domain in the
United States.  If an individual work is in the public domain in the
United States and you are located in the United States, we do not
claim a right to prevent you from copying, distributing, performing,
displaying or creating derivative works based on the work as long as
all references to Project Gutenberg are removed.  Of course, we hope
that you will support the Project Gutenberg-tm mission of promoting
free access to electronic works by freely sharing Project
Gutenberg-tm works in compliance with the terms of this agreement
for keeping the Project Gutenberg-tm name associated with the work.
You can easily comply with the terms of this agreement by keeping
this work in the same format with its attached full Project
Gutenberg-tm License when you share it without charge with others.

1.D.  The copyright laws of the place where you are located also
govern what you can do with this work.  Copyright laws in most
countries are in a constant state of change.  If you are outside the
United States, check the laws of your country in addition to the
terms of this agreement before downloading, copying, displaying,
performing, distributing or creating derivative works based on this
work or any other Project Gutenberg-tm work.  The Foundation makes
no representations concerning the copyright status of any work in
any country outside the United States.

1.E.  Unless you have removed all references to Project Gutenberg:

1.E.1.  The following sentence, with active links to, or other
immediate access to, the full Project Gutenberg-tm License must
appear prominently whenever any copy of a Project Gutenberg-tm work
(any work on which the phrase "Project Gutenberg" appears, or with
which the phrase "Project Gutenberg" is associated) is accessed,
displayed, performed, viewed, copied or distributed:

This eBook is for the use of anyone anywhere at no cost and with
almost no restrictions whatsoever.  You may copy it, give it away or
re-use it under the terms of the Project Gutenberg License included
with this eBook or online at www.gutenberg.net

1.E.2.  If an individual Project Gutenberg-tm electronic work is
derived from the public domain (does not contain a notice indicating
that it is posted with permission of the copyright holder), the work
can be copied and distributed to anyone in the United States without
paying any fees or charges.  If you are redistributing or providing
access to a work with the phrase "Project Gutenberg" associated with
or appearing on the work, you must comply either with the
requirements of paragraphs 1.E.1 through 1.E.7 or obtain permission
for the use of the work and the Project Gutenberg-tm trademark as
set forth in paragraphs 1.E.8 or 1.E.9.

1.E.3.  If an individual Project Gutenberg-tm electronic work is
posted with the permission of the copyright holder, your use and
distribution must comply with both paragraphs 1.E.1 through 1.E.7
and any additional terms imposed by the copyright holder.
Additional terms will be linked to the Project Gutenberg-tm License
for all works posted with the permission of the copyright holder
found at the beginning of this work.

1.E.4.  Do not unlink or detach or remove the full Project
Gutenberg-tm License terms from this work, or any files containing a
part of this work or any other work associated with Project
Gutenberg-tm.

1.E.5.  Do not copy, display, perform, distribute or redistribute
this electronic work, or any part of this electronic work, without
prominently displaying the sentence set forth in paragraph 1.E.1
with active links or immediate access to the full terms of the
Project Gutenberg-tm License.

1.E.6.  You may convert to and distribute this work in any binary,
compressed, marked up, nonproprietary or proprietary form, including
any word processing or hypertext form.  However, if you provide
access to or distribute copies of a Project Gutenberg-tm work in a
format other than "Plain Vanilla ASCII" or other format used in the
official version posted on the official Project Gutenberg-tm web
site (www.gutenberg.net), you must, at no additional cost, fee or
expense to the user, provide a copy, a means of exporting a copy, or
a means of obtaining a copy upon request, of the work in its
original "Plain Vanilla ASCII" or other form.  Any alternate format
must include the full Project Gutenberg-tm License as specified in
paragraph 1.E.1.

1.E.7.  Do not charge a fee for access to, viewing, displaying,
performing, copying or distributing any Project Gutenberg-tm works
unless you comply with paragraph 1.E.8 or 1.E.9.

1.E.8.  You may charge a reasonable fee for copies of or providing
access to or distributing Project Gutenberg-tm electronic works
provided that

- You pay a royalty fee of 20% of the gross profits you derive from
   the use of Project Gutenberg-tm works calculated using the method
   you already use to calculate your applicable taxes.  The fee is
   owed to the owner of the Project Gutenberg-tm trademark, but he
   has agreed to donate royalties under this paragraph to the
   Project Gutenberg Literary Archive Foundation.  Royalty payments
   must be paid within 60 days following each date on which you
   prepare (or are legally required to prepare) your periodic tax
   returns.  Royalty payments should be clearly marked as such and
   sent to the Project Gutenberg Literary Archive Foundation at the
   address specified in Section 4, "Information about donations to
   the Project Gutenberg Literary Archive Foundation."

- You provide a full refund of any money paid by a user who notifies
   you in writing (or by e-mail) within 30 days of receipt that s/he
   does not agree to the terms of the full Project Gutenberg-tm
   License.  You must require such a user to return or
   destroy all copies of the works possessed in a physical medium
   and discontinue all use of and all access to other copies of
   Project Gutenberg-tm works.

- You provide, in accordance with paragraph 1.F.3, a full refund of
   any money paid for a work or a replacement copy, if a defect in
   the electronic work is discovered and reported to you within 90
   days of receipt of the work.

- You comply with all other terms of this agreement for free
   distribution of Project Gutenberg-tm works.

1.E.9.  If you wish to charge a fee or distribute a Project
Gutenberg-tm electronic work or group of works on different terms
than are set forth in this agreement, you must obtain permission in
writing from both the Project Gutenberg Literary Archive Foundation
and Michael Hart, the owner of the Project Gutenberg-tm trademark.
Contact the Foundation as set forth in Section 3 below.

1.F.

1.F.1.  Project Gutenberg volunteers and employees expend
considerable effort to identify, do copyright research on,
transcribe and proofread public domain works in creating the Project
Gutenberg-tm collection.  Despite these efforts, Project
Gutenberg-tm electronic works, and the medium on which they may be
stored, may contain "Defects," such as, but not limited to,
incomplete, inaccurate or corrupt data, transcription errors, a
copyright or other intellectual property infringement, a defective
or damaged disk or other medium, a computer virus, or computer codes
that damage or cannot be read by your equipment.

1.F.2.  LIMITED WARRANTY, DISCLAIMER OF DAMAGES - Except for the
"Right of Replacement or Refund" described in paragraph 1.F.3, the
Project Gutenberg Literary Archive Foundation, the owner of the
Project Gutenberg-tm trademark, and any other party distributing a
Project Gutenberg-tm electronic work under this agreement, disclaim
all liability to you for damages, costs and expenses, including
legal fees.  YOU AGREE THAT YOU HAVE NO REMEDIES FOR NEGLIGENCE,
STRICT LIABILITY, BREACH OF WARRANTY OR BREACH OF CONTRACT EXCEPT
THOSE PROVIDED IN PARAGRAPH F3.  YOU AGREE THAT THE FOUNDATION, THE
TRADEMARK OWNER, AND ANY DISTRIBUTOR UNDER THIS AGREEMENT WILL NOT
BE LIABLE TO YOU FOR ACTUAL, DIRECT, INDIRECT, CONSEQUENTIAL,
PUNITIVE OR INCIDENTAL DAMAGES EVEN IF YOU GIVE NOTICE OF THE
POSSIBILITY OF SUCH DAMAGE.

1.F.3.  LIMITED RIGHT OF REPLACEMENT OR REFUND - If you discover a
defect in this electronic work within 90 days of receiving it, you
can receive a refund of the money (if any) you paid for it by
sending a written explanation to the person you received the work
from.  If you received the work on a physical medium, you must
return the medium with your written explanation.  The person or
entity that provided you with the defective work may elect to
provide a replacement copy in lieu of a refund.  If you received the
work electronically, the person or entity providing it to you may
choose to give you a second opportunity to receive the work
electronically in lieu of a refund.  If the second copy is also
defective, you may demand a refund in writing without further
opportunities to fix the problem.

1.F.4.  Except for the limited right of replacement or refund set
forth in paragraph 1.F.3, this work is provided to you 'AS-IS', WITH
NO OTHER WARRANTIES OF ANY KIND, EXPRESS OR IMPLIED, INCLUDING BUT
NOT LIMITED TO WARRANTIES OF MERCHANTIBILITY OR FITNESS FOR ANY
PURPOSE.

1.F.5.  Some states do not allow disclaimers of certain implied
warranties or the exclusion or limitation of certain types of
damages. If any disclaimer or limitation set forth in this agreement
violates the law of the state applicable to this agreement, the
agreement shall be interpreted to make the maximum disclaimer or
limitation permitted by the applicable state law.  The invalidity or
unenforceability of any provision of this agreement shall not void
the remaining provisions.

1.F.6.  INDEMNITY - You agree to indemnify and hold the Foundation,
the trademark owner, any agent or employee of the Foundation, anyone
providing copies of Project Gutenberg-tm electronic works in
accordance with this agreement, and any volunteers associated with
the production, promotion and distribution of Project Gutenberg-tm
electronic works, harmless from all liability, costs and expenses,
including legal fees, that arise directly or indirectly from any of
the following which you do or cause to occur: (a) distribution of
this or any Project Gutenberg-tm work, (b) alteration, modification,
or additions or deletions to any Project Gutenberg-tm work, and (c)
any Defect you cause.


Section  2.  Information about the Mission of Project Gutenberg-tm

Project Gutenberg-tm is synonymous with the free distribution of
electronic works in formats readable by the widest variety of
computers including obsolete, old, middle-aged and new computers.
It exists because of the efforts of hundreds of volunteers and
donations from people in all walks of life.

Volunteers and financial support to provide volunteers with the
assistance they need, is critical to reaching Project Gutenberg-tm's
goals and ensuring that the Project Gutenberg-tm collection will
remain freely available for generations to come.  In 2001, the
Project Gutenberg Literary Archive Foundation was created to provide
a secure and permanent future for Project Gutenberg-tm and future
generations. To learn more about the Project Gutenberg Literary
Archive Foundation and how your efforts and donations can help, see
Sections 3 and 4 and the Foundation web page at
http://www.pglaf.org.


Section 3.  Information about the Project Gutenberg Literary Archive
Foundation

The Project Gutenberg Literary Archive Foundation is a non profit
501(c)(3) educational corporation organized under the laws of the
state of Mississippi and granted tax exempt status by the Internal
Revenue Service.  The Foundation's EIN or federal tax identification
number is 64-6221541.  Its 501(c)(3) letter is posted at
http://pglaf.org/fundraising.  Contributions to the Project
Gutenberg Literary Archive Foundation are tax deductible to the full
extent permitted by U.S. federal laws and your state's laws.

The Foundation's principal office is located at 4557 Melan Dr. S.
Fairbanks, AK, 99712., but its volunteers and employees are
scattered throughout numerous locations.  Its business office is
located at 809 North 1500 West, Salt Lake City, UT 84116, (801)
596-1887, email business@pglaf.org.  Email contact links and up to
date contact information can be found at the Foundation's web site
and official page at http://pglaf.org

For additional contact information:
     Dr. Gregory B. Newby
     Chief Executive and Director
     gbnewby@pglaf.org

Section 4.  Information about Donations to the Project Gutenberg
Literary Archive Foundation

Project Gutenberg-tm depends upon and cannot survive without wide
spread public support and donations to carry out its mission of
increasing the number of public domain and licensed works that can
be freely distributed in machine readable form accessible by the
widest array of equipment including outdated equipment.  Many small
donations ($1 to $5,000) are particularly important to maintaining
tax exempt status with the IRS.

The Foundation is committed to complying with the laws regulating
charities and charitable donations in all 50 states of the United
States.  Compliance requirements are not uniform and it takes a
considerable effort, much paperwork and many fees to meet and keep
up with these requirements.  We do not solicit donations in
locations where we have not received written confirmation of
compliance.  To SEND DONATIONS or determine the status of compliance
for any particular state visit http://pglaf.org

While we cannot and do not solicit contributions from states where
we have not met the solicitation requirements, we know of no
prohibition against accepting unsolicited donations from donors in
such states who approach us with offers to donate.

International donations are gratefully accepted, but we cannot make
any statements concerning tax treatment of donations received from
outside the United States.  U.S. laws alone swamp our small staff.

Please check the Project Gutenberg Web pages for current donation
methods and addresses.  Donations are accepted in a number of other
ways including including checks, online payments and credit card
donations.  To donate, please visit: http://pglaf.org/donate


Section 5.  General Information About Project Gutenberg-tm
electronic works.

Professor Michael S. Hart is the originator of the Project
Gutenberg-tm concept of a library of electronic works that could be
freely shared with anyone.  For thirty years, he produced and
distributed Project Gutenberg-tm eBooks with only a loose network of
volunteer support.

Project Gutenberg-tm eBooks are often created from several printed
editions, all of which are confirmed as Public Domain in the U.S.
unless a copyright notice is included.  Thus, we do not necessarily
keep eBooks in compliance with any particular paper edition.

Most people start at our Web site which has the main PG search
facility:

     http://www.gutenberg.net

This Web site includes information about Project Gutenberg-tm,
including how to make donations to the Project Gutenberg Literary
Archive Foundation, how to help produce our new eBooks, and how to
subscribe to our email newsletter to hear about new eBooks.

*** END: FULL LICENSE ***

\end{verbatim}
\end{document}
This is pdfeTeX, Version 3.141592-1.21a-2.2 (MiKTeX 2.4) (preloaded format=latex 2005.4.4)  23 DEC 2005 23:40
entering extended mode
**hilbert
(hilbert.tex
LaTeX2e <2003/12/01>
Babel <v3.8a> and hyphenation patterns for english, french, german, ngerman, du
mylang, nohyphenation, loaded.
(C:\texmf\tex\latex\base\book.cls
Document Class: book 2004/02/16 v1.4f Standard LaTeX document class
(C:\texmf\tex\latex\base\bk11.clo
File: bk11.clo 2004/02/16 v1.4f Standard LaTeX file (size option)
)
\c@part=\count79
\c@chapter=\count80
\c@section=\count81
\c@subsection=\count82
\c@subsubsection=\count83
\c@paragraph=\count84
\c@subparagraph=\count85
\c@figure=\count86
\c@table=\count87
\abovecaptionskip=\skip41
\belowcaptionskip=\skip42
\bibindent=\dimen102
) (C:\texmf\tex\latex\amsmath\amsmath.sty
Package: amsmath 2000/07/18 v2.13 AMS math features
\@mathmargin=\skip43

For additional information on amsmath, use the `?' option.
(C:\texmf\tex\latex\amsmath\amstext.sty
Package: amstext 2000/06/29 v2.01
 (C:\texmf\tex\latex\amsmath\amsgen.sty
File: amsgen.sty 1999/11/30 v2.0
\@emptytoks=\toks14
\ex@=\dimen103
)) (C:\texmf\tex\latex\amsmath\amsbsy.sty
Package: amsbsy 1999/11/29 v1.2d
\pmbraise@=\dimen104
)
(C:\texmf\tex\latex\amsmath\amsopn.sty
Package: amsopn 1999/12/14 v2.01 operator names
)
\inf@bad=\count88
LaTeX Info: Redefining \frac on input line 211.
\uproot@=\count89
\leftroot@=\count90
LaTeX Info: Redefining \overline on input line 307.
\classnum@=\count91
\DOTSCASE@=\count92
LaTeX Info: Redefining \ldots on input line 379.
LaTeX Info: Redefining \dots on input line 382.
LaTeX Info: Redefining \cdots on input line 467.
\Mathstrutbox@=\box26
\strutbox@=\box27
\big@size=\dimen105
LaTeX Font Info:    Redeclaring font encoding OML on input line 567.
LaTeX Font Info:    Redeclaring font encoding OMS on input line 568.
\macc@depth=\count93
\c@MaxMatrixCols=\count94
\dotsspace@=\muskip10
\c@parentequation=\count95
\dspbrk@lvl=\count96
\tag@help=\toks15
\row@=\count97
\column@=\count98
\maxfields@=\count99
\andhelp@=\toks16
\eqnshift@=\dimen106
\alignsep@=\dimen107
\tagshift@=\dimen108
\tagwidth@=\dimen109
\totwidth@=\dimen110
\lineht@=\dimen111
\@envbody=\toks17
\multlinegap=\skip44
\multlinetaggap=\skip45
\mathdisplay@stack=\toks18
LaTeX Info: Redefining \[ on input line 2666.
LaTeX Info: Redefining \] on input line 2667.
)
(C:\texmf\tex\latex\amsfonts\amssymb.sty
Package: amssymb 2002/01/22 v2.2d

(C:\texmf\tex\latex\amsfonts\amsfonts.sty
Package: amsfonts 2001/10/25 v2.2f
\symAMSa=\mathgroup4
\symAMSb=\mathgroup5
LaTeX Font Info:    Overwriting math alphabet `\mathfrak' in version `bold'
(Font)                  U/euf/m/n --> U/euf/b/n on input line 132.
))
(C:\texmf\tex\latex\tools\tabularx.sty
Package: tabularx 1999/01/07 v2.07 `tabularx' package (DPC)
 (C:\texmf\tex\latex\tools\array.sty
Package: array 2003/12/17 v2.4a Tabular extension package (FMi)
\col@sep=\dimen112
\extrarowheight=\dimen113
\NC@list=\toks19
\extratabsurround=\skip46
\backup@length=\skip47
)
\TX@col@width=\dimen114
\TX@old@table=\dimen115
\TX@old@col=\dimen116
\TX@target=\dimen117
\TX@delta=\dimen118
\TX@cols=\count100
\TX@ftn=\toks20
)
(wrapfig.sty
\wrapoverhang=\dimen119
\WF@size=\dimen120
\c@WF@wrappedlines=\count101
\WF@box=\box28
\WF@everypar=\toks21
Package: wrapfig 2003/01/31  v 3.6
) (C:\texmf\tex\latex\graphics\graphicx.sty
Package: graphicx 1999/02/16 v1.0f Enhanced LaTeX Graphics (DPC,SPQR)

(C:\texmf\tex\latex\graphics\keyval.sty
Package: keyval 1999/03/16 v1.13 key=value parser (DPC)
\KV@toks@=\toks22
)
(C:\texmf\tex\latex\graphics\graphics.sty
Package: graphics 2001/07/07 v1.0n Standard LaTeX Graphics (DPC,SPQR)
 (C:\texmf\tex\latex\graphics\trig.sty
Package: trig 1999/03/16 v1.09 sin cos tan (DPC)
) (C:\texmf\tex\latex\00miktex\graphics.cfg
File: graphics.cfg 2003/03/12 v1.1 MiKTeX 'graphics' configuration
)
Package graphics Info: Driver file: pdftex.def on input line 80.

(C:\texmf\tex\latex\graphics\pdftex.def
File: pdftex.def 2002/06/19 v0.03k graphics/color for pdftex
\Gread@gobject=\count102
))
\Gin@req@height=\dimen121
\Gin@req@width=\dimen122
)
(C:\texmf\tex\latex\geometry\geometry.sty
Package: geometry 2002/07/08 v3.2 Page Geometry
\Gm@cnth=\count103
\Gm@cntv=\count104
\c@Gm@tempcnt=\count105
\Gm@bindingoffset=\dimen123
\Gm@wd@mp=\dimen124
\Gm@odd@mp=\dimen125
\Gm@even@mp=\dimen126
\Gm@dimlist=\toks23

(C:\texmf\tex\latex\geometry\geometry.cfg)) (hilbert.aux)
LaTeX Font Info:    Checking defaults for OML/cmm/m/it on input line 54.
LaTeX Font Info:    ... okay on input line 54.
LaTeX Font Info:    Checking defaults for T1/cmr/m/n on input line 54.
LaTeX Font Info:    ... okay on input line 54.
LaTeX Font Info:    Checking defaults for OT1/cmr/m/n on input line 54.
LaTeX Font Info:    ... okay on input line 54.
LaTeX Font Info:    Checking defaults for OMS/cmsy/m/n on input line 54.
LaTeX Font Info:    ... okay on input line 54.
LaTeX Font Info:    Checking defaults for OMX/cmex/m/n on input line 54.
LaTeX Font Info:    ... okay on input line 54.
LaTeX Font Info:    Checking defaults for U/cmr/m/n on input line 54.
LaTeX Font Info:    ... okay on input line 54.
 (supp-pdf.tex
(supp-mis.tex
loading : Context Support Macros / Missing
\protectiondepth=\count106
\scratchcounter=\count107
\scratchdimen=\dimen127
\scratchskip=\skip48
\scratchmuskip=\muskip11
\scratchbox=\box29
\scratchread=\read1
\scratchwrite=\write3
\nextbox=\box30
\nextdepth=\dimen128
\everyline=\toks24
\!!counta=\count108
\!!countb=\count109
\recursecounter=\count110
)
loading : Context Support Macros / PDF
\nofMPsegments=\count111
\nofMParguments=\count112
)
-------------------- Geometry parameters
paper: class default
landscape: --
twocolumn: --
twoside: --
asymmetric: --
h-parts: 93.95122pt, 426.39256pt, 93.95122pt
v-parts: 93.95122pt, 607.06755pt, 93.95122pt
hmarginratio: --
vmarginratio: --
lines: --
heightrounded: --
bindingoffset: 0.0pt
truedimen: --
includehead: --
includefoot: --
includemp: --
driver: pdftex
-------------------- Page layout dimensions and switches
\paperwidth  614.295pt
\paperheight 794.96999pt
\textwidth  426.39256pt
\textheight 607.06755pt
\oddsidemargin  21.68123pt
\evensidemargin 21.68123pt
\topmargin  -10.19257pt
\headheight 12.0pt
\headsep    19.8738pt
\footskip   27.46295pt
\marginparwidth 62.0pt
\marginparsep   7.0pt
\columnsep  10.0pt
\skip\footins  10.0pt plus 4.0pt minus 2.0pt
\hoffset 0.0pt
\voffset 0.0pt
\mag 1000

(1in=72.27pt, 1cm=28.45pt)
-----------------------
 [1

{psfonts.map}] [1


] [1

] [2

]
LaTeX Font Info:    Try loading font information for U+msa on input line 204.
 (C:\texmf\tex\latex\amsfonts\umsa.fd
File: umsa.fd 2002/01/19 v2.2g AMS font definitions
)
LaTeX Font Info:    Try loading font information for U+msb on input line 204.

(C:\texmf\tex\latex\amsfonts\umsb.fd
File: umsb.fd 2002/01/19 v2.2g AMS font definitions
)
Overfull \hbox (4.43442pt too wide) in paragraph at lines 205--205
[]\OT1/cmr/m/n/9 PAGE 
 []

LaTeX Font Info:    Try loading font information for OMS+cmr on input line 211.

(C:\texmf\tex\latex\base\omscmr.fd
File: omscmr.fd 1999/05/25 v2.5h Standard LaTeX font definitions
)
LaTeX Font Info:    Font shape `OMS/cmr/m/n' in size <10.95> not available
(Font)              Font shape `OMS/cmsy/m/n' tried instead on input line 211.
 [0

]
Overfull \hbox (3.08954pt too wide) in paragraph at lines 246--293
[][] 
 []

[1] [1

]
LaTeX Font Info:    Font shape `OMS/cmr/bx/n' in size <12> not available
(Font)              Font shape `OMS/cmsy/b/n' tried instead on input line 339.
 [2] <images/f001.png, id=72, 402.5439pt x 38.3031pt>
File: images/f001.png Graphic file (type png)

<use images/f001.png> <images/f002.png, id=74, 408.0846pt x 36.8577pt>
File: images/f002.png Graphic file (type png)

<use images/f002.png> [3] <images/f003.png, id=79, 252.945pt x 214.1601pt>
File: images/f003.png Graphic file (type png)

<use images/f003.png> [4 <images/f001.png> <images/f002.png> <images/f003.png>]
<images/f004.png, id=90, 448.3149pt x 36.135pt>
File: images/f004.png Graphic file (type png)
 <use images/f004.png>
<images/f005.png, id=92, 428.3202pt x 192.72pt>
File: images/f005.png Graphic file (type png)
 <use images/f005.png> [5 <images/f004.png> <images/f005.png>]
<images/f006.png, id=104, 423.0204pt x 28.1853pt>
File: images/f006.png Graphic file (type png)
 <use images/f006.png>
<images/f007.png, id=106, 321.1197pt x 259.6902pt>
File: images/f007.png Graphic file (type png)
 <use images/f007.png>
[6 <images/f006.png>] [7 <images/f007.png>]
<images/f008.png, id=119, 435.0654pt x 99.0099pt>
File: images/f008.png Graphic file (type png)
 <use images/f008.png>
[8 <images/f008.png>] [9] [10]
<images/f009.png, id=135, 493.845pt x 111.0549pt>
File: images/f009.png Graphic file (type png)
 <use images/f009.png>
<images/f010.png, id=137, 496.4949pt x 94.9146pt>
File: images/f010.png Graphic file (type png)
 <use images/f010.png>
[11 <images/f009.png> <images/f010.png>]
<images/f011.png, id=146, 489.7497pt x 115.1502pt>
File: images/f011.png Graphic file (type png)
 <use images/f011.png>
[12 <images/f011.png>] <images/f012.png, id=153, 496.4949pt x 164.5347pt>
File: images/f012.png Graphic file (type png)

<use images/f012.png> [13 <images/f012.png>] [14] [15] [16]
<images/f013.png, id=173, 500.5902pt x 34.6896pt>
File: images/f013.png Graphic file (type png)
 <use images/f013.png>
[17] <images/f014.png, id=178, 310.5201pt x 234.1548pt>
File: images/f014.png Graphic file (type png)
 <use images/f014.png>
[18 <images/f013.png> <images/f014.png>] [19]
<images/f015.png, id=194, 220.9053pt x 188.6247pt>
File: images/f015.png Graphic file (type png)
 <use images/f015.png>
[20] [21 <images/f015.png>] [22] [23]
Overfull \hbox (6.68538pt too wide) in paragraph at lines 1807--1812
\OT1/cmr/m/it/10.95 com-plex num-ber sys-tem\OT1/cmr/m/n/10.95 , or sim-ply a \
OT1/cmr/m/it/10.95 num-ber sys-tem\OT1/cmr/m/n/10.95 . A num-ber sys-tem is cal
led \OT1/cmr/m/it/10.95 archimedean\OT1/cmr/m/n/10.95 ,
 []

<images/f016.png, id=210, 412.1799pt x 246.1998pt>
File: images/f016.png Graphic file (type png)
 <use images/f016.png>
[24] <images/f017.png, id=215, 183.3249pt x 128.3997pt>
File: images/f017.png Graphic file (type png)
 <use images/f017.png>
<images/f018.png, id=217, 317.2653pt x 219.4599pt>
File: images/f018.png Graphic file (type png)
 <use images/f018.png>
[25 <images/f016.png> <images/f017.png>]
<images/f019.png, id=227, 412.1799pt x 251.4996pt>
File: images/f019.png Graphic file (type png)
 <use images/f019.png>
[26 <images/f018.png>] [27 <images/f019.png>]
<images/f020.png, id=239, 396.0396pt x 322.5651pt>
File: images/f020.png Graphic file (type png)
 <use images/f020.png>
[28 <images/f020.png>] <images/f021.png, id=246, 438.9198pt x 100.4553pt>
File: images/f021.png Graphic file (type png)

<use images/f021.png> [29 <images/f021.png>]
<images/f022.png, id=253, 298.4751pt x 171.2799pt>
File: images/f022.png Graphic file (type png)
 <use images/f022.png>
Overfull \hbox (9.95847pt too wide) in paragraph at lines 2158--2158
 [] 
 []

[30 <images/f022.png>] <images/f023.png, id=260, 283.7802pt x 274.3851pt>
File: images/f023.png Graphic file (type png)

<use images/f023.png> <images/f024.png, id=262, 398.9304pt x 259.6902pt>
File: images/f024.png Graphic file (type png)

<use images/f024.png> <images/f025.png, id=264, 366.6498pt x 248.8497pt>
File: images/f025.png Graphic file (type png)

<use images/f025.png> [31 <images/f023.png> <images/f024.png>]
<images/f026.png, id=274, 283.7802pt x 219.4599pt>
File: images/f026.png Graphic file (type png)
 <use images/f026.png>
[32 <images/f025.png>] <images/f027.png, id=281, 289.08pt x 198.0198pt>
File: images/f027.png Graphic file (type png)

<use images/f027.png> [33 <images/f026.png> <images/f027.png>]
<images/f028.png, id=290, 455.0601pt x 284.9847pt>
File: images/f028.png Graphic file (type png)
 <use images/f028.png>
[34 <images/f028.png>] [35] [36]
LaTeX Font Info:    Font shape `OMS/cmr/m/n' in size <9> not available
(Font)              Font shape `OMS/cmsy/m/n' tried instead on input line 2548.

 [37]
<images/f029.png, id=313, 509.9853pt x 155.1396pt>
File: images/f029.png Graphic file (type png)
 <use images/f029.png>
[38 <images/f029.png>] <images/f030.png, id=320, 481.8pt x 104.3097pt>
File: images/f030.png Graphic file (type png)

<use images/f030.png> <images/f031.png, id=322, 230.3004pt x 194.1654pt>
File: images/f031.png Graphic file (type png)

<use images/f031.png> [39 <images/f030.png> <images/f031.png>]
<images/f032.png, id=331, 228.855pt x 160.6803pt>
File: images/f032.png Graphic file (type png)
 <use images/f032.png>
[40] <images/f033.png, id=336, 243.5499pt x 148.6353pt>
File: images/f033.png Graphic file (type png)
 <use images/f033.png>
[41 <images/f032.png> <images/f033.png>]
<images/f034.png, id=345, 216.81pt x 254.3904pt>
File: images/f034.png Graphic file (type png)
 <use images/f034.png>
Missing character: There is no � in font cmr10!
 [42 <images/f034.png>] <images/f035.png, id=352, 342.5598pt x 222.1098pt>
File: images/f035.png Graphic file (type png)

<use images/f035.png> [43] <images/f036.png, id=358, 440.3652pt x 191.2746pt>
File: images/f036.png Graphic file (type png)

<use images/f036.png> [44 <images/f035.png> <images/f036.png>] [45]
<images/f037.png, id=370, 401.5803pt x 222.1098pt>
File: images/f037.png Graphic file (type png)
 <use images/f037.png>
[46 <images/f037.png>] [47] <images/f038.png, id=380, 495.2904pt x 382.7901pt>
File: images/f038.png Graphic file (type png)

<use images/f038.png> [48] <images/f039.png, id=385, 398.9304pt x 377.4903pt>
File: images/f039.png Graphic file (type png)

<use images/f039.png> <images/f040.png, id=387, 483.2454pt x 344.0052pt>
File: images/f040.png Graphic file (type png)

<use images/f040.png> [49 <images/f038.png>] [50 <images/f039.png>]
<images/f041.png, id=400, 330.5148pt x 214.1601pt>
File: images/f041.png Graphic file (type png)
 <use images/f041.png>
[51 <images/f040.png>] <images/f042.png, id=410, 461.8053pt x 218.2554pt>
File: images/f042.png Graphic file (type png)

<use images/f042.png> [52 <images/f041.png> <images/f042.png>]
<images/f043.png, id=419, 286.4301pt x 208.8603pt>
File: images/f043.png Graphic file (type png)
 <use images/f043.png>
<images/f044.png, id=421, 381.3447pt x 236.8047pt>
File: images/f044.png Graphic file (type png)
 <use images/f044.png>
[53 <images/f043.png>] <images/f045.png, id=428, 503.2401pt x 334.6101pt>
File: images/f045.png Graphic file (type png)

<use images/f045.png> [54 <images/f044.png> <images/f045.png>]
<images/f046.png, id=437, 481.8pt x 305.2203pt>
File: images/f046.png Graphic file (type png)
 <use images/f046.png> [55]
[56 <images/f046.png>] <images/f047.png, id=448, 472.4049pt x 264.99pt>
File: images/f047.png Graphic file (type png)

<use images/f047.png> [57 <images/f047.png>] [58]
<images/f048.png, id=458, 522.0303pt x 294.3798pt>
File: images/f048.png Graphic file (type png)
 <use images/f048.png>
<images/f049.png, id=460, 492.3996pt x 232.9503pt>
File: images/f049.png Graphic file (type png)
 <use images/f049.png>
[59 <images/f048.png>] [60 <images/f049.png>] [61] [62] [63] [64] [65] [66]
[67] [68] [69] [70] <images/f050.png, id=507, 262.3401pt x 208.8603pt>
File: images/f050.png Graphic file (type png)

<use images/f050.png> [71] <images/f051.png, id=512, 428.3202pt x 270.2898pt>
File: images/f051.png Graphic file (type png)

<use images/f051.png> <images/f052.png, id=514, 287.6346pt x 220.9053pt>
File: images/f052.png Graphic file (type png)

<use images/f052.png> [72 <images/f050.png> <images/f051.png>] [73 <images/f052
.png>] [74] [75] [76] [77] [78] [79] [80] [81] [82] [83] [84] [85] [86]
[87] [1] [2] [3] [4] [5] [6] [7] [8] (hilbert.aux)

 *File List*
    book.cls    2004/02/16 v1.4f Standard LaTeX document class
    bk11.clo    2004/02/16 v1.4f Standard LaTeX file (size option)
 amsmath.sty    2000/07/18 v2.13 AMS math features
 amstext.sty    2000/06/29 v2.01
  amsgen.sty    1999/11/30 v2.0
  amsbsy.sty    1999/11/29 v1.2d
  amsopn.sty    1999/12/14 v2.01 operator names
 amssymb.sty    2002/01/22 v2.2d
amsfonts.sty    2001/10/25 v2.2f
tabularx.sty    1999/01/07 v2.07 `tabularx' package (DPC)
   array.sty    2003/12/17 v2.4a Tabular extension package (FMi)
 wrapfig.sty    2003/01/31  v 3.6
graphicx.sty    1999/02/16 v1.0f Enhanced LaTeX Graphics (DPC,SPQR)
  keyval.sty    1999/03/16 v1.13 key=value parser (DPC)
graphics.sty    2001/07/07 v1.0n Standard LaTeX Graphics (DPC,SPQR)
    trig.sty    1999/03/16 v1.09 sin cos tan (DPC)
graphics.cfg    2003/03/12 v1.1 MiKTeX 'graphics' configuration
  pdftex.def    2002/06/19 v0.03k graphics/color for pdftex
geometry.sty    2002/07/08 v3.2 Page Geometry
geometry.cfg
supp-pdf.tex
    umsa.fd    2002/01/19 v2.2g AMS font definitions
    umsb.fd    2002/01/19 v2.2g AMS font definitions
  omscmr.fd    1999/05/25 v2.5h Standard LaTeX font definitions
images/f001.png
images/f002.png
images/f003.png
images/f004.png
images/f005.png
images/f006.png
images/f007.png
images/f008.png
images/f009.png
images/f010.png
images/f011.png
images/f012.png
images/f013.png
images/f014.png
images/f015.png
images/f016.png
images/f017.png
images/f018.png
images/f019.png
images/f020.png
images/f021.png
images/f022.png
images/f023.png
images/f024.png
images/f025.png
images/f026.png
images/f027.png
images/f028.png
images/f029.png
images/f030.png
images/f031.png
images/f032.png
images/f033.png
images/f034.png
images/f035.png
images/f036.png
images/f037.png
images/f038.png
images/f039.png
images/f040.png
images/f041.png
images/f042.png
images/f043.png
images/f044.png
images/f045.png
images/f046.png
images/f047.png
images/f048.png
images/f049.png
images/f050.png
images/f051.png
images/f052.png
 ***********

 ) 
Here is how much of TeX's memory you used:
 2811 strings out of 95512
 31927 string characters out of 1189449
 94381 words of memory out of 1092525
 5660 multiletter control sequences out of 60000
 17931 words of font info for 69 fonts, out of 500000 for 1000
 14 hyphenation exceptions out of 607
 27i,16n,29p,219b,331s stack positions out of 1500i,500n,5000p,200000b,32768s
PDF statistics:
 597 PDF objects out of 300000
 0 named destinations out of 300000
 573 words of extra memory for PDF output out of 65536
<C:\texmf\fonts\type1\bluesky\cm\cmsy6.pfb><C:\texmf\fonts\type1\bluesky\cm\c
mr7.pfb><C:\texmf\fonts\type1\bluesky\cm\cmsy9.pfb><C:\texmf\fonts\type1\bluesk
y\cm\cmmi6.pfb><C:\texmf\fonts\type1\bluesky\cm\cmex10.pfb><C:\texmf\fonts\type
1\bluesky\cm\cmmi8.pfb><C:\texmf\fonts\type1\bluesky\symbols\msam10.pfb><C:\tex
mf\fonts\type1\bluesky\cm\cmsy8.pfb><C:\texmf\fonts\type1\bluesky\cm\cmmi10.pfb
><C:\texmf\fonts\type1\bluesky\cm\cmbx10.pfb><C:\texmf\fonts\type1\bluesky\cm\c
mbsy10.pfb><C:\texmf\fonts\type1\bluesky\cm\cmti9.pfb><C:\texmf\fonts\type1\blu
esky\cm\cmr8.pfb><C:\texmf\fonts\type1\bluesky\cm\cmti10.pfb><C:\texmf\fonts\ty
pe1\bluesky\cm\cmsy10.pfb><C:\texmf\fonts\type1\bluesky\cm\cmr9.pfb><C:\texmf\f
onts\type1\bluesky\cm\cmbx12.pfb><C:\texmf\fonts\type1\bluesky\cm\cmcsc10.pfb><
C:\texmf\fonts\type1\bluesky\cm\cmr6.pfb><C:\texmf\fonts\type1\bluesky\cm\cmr10
.pfb><C:\texmf\fonts\type1\bluesky\cm\cmr17.pfb><C:\texmf\fonts\type1\bluesky\c
m\cmr12.pfb><C:\texmf\fonts\type1\bluesky\cm\cmtt10.pfb>
Output written on hilbert.pdf (101 pages, 973287 bytes).
